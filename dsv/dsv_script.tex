% TODO next Iteration
% - ref-prefixes: sec, py, img, eq
% - refnumbers within section
% - \Tsys, \Xsig, \Ysig makros
% - factboxes
% - index terms
\documentclass[ngerman]{article}
%
\usepackage[fleqn]{amsmath}
\usepackage{amsfonts, amssymb, yhmath}
\usepackage{amsthm}
\usepackage{thmtools}
\usepackage{bm}
\usepackage{graphicx}
\usepackage[]{pgfplots}
\pgfplotsset{compat=newest}
\usepackage[]{glossaries}

\usepackage[]{hyperref}
\usepackage[]{cleveref}

%% end packages
%% 

\declaretheorem[
	numberwithin=section,
  title=Lemma,
  refname={lemma,lemmas},
  Refname={Lemma,Lemmas}
]{Lem}

\declaretheorem[
	numberwithin=section,
  title=Theorem,
  refname={theorem,theorems},
  Refname={Theorem,Theorems}
]{Thm}

\declaretheorem[
	numberwithin=section,
  title=Corollary,
  refname={corollary,corollaries},
  Refname={Corollary,Corollaries}
]{Cor}

\declaretheorem[
	numberwithin=section,
  title=Example,
  refname={example,examples},
  Refname={Example,Examples}
]{Exm}

\declaretheorem[
	numberwithin=section,
  title=Remark,
  refname={remark,remarks},
  Refname={Remark,Remarks}
]{Rem}

\declaretheorem[
	numberwithin=section,
  title=Algorithm,
  refname={algorithm,algorithms},
  Refname={Algorithm,Algorithms}
]{Alg}

\declaretheorem[
	numberwithin=section,
  title=Definition,
  refname={definition,definitions},
  Refname={Definition,Definitions}
]{Def}

\declaretheorem[
	numberwithin=section,
  title=Discussion,
  refname={discussion,discussions},
  Refname={Discussion,Discussions}
]{Dis}

\newcommand{\Ind}[1]{{\boldsymbol{\mathrm{#1}}}}

\newcommand{\trans}{\mathrm{T}}
\newcommand{\herm}{\mathrm{H}}
\newcommand{\forward}[2]{\bm{\phi}_{#1}(#2)}
\newcommand{\backward}[2]{\bm{\beta}_{#1}(#2)}

\newcommand{\R}{\mathbb{R}}
\newcommand{\C}{\mathbb{C}}
\newcommand{\K}{\mathbb{K}}
\newcommand{\N}{\mathbb{N}}
\newcommand{\Z}{\mathbb{Z}}
\newcommand{\PP}{\mathbb{P}}
\newcommand{\Ps}[1]{{\mathfrak{P}\left( #1\right)}}
\newcommand{\As}{\mathcal{A}}
\newcommand{\Hs}{\mathcal{H}}
\newcommand{\Fs}{\mathcal{F}}
\newcommand{\Gs}{\mathcal{G}}
\newcommand{\Ts}{\mathcal{T}}
\newcommand{\Li}{\mathcal{L}}
\newcommand{\RT}{{\mathscr{R}}}
\newcommand{\FT}{{\mathscr{F}}}
\newcommand{\BT}{{\mathscr{B}}}
\newcommand{\HT}{{\mathscr{H}}}

\newcommand{\Real}[1]{{\rm Re}\left\{#1\right\}}
\newcommand{\Imag}[1]{{\rm Im}\left\{#1\right\}}
\newcommand{\Conj}[1]{\overline{#1}}

\newcommand{\E}{\bm{\mathrm{E}}}
\newcommand{\Prb}{\bm{\mathrm{P}}}

\newcommand{\ScPr}[2]{{\left\langle #1,#2 \right\rangle}}
\newcommand{\Norm}[1]{{\left\Vert #1\right\Vert}}
\newcommand{\Abs}[1]{{\left| #1 \right|}}
\newcommand{\Text}[1]{{\hspace{3mm} \text{#1} \hspace{3mm}}}
\newcommand{\brac}[2]{{\left(\frac{#1}{#2}\right)}}
\newcommand{\br}[1]{{\left(#1\right)}}

\newcommand{\Int}[4]{{\int\limits_{#1}^{#2}{#3}\,\mathrm{d}{#4}}}
\newcommand{\Sum}[3]{{\sum\limits_{#1}^{#2}{#3}}}
\newcommand{\Prod}[3]{{\prod\limits_{#1}^{#2}{#3}}}
\newcommand{\SumB}[3]{{\left(\sum\limits_{#1}^{#2}{#3}\right)}}
\newcommand{\ProdB}[3]{{\left(\prod\limits_{#1}^{#2}{#3}\right)}}

\newcommand{\ConvA}[2]{{\xrightarrow[]{#1 \rightarrow #2}}}
\newcommand{\RA}[1]{\overset{#1}{\Rightarrow}}
\newcommand{\LRA}[1]{\overset{#1}{\Leftrightarrow}}
\newcommand{\V}[0]{\hspace{1.2mm}\middle\vert\hspace{1.5mm}}
\newcommand{\D}[0]{\hspace{0.5mm}:\hspace{1.0mm}}

\DeclareMathOperator*{\Argmin}{argmin}
\DeclareMathOperator*{\Argmax}{argmax}
\DeclareMathOperator*{\Arg}{arg}
\DeclareMathOperator*{\BlkDiag}{blkdiag}
\DeclareMathOperator*{\Conv}{conv}
\DeclareMathOperator*{\Cov}{Cov}
\DeclareMathOperator*{\Cone}{cone}
\DeclareMathOperator*{\Diag}{diag}
\DeclareMathOperator*{\id}{id}
\DeclareMathOperator*{\Ker}{ker}
\DeclareMathOperator*{\Median}{median}
\DeclareMathOperator*{\Max}{max}
\DeclareMathOperator*{\Mat}{mat}
\DeclareMathOperator*{\Min}{min}
\DeclareMathOperator*{\Mod}{Mod}
\DeclareMathOperator*{\Toep}{\bm{\mathrm{Toep}}}
\DeclareMathOperator*{\Tr}{tr}
\DeclareMathOperator*{\Rk}{rk}
\DeclareMathOperator*{\Ran}{ran}
\DeclareMathOperator*{\Sign}{sign}
\DeclareMathOperator*{\Sinc}{sinc}
\DeclareMathOperator*{\Supp}{supp}
\DeclareMathOperator*{\Sup}{sup}
\DeclareMathOperator*{\Span}{span}
\DeclareMathOperator*{\Spark}{spark}
\DeclareMathOperator*{\Surf}{surf}
\DeclareMathOperator*{\Vol}{vol}
\DeclareMathOperator*{\Pack}{pack}
\DeclareMathOperator*{\Pb}{\bm{\mathrm{P}}}
\DeclareMathOperator*{\Vectorize}{vec}
\DeclareMathOperator*{\Unif}{Unif}

\newcommand{\for}{\Text{for}}

\newcommand{\tx}{\rm tx}
\newcommand{\rx}{\rm rx}
\newcommand{\mycaption}[2]{\caption[#2]{\emph{#1} -- {#2}}}
\newcommand{\ilcode}[1]{\texttt{#1}}
\newcommand{\TODO}[1]{{\color{red} TODO:#1}}

\date{\today}
\author{Sebastian Semper -- FG Elektrische Messtechnik und Signalverarbeitung -- EMS}

\usepackage[T1]{fontenc}
\usepackage[utf8]{inputenc}

\usepackage{csquotes}

\usepackage{mdframed}
\newmdenv[]{factbox}

\usepackage[ngerman]{babel}
\usepackage{fullpage}
\usepackage{microtype}
\microtypesetup{
activate={true,nocompatibility},
final,
tracking=true,
kerning=true,
spacing=alltext-nott,
factor=1100,
stretch=25,
shrink=25,
letterspace=-30,
}
\usepackage{siunitx}

\usepackage{nameref}

\usepackage{morewrites}
\usepackage{pgfplots}
\pgfplotsset{compat=1.15}
\usepackage{mathrsfs}
\usetikzlibrary{arrows}

\usepackage{siunitx}

\usepackage[colorlinks,linktocpage,allcolors=Links]{hyperref}

\usepackage[sorting=none,
maxbibnames=99,
defernumbers=true,
style=numeric,
backref=true,
giveninits=true,]{biblatex}
\AtEveryBibitem{
 \clearfield{addendum}
 \clearfield{month}
 \clearfield{eprint}
 \clearfield{volume}
 \clearfield{isbn}
 \clearfield{issn}
 \clearfield{pages}
 \clearlist{location}
}
\addbibresource{../lib/sempersn.bib}

\renewcommand{\baselinestretch}{1.2}
\definecolor{Links}{RGB}{255,93,0}

\usepackage{cleveref}

\usepackage[acronym,hyperfirst,nomain]{glossaries}
\newacronym{3gpp}{3GPP}{Third Generation Partnership Project}

\newacronym{admm}{ADMM}{Alternating Directions of Multipliers Method}
\newacronym{anm}{ANM}{Atomic Norm Minimization}
\newacronym{adc}{ADC}{Analog-to-Digital Converter}
\newacronym{awgn}{AWGN}{Additive White Gaussian Noise}
\newacronym{asic}{ASIC}{Application Specific Integrated Circuit}
\newacronym{arpack}{ARPACK}{ARnoldi PACKage}
\newacronym{api}{API}{Application Programmable Interface}
\newacronym{aut}{AUT}{Antenna Under Test}
\newacronym[plural=AOI, firstplural=Areas of Interest (AOI)]{aoi}{AOI}{Area of Interest}
\newacronym{ai}{AI}{Artifical Intelligence}
\newacronym{aic}{AIC}{Akaike Information Criterion}

\newacronym{bp}{BP}{Basis Pursuit}
\newacronym{bpdn}{BPDN}{Basis Pursuit Denoising}
\newacronym{blue}{BLUE}{Best Linear Unbiased Estimator}

\newacronym{cnn}{CNN}{Convolutional Neural Network}
\newacronym{crb}{CRB}{Cramér-Rao Bound}
\newacronym{cs}{CS}{Compressed Sensing}
\newacronym{cf}{CF}{Crest factor}
\newacronym{cr}{CR}{Compression Ratio}
\newacronym{cmos}{CMOS}{Complementary Metal Oxide Semiconductor}
\newacronym{cots}{COTS}{Commercial-Off-The-Shelf}
\newacronym{cpu}{CPU}{Central Processing Unit}
\newacronym{cfar}{CFAR}{Constant False Alarm Rate}
\newacronym{cvx}{CVX}{Convex Optimization toolboX}
\newacronym{comp}{CoMP}{Cooperative Multi-Point}
\newacronym{cpcl}{CPCL}{Cooperative Passive Coherent Location}

\newacronym{doa}{DoA}{Direction of Arrival}
\newacronym{dod}{DoD}{Direction of Departure}
\newacronym[plural=DMC,firstplural=Dense Multipath Components (DMC)]{dmc}{DMC}{Dense Multipath Components}
\newacronym{das}{DAS}{Delay-and-Sum}
\newacronym{dft}{DFT}{Discrete Fourier Transform}
\newacronym{idft}{IDFT}{Inverse Discrete Fourier Transform}
\newacronym{dct}{DCT}{Discrete Cosine Transform}
\newacronym{dsp}{DSP}{Digital Signal Processor}
\newacronym{dnn}{DNN}{Deep Neural Network}

\newacronym{eadf}{EADF}{Effective Aperture Distribution Function}
\newacronym{etadf}{ETADF}{Effective Time-Aperture Distribution Function}
\newacronym{esprit}{ESPRIT}{Estimation of Signal Parameters via Rotational Invariance Techniques}
\newacronym{ett}{ETT}{Eigenvalue Threshold Test}
\newacronym{edc}{EDC}{Efficient Detection Criterion}
\newacronym{eft}{EFT}{Exponential Fitting Test}
\newacronym{expm}{EM}{Expectation Maximization}
\newacronym{ekf}{EKF}{Extended Kalman Filter}

\newacronym{fc}{FC}{fully-connected}
\newacronym{fht}{FHT}{Fast Hadamard Transform}
\newacronym[longplural={Fast Fourier Transforms}]{fft}{FFT}{Fast Fourier Transform}
\newacronym{fmcw}{FMCW}{Frequency-Modulated Continuous-Wave}
\newacronym{fpga}{FPGA}{Field Programmable Gate Array}
\newacronym{fri}{FRI}{Finite Rate of Innovation}
\newacronym{fir}{FIR}{Finite Impulse Response}
\newacronym{fim}{FIM}{Fisher Information Matrix}
\newacronym{fmc}{FMC}{Full Matrix Capture}
\newacronym{fista}{FISTA}{Fast Iterative Shrinkage-Thresholding Algorithm}
\newacronym{frvm}{FRVM}{Fast Relevance Vector Machine}
\newacronym{flop}{FLOP}{Floating Point Operation}
\newacronym{fl}{FL}{Federated Learning}

\newacronym{gfcs}{grid-free CS}{grid-free compressive sensing}
\newacronym{gpu}{GPU}{Graphical Processing Unit}
\newacronym{gtd}{GTD}{Geometrical Theory of Diffraction}
\newacronym{gan}{GAN}{Generative Adversarial Network}

\newacronym{hrpe}{HRPE}{High Resolution Parameter Estimator}
\newacronym{hfss}{Ansys HFSS}{Ansys High Frequency Electromagnetic Simulation Software}

\newacronym[longplural={Inverse Fast Fourier Transforms}]{ifft}{IFFT}{Inverse Fast Fourier Transform}
\newacronym{ir}{IR}{Impulse Response}
\newacronym{iid}{iid}{independent and identically distributed}
\newacronym{iir}{IIR}{Infinite Impulse Response}
\newacronym{irf}{IRF}{Impulse Response Function}
\newacronym{icas}{ICAS}{Integrated Communications and Sensing}
\newacronym{ista}{ISTA}{Iterative Shrinkage-Thresholding Algorithm}

\newacronym{kld}{KLD}{Kullback-Leibler Divergence}

\newacronym{ls}{LS}{Least Squares}
\newacronym{lasso}{LASSO}{Least Absolute Shrinkage and Selection Operator}
\newacronym{lse}{LSE}{Line Spectral Estimation}
\newacronym{lfsr}{LFSR}{Linear Feedback Shift Register}
\newacronym{lo}{LO}{Local Oscillator}
\newacronym{los}{LOS}{Line of Sight}
\newacronym{lti}{LTI}{Linear Time-Invariant}
\newacronym{lam}{LAM}{Large Area Monitoring}

\newacronym{mimo}{MIMO}{Multiple Input Multiple Output}
\newacronym{mmv}{MMV}{multiple measurement vectors}
\newacronym{ml}{ML}{maximum likelihood}
\newacronym{mmse}{MMSE}{misspecified mean squared error}
\newacronym{mse}{MSE}{mean squared error}
\newacronym{bce}{BCE}{Binary Crossentropy}
\newacronym{mlbs}{MLBS}{Maximum Length Binary Sequence}
\newacronym{mpc}{MPC}{Multipath Component}
\newacronym{msm}{MSM}{M-Sequence Method}
\newacronym{mwc}{MWC}{Modulated Wideband Converter}
\newacronym{mpm}{MPM}{Matrix Pencil Method}
\newacronym{mpu}{MPU}{Microprocessor Unit}
\newacronym{mri}{MRI}{Magnetic resonance imaging}
\newacronym{music}{MUSIC}{Multiple Signal Classification}
\newacronym{mkl}{MKL}{Math Kernel Library}
\newacronym{mcrb}{MCRB}{Misspecified Cramér-Rao Bound}
\newacronym{mmle}{MMLE}{Misspecified Maximum-Likelihood Estimator}
\newacronym{mbpe}{MBPE}{Model-Based Propagation Parameter Estimation}
\newacronym{mec}{MEC}{Mobile Edge Computing}
\newacronym{ndt}{NDT}{Nondestructive Testing}
\newacronym{nde}{NDE}{Nondestructive Evaluation}
\newacronym{nn}{NN}{Neural Net}
\newacronym{nist}{NIST}{National Institute of Standards and Technology}

\newacronym{omp}{OMP}{Orthogonal Matching Pursuit}
\newacronym{oop}{OOP}{Object Oriented Programming}
\newacronym{ofdm}{OFDM}{Orthogonal Frequency-Division Multiplexing}
\newacronym{ofdma}{OFDMA}{Orthogonal Frequency-Division Multiple Access}

\newacronym{pdp}{PDP}{Power Delay Profile}
\newacronym{pn}{PN}{Pseudo-Noise}
\newacronym{pwc}{PWC}{Plane Wave Compounding}
\newacronym{pcl}{PCL}{Passive Coherent Location}
\newacronym{pwi}{PWI}{Plane Wave Imaging}
\newacronym{pura}{PURA}{Patch Uniform Rectangular Array}
\newacronym{pymax}{PyMAX}{Python Maximization Approach}
\newacronym{pts}{PTS}{Pseudo-True Solution}
\newacronym{pdf}{pdf}{probability density function}
\newacronym{pi}{PI}{Principal Investigator}

\newacronym{ran}{RAN}{Radio Access Network}
\newacronym{ranic}{RIC}{RAN Intelligent Controller}
\newacronym{relu}{ReLU}{Rectified Linear Unit}
\newacronym{resnet}{ResNet}{Residual Neural Network}
\newacronym{ram}{RAM}{Random Access Memory}
\newacronym{rcs}{RCS}{Radar Cross Section}
\newacronym{rd}{RD}{Random Demodulator}
\newacronym{rx}{RX}{receiver}
\newacronym{rem}{REM}{Reconstruction Error Metric}
\newacronym{rmse}{RMSE}{Root Mean Squared Error}
\newacronym{rms}{RMS}{root mean squared}
\newacronym{ric}{RIC}{Restricted Isometry Constant}
\newacronym{rip}{RIP}{Restricted Isometry Property}
\newacronym{rc}{RC}{Raised Cosine}
\newacronym{roi}{ROI}{Region of Interest}
\newacronym{rimax}{RIMAX}{Richter Maximization Approach}
\glsunset{rimax}
\newacronym{rvm}{RVM}{Relevance Vector Machine}

\newacronym{scf}{SCF}{spatial correlation function}
\newacronym{samurai}{SAMURAI}{Synthetic Aperture Measurements of Uncertainty in Angle of Incidence}
\newacronym[plural=SC,firstplural=Specular Components (SC)]{sc}{SC}{Specular Components}
\newacronym{sdp}{SDP}{semi-definite program}
\newacronym{sdr}{SDR}{Signal to Diffuse Ratio}
\newacronym{simd}{SIMD}{Single Instruction Multiple Data}
\newacronym{svd}{SVD}{singular value decomposition}
\newacronym{svm}{SVM}{Support Vector Machine}
\newacronym{soe}{SOE}{Sparsity Order Estimation}
\newacronym{sgd}{SGD}{Stochastic Gradient Descent}
\newacronym{stuca}{StUCA}{Stacked Uniform Circular Array}
\newacronym{spucpa}{SPUCPA}{Stacked Polarimetric Uniform Circular Patch Array}
\newacronym{suca}{SUCA}{Stacked Uniform Circular Array}
\newacronym{saft}{SAFT}{Synthetic Aperture Focusing Technique}
\newacronym{sota}{SOTA}{State of the Art}
\newacronym{ssd}{SSD}{Solid State Device}
\newacronym{ssr}{SSR}{Sparse Signal Recovery}
\newacronym{sa}{SA}{Synthetic Aperture}
\newacronym{spw}{SPW}{Single Plane Wave}
\newacronym{shm}{SHM}{Structural Health Monitoring}
\newacronym{snr}{SNR}{Signal-to-Noise Ratio}
\newacronym{stela}{STELA}{Soft-Thresholding with Exact Line Search Algorithm}
\newacronym{siso}{SISO}{Single Input Single Output}
\newacronym{simo}{SIMO}{Single Input Multiple Output}
\newacronym{swe}{SWE}{Spherical Wave Expansion}
\newacronym{sme}{SME}{Spherical Mode Expansion}
\newacronym{sage}{SAGE}{Space-Alternating Generalized Expectation-Maximization}

\newacronym{th}{T\&H}{Track and Hold}
\newacronym{tf}{TF}{Transfer Function}
\newacronym{tx}{TX}{transmitter}
\newacronym{twista}{TWISTA}{Two-step Iterative Shrinkage-Thresholding Algorithm}
\newacronym{tof}{ToF}{Time-of-Flight}

\newacronym{uca}{UCA}{uniform circular array}
\newacronym{ura}{URA}{Uniform Rectangular Array}
\newacronym{ula}{ULA}{Uniform Linear Array}
\newacronym{uwb}{UWB}{Ultra-Wideband}
\newacronym{usndt}{US-NDT}{Ultrasonic Non-destructive Testing}

\newacronym{vna}{VNA}{Vector Network Analyser}
\newacronym{vsh}{VSH}{Vector Spherical Harmonics}
\hyphenation{op-tical}
\hyphenation{net-works}
\hyphenation{semi-conduc-tor}

\makeglossaries


\usepackage{placeins}
\usepackage{etoolbox}

\numberwithin{equation}{subsection}

\usepackage{minted}
\setminted{fontsize=\footnotesize}
\setminted{escapeinside=||}

\newcommand{\code}{Codeschnipsel}
\renewcommand\listingscaption{\code}
\renewcommand\listoflistingscaption{\code{verzeichnis}}
\crefname{listing}{\code}{\code}  
\Crefname{listing}{\code}{\code}

\usepackage{helvet}
\usepackage{mathpazo}
\usepackage{sourcecodepro}

\usepackage{esint}
\usepackage{mathtools}

\newcommand{\Start}[1]{\underset{\uparrow}{#1}}
\newcommand{\z}{{\rm z}}
\newcommand{\fuer}{\Text{f\"ur}}

\newcommand{\q}[1]{{\glqq{}#1\grqq{}}}
\newcommand{\myemph}[1]{\emph{#1}}
\newcommand{\codecaption}[2]{
  \caption[#2]{%
    % \backrefsetup{disable}%
    #2, % 
    siehe \href{https://github.com/SebastianSemper/lecturenotes/blob/main/#1}{%
        \path{#1}%
      }%
    % \backrefsetup{enable}%
  }%
}

\crefname{Thm}{Theorem}{Theorems}

\declaretheorem[
  numberwithin=section,
  title=Beispiel,
  refname={Beispiel,Beispiele},
  Refname={Beispiel,Beispiel}
]{Bsp}

\usepackage[shortlabels]{enumitem}

% widows and orphans
\widowpenalty=1000
\clubpenalty=1000

\newcommand{\glshead}[1]{\texorpdfstring{\acrshort*{\lowercase{#1}}}{#1}}
\newcommand{\ENDDOCUMENT}{
\clearpage
\microtypesetup{protrusion=false}
\addcontentsline{toc}{section}{Abk"urzungsverzeichnis}
\printglossary[type=\acronymtype]
\microtypesetup{protrusion=true}
%
%
%
\clearpage
\microtypesetup{protrusion=false}
\addcontentsline{toc}{section}{Literaturverzeichnis}
\printbibliography
\microtypesetup{protrusion=true}
\end{document}
}
%
\title{Skript Digitale Signalverarbeitung}
%
\begin{document}
\pagenumbering{roman}
\glsunsetall
%
\maketitle
\tableofcontents
\clearpage
%
\listoffigures
\addcontentsline{toc}{section}{Abbildungsverzeichnis}
\listoflistings
\addcontentsline{toc}{section}{Code-Schnipsel-Verzeichnis}
\clearpage
%
\pretocmd{\section}{\FloatBarrier\clearpage}{}{}
\pretocmd{\subsection}{\FloatBarrier}{}{}
%
\glsresetall
\setcounter{page}{1}
\pagenumbering{arabic}

\begin{center}
    \begin{minipage}{0.6\textwidth}
        Die Idee hinter dem Skript zur Vorlesung ist, dass es die Zuh\"orer der B\"urde des Mitschreibens entledigt und Zeit und Platz zum Folgen der Vorlesung frei macht.
        Das Skript sollte deshalb immer zur Vorlesung und \"Ubung mitgebracht und im Idealfall mit Notizen versehen werden, bzw. zum Nachschlagen verwendet werden.
    \end{minipage}
\end{center}
%
%
\section{Theoretische Grundlagen}\label{basics}
%
%
Digitale Signalverarbeitung ist ein Feld, das sich vieler verschiedener mathematischer Grundlagen bedient, um die gefundenen Zusammenhänge rigoros, knapp und gleichzeitig elegant zu formulieren.
Deshalb kommen wir nicht umhin, uns einiger dieser Grundlagen zu erinnern. 
Alles hier knapp aufgelistete sollte schon bekannt sein und dient nur als bequemes Nachschlagewerk für das kommende Semester.
%
\subsection{Komplexe Zahlen}
%
Die \emph{komplexen Zahlen} $\C$ sind die Menge aller $z = x + \jmath y$, wobei $x,y \in \R$ und für die imaginäre Einheit $\jmath$ gilt, dass $\jmath^2 = -1$.
Wir nutzen hier speziell $\jmath$ in Abgrenzung zu $i$ oder $j$, da diese oft als Indices oder Laufvariablen auftreten.
Bei $z = x + \jmath y$ nennen wir $x =\Re(z)$ den Realteil und respektive $y = \Im(z)$ den Imaginärteil.
Komplexe Zahlen lassen sich auch in der Polarform $z = r \exp(\jmath \phi)$ darstellen, wobei $r = \Abs{z} = \sqrt{\Re(z)^2 + \Im(z)^2} \geq 0$ den Betrag und $\phi = \angle(z) = \arctan(y,x)$ das Argument von $z$ darstellen.
Die zu $z = r \exp(\jmath \phi) = x + \jmath y$ komplex konjugierte Zahl ist $z^\ast = r \exp(-\jmath \phi) = x - \jmath y$.

Komplexe Zahlen haben viele interessante Eigenschaften und Anwendungen, vor allem in der digitalen Signalverarbeitung.
Beispielsweise für die Darstellung von einem modulierten reellen Passband Signal $s \D \R \rightarrow \R$ dargestellt durch
\[
s(t) = x(t)\cos(\omega t) + y(t)\sin(\omega t) \in \R,
\]
die "aquivalente Darstellung im komplexen Basisband
\begin{equation}\label{complex_baseband}
    s_B(t) = x(t) + \jmath y(t) \Text{mit} \Re(s_B(t)\exp(\jmath \omega t)) = s(t).
\end{equation}

existiert. Man sagt auch, dass $s_B \exp(\jmath \omega \cdot)$ das analytische Signal zu $s$ darstellt. Dass komplexe Zahlen viele überaschungen bereithalten sieht man wenn man simuliert und visualisiert für welche $c \in \C$ die Folge
\[
z_{n+1} = z_{n}^2 + c \Text{mit} z_0 = c
\]
konvergiert oder divergiert (Übung).
%
%
\subsection{Signale}
%
\subsubsection{Definition und Typen}
%
Wir haben gerade schon von Signalen gesprochen, ohne sie etwas genauer einzuführen. 
Ganz allgemein kann man sich Signale als Objekte vorstellen, die abhängig von Raum, Zeit, oder beidem, physikalische Messgrößen, wie Spannungen, Feldstärken, oder Temperaturen modellieren/abbilden.
Es kann aber auch vorkommen, dass das Signal durch gewissen Transformationen inhärent einheitenlos is.

Die theoretische Darstellung von Signalen erfolgt durch \emph{Funktionen}, die dann noch mit einer impliziten Interpretation versehen werden.
Eine Funktion $s : D \rightarrow B$ besitzt einen Namen ($s$), einen Definitionsbereich ($D$) und einen Bildbereich ($B$).
Hierbei sind $D$ und $B$ zunächst irgendwelche Mengen. 
Die Funktion $s$ bildet nun Paare $(d,b)$ zwischen Mengenelementen von $d \in D$ und $b \in B$, indem man schreibt $(d, s(d))$, oder $d \mapsto s(d) = b$.
Der Witz ist nun, dass man ein Signal mit physikalischer Bedeutung erhält, indem man lediglich $D$ und $B$ geschickt wählt, sodass diese Mengen zu der Interpretation passen, die man im Hinterkopf hat.

Ist $D = B = \R$ so sprechen wir von einem reellen Signal $s$ und meist denken wir dabei bei $D$ an die Zeitachse, weshalb wir auch $s \mapsto s(t)$, oder einfach $s(\cdot)$ schreiben. 
Wir werden im Folgenden fast ausschließlich von der \q{Zeitachse} sprechen, aber alle Überlegungen lassen sich beispielsweise auf den Raum, oder andere Bereiche übertragen, wie wir beispielsweise in \Cref{sec:eadf} sehen werden.
Ist $D = \R^3$, $B = \R$, so denken wir meist an den dreidimensionalen Raum für den Definitionsbereich und haben also ein Signal im Raum gegeben.

Ist nun jedoch $D = \Z$, $B = \R$, so ist das Signal nur für die ganzen Zahlen $\Z$ definiert, weshalb wir dann mathematisch von einer Folge, oder in unserem Fall von einem \emph{Zeitdiskreten Signal} sprechen.
Meist schreiben wir hierfür kurz $s[k] \in \R, k\in \Z$, oder einfach $s[\cdot]$ im Unterschied.
Man soll sich hier nicht vorstellen, dass die Werte \q{zwischen} den ganzen Zahlen nur fehlen würden. 
So ist dies \emph{nicht} zu verstehen. 
Zwischen den gegeben Werten ist keine Information vorhanden -- es gibt also nichts, was \q{fehlen} könnte!
In manchen Situationen werden wir die diskreten Signale explizit aufschreiben. 
Dann markieren wir die Stelle $k = 0$ via
\[
\dots, 0, 1, 2, \Start{3}, 2, 1, 0, \dots,
\]
um eine bequeme Schreibweise für solche Folgen (diskreten Signale) zu erhalten. 

Versuchen sie für möglichst viele verschiedene Kombinationen von $D$ und $B$ Beispiele zu finden (Übung).
%
\subsubsection{Signale als Vektoren}\label{sec:signals_vec}
%
Um mit Signalen gut umgehen zu können, ist es wichtig ihre Eigenschaften als mathematische Objekte zu kennen.
Wir müssen Signale also mit \q{Struktur} versehen.
Wir gehen hier zunächt der Einfachheit halber von $D = B = \R$ aus.

Intuitiv stellt man sich beispielsweise vor, dass man Signale in ihrer Intensität verändern können sollte, und für beliebige änderung der Intensität wieder ein Signal erhält.
Definiert man für $a \in \R$ das Objekt $a \cdot s$ als $t \mapsto a \cdot s(t)$ oder bildet nun die Paare $(t, a \cdot s(t))$ so erhält man wieder ein Signal.
Die Werte von $s$ werden also einfach punktweise durch das Skalar $a$ skaliert.
Betrachtet man nun zwei Signale $s_1, s_2$ und definiert $s_1 + s_2$ als $t \mapsto s_1(t) + s_2(t)$, oder $(t, s_1(t) + s_2(t))$, so erhalten wir die Summe oder die Superposition von $s_1$ und $s_2$.
Da Signale oft physikalische Messgrößen darstellen, macht dies auch oft Sinn, da in der Physik das Prinzip der Superposition oft eine Rolle spielt.
Superposition tritt meist dann auf, wenn die Messgrößen, die wir beobachten, auch als Lösung einer Differentialgleichung gefunden werden können.
Bei Differentialgleichungen gilt nämlich oft das Superpositionsprinzip im Lösungsraum, was bedeutet, dass für zwei Lösungen $x_1$ und $x_2$ deren Summe $x_1 + x_2$ auch wieder eine Lösung ist.

Wenn wir die beiden Fakten nun kombinieren erhalten wir für $a_1, a_2 \in \R$ und zwei Signale $s_1, s_2$ und deren Linearkombination $a_1 s_1 + a_2 s_2$, die mit
\[
(a_1 s_1 + a_2 s_2)(t) = a_1 s_1(t) + a_2 s_2(t)
\]
wieder ein Signal definiert.
Objekte, die diese Eigenschaft haben, nennt man \emph{Vektoren} und diese leben in einem \emph{Vektorraum}.
Das mag erstmal nicht so schockieren, aber wir gewinnen dadurch \emph{alle} Werkzeuge aus der linearen Algebra für unsere Zwecke.

Beispielsweise können wir nun geschickt Bausteine für eine gewisse Untermenge von Signalen finden, mit denen sich die Signale in dieser Untermenge gut und informativ beschreiben lassen.
Wir könnten uns fragen, ob es für den Vektorraum der Bild-Signale eine Basis gibt, sodass für jedes Bild $b$ eine Darstellung existiert, dass
\begin{equation}\label{eq:basics:image_base}
    b(x,y) = c_1 b_1(x,y) + c_2 b_2(x,y) + \dots,
\end{equation}
gilt. 
Die \emph{Zahlen} $c_1, c_2, \dots$ können also das Signal $b$ darstellen, indem man einfach die Elemente aus der Basis hernimmt, entsprechend skaliert und summiert.
Das dürfen wir, weil auch die Basisvektoren, eben \emph{Vektoren} sind.
In gewisser Weise \emph{sind} die Koeffizienten $c_i$ das Signal $b$.
Vielleicht gelingt es uns, die Menge $\{b_1, b_2, \dots\}$ so zu konstruieren, dass wir immer nur \emph{wenige} von diesen $b_i$ brauchen, sodass wir \emph{jedes beliebige} Bild aus einer Fotokamera durch geschickte Kombination von diesen $\{b_1, b_2, \dots\}$ darstellen können (*funnyMP3noises*)?
%
\subsubsection{Transformation von Signalen}
%
Noch interessanter ist aber die Manipulation von Signalen durch \emph{Transformationen}.
Der Sinn von Transformationen ist es, neue oder einfach bestimmte Einsichten in ein Signal zu gewinnen.
Es kann aber auch sein, dass man Operationen, die auf Signalen ausgeführt werden sollen, \q{einfach} mittransformieren kann.
Vielleicht ist die gewünschte Operation nach Transformation deutlich einfacher anzuwenden?
Jede Transformation liefert hierbei andere Informationen oder ist für andere Signale definiert.

Mathematisch ist eine Transformation nichts anderes als eine Abbildung zwischen Signalen.
Definieren wir uns zwei Mengen von Signalen $\mathcal{S}$ und $\mathcal{T}$.
Dann bildet die Transformation $T: \mathcal{S} \rightarrow \mathcal{T}$ Paare zwischen Objekten aus $\mathcal{S}$ und $\mathcal{T}$, also $s \in \mathcal{S} \mapsto Ts \in \mathcal{T}$.
Nach Anwendung der Transformation $T$ auf $s$ erhalten wir also ein anderes Signal $Ts$.
Gerade haben wir schon festgestellt, dass man Signale beliebig skalieren und addieren kann und es als eine Art grundlegende Eigenschaft von Signalen festgehalten.
Nehmen wir nun ein Signal mit Werten
\[
    s(t) = a_1 s_1(t) + a_2 s_2(t)
\] 
und wir wenden die Transformation $T$ auf beiden Seiten der Gleichung an
\[
    \{Ts\}(t) = \{T (a_1 s_1 + a_2 s_2)\}(t).
\]
Ist nun die Transformation so, dass wir für beliebige $a_{1,2}$ und $s_{1,2}$ immer
\begin{equation}\label{eq:basics:trafo_linear}
    \{Ts\}(t) = \{T (a_1 s_1 + a_2 s_2)\}(t) = a_1 \{Ts_1\}(t) + a_2 \{Ts_2\}(t)
\end{equation}
schreiben können, so nennen wir $T$ eine \emph{lineare} Transformation.
Zusammen mit der Superpositionseigenschaft von Signalen sieht man nun, warum
Linearität so wichtig für Transformationen ist, weil es einfach zur Vektorraumstruktur von Signalen passt.
Die Linearität erlaubt es uns beispielsweise auch das obige Signal $b$ in \eqref{eq:basics:image_base} ganz einfach zu transformieren.
Nehmen wir es in seiner Darstellung als
\[
    b(x,y) = c_1 b_1(x,y) + c_2 b_2(x,y) + \dots,
\]
und wir haben eine beliebige lineare Transformation $T$, deren Effekt wir auf $b$ angewendet sehen wollen. 
Wir suchen also $\{Tb\}(x,y)$.
Aber das ist mit der Linearität ganz einfach. Wir müssen nur $Tb_i$ kennen, also die Wirkung von $T$ auf die Basisvektoren $b_i$, denn
\[
    \{Tb\}(x,y) = c_1 \{Tb_1\}(x,y) + c_2 \{Tb_2\}(x,y) + \dots,
\]
ist eine valide Darstellung von $Tb$.
Cool!

Beispiele für solche linearen Transformationen sind Differentiation (falls möglich), bilden der Stammfunktion (falls möglich), Verzögerung eines Zeitsignals um Zeit $a \in \R$, Stauchung und Streckung in eines Zeitsignals, Rotation eines Bildes, die Fourier-Transformation, die diskrete Fourier-Transformation, zyklische Faltung, oder Korrelation mit einem anderen Signal $p$.
Gegenbeispiele sind $p(t) = \sin(s(t))$, oder $p(t) = (s(t))^\alpha$ für $\alpha \neq 1$, oder $s \ast s$, wobei $\ast$ die Faltung darstellt.

Man sieht, dass viele wichtige Operationen lineare Transformationen darstellen und wir haben mit linearer Algebra ein mächtiges Tool an unserer Seite, um mit ihnen umzugehen.
%
%
\subsubsection{Zufällige Signale}
%
Man kann auch noch eine weitere Sichtweise auf Signale haben. 
In manchen Fällen ist es nicht zweckmäßig, dass man ein Signal $s$ als vollständig bekannte und fixe Funktion modelliert.
Stattdessen modelliert man den Wert $s(t)$ des Signals $s$ an der Stellen $t$ als \emph{Zufallsgröße}.
Das heißt, dass der Wert $s(t)$ einer Verteilung $X(t)$ folgt. 
An jedem Zeitpunkt $t$ \q{hängt} eine solche Verteilung, die bestimmt mit welcher Wahrscheinlichkeit die Werte $s(t)$ in einem gewissen Intervall liegen.
Man spricht in diesem Fall auch von \emph{stochastischen} Signalen, im Gegensatz zu den obigen \emph{deterministischen} Signalen.

Es kann verschiedene Gründe haben, dass man ein Signal nicht mehr deterministisch beschreiben kann/will/sollte:
%
\begin{itemize}
    \item Sobald die Werte von $s$ durch eine Messung entstanden sind, enthalten diese normalerweise Messrauschen.
    Dann modelliert man $s$ meistens als Summe
    \[
    s(t) = x(t) + n(t),
    \]
    wobei $n(t) \sim \mathcal{N}(0, \sigma^2(t))$ meist als eine Realisierung einer mittelwertfreien Normalverteilung mit Varianz $\sigma^2(t)$ angenommen wird, und $x$ als ein deterministisches Signal.
    \item Wenn man generell nicht genug Information über das Signal hat, beispielsweise, kennt man nur dessen Verteilung im Frequenzbereich, also Wahrscheinlichkeiten, dass gewisse Frequenzen vorhanden sind, oder nicht.
    Dennoch ist man natürlich an dem Verhalten des Signals im Zeitbereich interessiert.
    \item Wenn es für die Anwendung nicht notwendig ist.
    Dies kann der Fall sein, wenn man einen Filter entwickelt, der eine gewisse Klasse von Signalen als Eingang bekommt, kann es reichen die Verteilung der Signale zu kennen und dann den Ausgang des Filters nur stochastisch zu beschreiben.

    \q{In \SI{99.99}{\percent} der Fälle ist der nachgeschaltete Verstärker nicht übersteuert.}

    Für solche Aussagen ist es sogar \emph{notwendig} die Verteilung der Eingangssignale zu kennen, ansonsten ist so eine Aussage gar nicht möglich, da man eben keine Verteilung f"ur ein deterministisches Signal angeben kann.
\end{itemize}
%
%
Um stochastische Signale korrekt handhaben zu können, ist einige Mathematik notwendig, die wir in \Cref{sec:random} lediglich skizzieren werden und stattdessen versuchen ein \emph{intuitives} Verständnis zu entwickeln.
%
%
\subsubsection{Spezielle Signale}\label{sec:spec_sig}
%
Uns werden immer wieder einige spezielle Signale begegnen, die wir hier kurz auflisten wollen.
\begin{itemize}
    \item Die \emph{Delta-Funktion (Dirac-$\delta$)} als Funktional (eine Abbildung, die Paare von Funktionen und Skalaren bildet) $\delta$, das angewendet auf ein Signal $s$, immer den Wert $s(0)$ liefert. 
    Visualisiert wird dieses \emph{nicht}-Signal, durch einen Impuls der Höhe $1$ bei $t = 0$. 
    Es ist nicht ohne Ironie, dass eines der wichtigsten Objekte der Signalverarbeitung selbst kein Signal ist, jedoch wie eines behandelt wird, aber dann immer mit Vorsicht.
    \item Die \emph{Heavyside-Funktion} $u : \R \rightarrow \R$ mit
    \[
        u(t) = \begin{cases}
            1 \Text{für} t>0 \\
            \frac{1}{2} \Text{für} t = 0 \\
            0 \Text{für} t < 0.
        \end{cases}
    \]
    Man kann $\delta$ als distributionelle Ableitung von $u$ auffassen.
    \item Die \emph{komplexe Schwingung} $s : \R \rightarrow \C$ bei Frequenz $f \in \R$ is definiert als 
        $s_f(t) = \exp(\jmath f t)$
    und wir uns im Verlauf des Semesters noch einige Male begegnen. 
    Beispielsweise gelten $s_f^\ast(t) = s_f(-t)$ und $s_{-f}(t)^\ast = s_f(t)$.
    \item Der \emph{diskrete $\delta$-Stoß} $\delta[k]$ ist definiert als
    \[\dots, 0, \Start{1}, 0, \dots\]
    \item Endliche, diskrete Signale können wir entweder durch
    \[
        s = [0,1,2,\Start{3},2,1,0]
    \]
    darstellen, oder als endliche Summe von einigen diskreten $\delta$-Stößen:
    \[
        s[n] = \Sum{k=-2}{k=+2} s[k] \delta[n - k]
    \]
\end{itemize}
%
%
\subsubsection{Beispiel: LTI-Systeme}\label{exm:basics:cont_lit}
Wir werden uns zwar noch später ausführlich mit \gls{lti} Systemen beschäftigen, doch sie sollen hier schon als nicht-triviales Beispiel dienen.
Wir sind also mit einem System $\mathcal{H}$.
Ein System ist ein Objekt, das Eingangssignale auf Ausgangssignale abbildet, also Paare von Funktionen/Signalen $(x, \mathcal{H}x)$ oder $(x, y)$ bildet.

Weiterhin hat $\mathcal{H}$ die Eigenschaft, dass 
für Anregungen $x : \R \rightarrow \R$ eine Verschiebungsinvarianz mit $y(t - \tau) = (\mathcal{H}x(\cdot  - \tau))(t)$ gilt -- es ist \emph{zeitinvariant}.
Außerdem ist $\mathcal{H}$ \emph{linear} wie in \eqref{eq:basics:trafo_linear} dargestellt.

Dann kann man die Wirkung von $\mathcal{H}$ auch durch Faltung mit der sog. Impulsantwort $h$ des Systems darstellen, also
%
\begin{equation}\label{lti-conv}
    y(t) 
        = (\mathcal{H}x)(t) 
        = \Int{-\infty}{+\infty}{x(t - \tau) h(\tau)}{\tau} 
        = (x \ast h)(t),
\end{equation}
%
wobei $h = \mathcal{H}\delta$, also die Reaktion des Systems auf einen Dirac-Stoß darstellt.
An dieser Darstellung sieht man sehr gut, dass das System $\mathcal{H}$ linear ist, weil die Integration linear in $x$ ist.

Natürlich ist der Zeitbereich für diese Art von System nicht der richtige Anschauungsort. 
Nach Laplace-Transformation von $y$ zu $Y = \mathcal{L}y$ sehen wir, dass wir dort stattdessen 
\[
Y(s) = X(s) \cdot H(s),
\]
schreiben können. 
Hierbei sind $X = \mathcal{L}x$ und $H = \mathcal{L}h$ die Laplace-Transformationen des Eingangs und der Impulsantwort $h$.
Nicht nur hat sich die \q{Berechnung} von $Y$ vereinfacht, sondern wir haben auch ein besseres Gefühl für das Verhalten des Systems in Abhängigkeit von $h$, bzw. $H$, weil der Einfluss einfach multiplikativ ist.

Wir können die lineare Algebra noch ein wenig weiter treiben. Betrachten wir als Eingang die Funktion $x_s(t) = \exp(s t)$ für ein beliebiges $s \in \C$.
Dann rechnen wir einfach mit \eqref{lti-conv} nach, dass
\[
(H\exp(s\cdot))(t) 
    = \Int{-\infty}{+\infty}{\exp(s (t-\tau))h(\tau)}{\tau}
    = \exp(s t) \Int{-\infty}{+\infty}{\exp(-s\tau)h(\tau)}{\tau}
    = \exp(s t) H(s),
\]
gilt. Das heißt, dass die Funktionen $\exp(s \cdot)$ die \emph{Eigenvektoren} des Operators $\mathcal{H}$ sind, denn es gilt 
\[
    (\mathcal{H} x_s)(t) = x_s(t) \cdot H(s),
\]
wobei $H$ die Laplace-Transformation von $h$ ist.
Das heißt auch, dass $H(s)$ die zugehörigen \emph{Eigenwerte} sind\footnote{\url{https://imgflip.com/memetemplate/16005384/mind-blown}}.
Wir sehen hier also, dass Signale \emph{wirklich} wie Vektoren funktionieren können und es sich im Fall von linearen System förmlich aufzwingt, da die Linearität des Systems zur linearen Vektorraumstruktur \q{passt}.

%
%
\section{Abtastung von Signalen}\label{sec:sampling}
%
%
Als ersten Schritt der digitalen Signalverarbeitung wollen wir uns den \"Ubergang von einem analogen Signal zu einem digitalen n\"aher ansehen.
Intuitiv k\"onnen wir diesen Vorgang in vielen Anwendungen beobachten.
Wir nehmen im Tonstudio mit einem Mikrophon Ton auf und eine Soundkarte wandelt das analoge Signal in einen WAV-Datenstrom um.
In einer Fotokamera, trifft ein Feld von Lichtstrahlen ein und wird von einem \gls{cmos}-Sensor \q{direkt} abgetastet und in Helligkeitswerte pro Farbkanal umgewandelt.
Eine Antenne wandelt ein anliegendes elektro-magnetisches Feld in eine Spannung um, welche nachtr\"aglich von einem \gls{adc} abgetastet und quantisiert wird.

Mathematisch modellieren wir analoge Signale $s_a : D \rightarrow B$ meist mit $D$ und $B$, die auf die reellen Zahlen $\R$ zur\"uckgreifen.
Die Wandlung von analog zu digital transformiert dieses Signal in eine Funktion $s: \Z \rightarrow Q$ um, wobei auch $\Abs{Q} < \infty$ gilt.
Das hei\ss{}t, dass das Signal nach AD-Wandlung nur noch endliche Werte annehmen kann und, dass es nur noch aus eine \emph{Folge} von Werten aus der Menge $Q$ besteht. 
Es wurde also zeit- und wertdiskretisiert.
Wir werden uns zun"achst nur mit der Diskretisierung in Zeit befassen, weil es einfach einfacher ist.
Das hei"st, dass wir uns vorstellen, dass das diskretisierte Signal nur an einer diskreten Menge an Punkten noch Informationen "uber das abgetastete Signal beinhaltet.
Weiterhin sind wir nicht an der physikalischen Umsetzung von \glspl{adc} interessiert, sondern h"ochstens an deren systemtheoretischer Modellierung.

Die zentralen Fragen sind nun:
\begin{itemize}
    \item Wie muss der Vorgang der Abtastung gestaltet sein, dass keine Information verloren geht?
    \item Wie k\"onnen wir die Eigenschaften des analogen Signals in dessen abgetasteter Version wiederfinden?
    \item Welche Operationen k"onnen wir auf digitalen Signalen wie effizient ausf"uhren?
\end{itemize}
%
\subsection{Frequenz von Signalen}
%
\subsubsection{Zeit-Kontinuierliche Harmonische}
%
Meistens werden wir uns in der Vorlesung mit reell- oder komplexwertigen Zeitsignalen befassen, d.h. wir modellieren unsere Signale als $x_a : \R \rightarrow \R$ oder $x_a: \R \rightarrow \C$.
Wobei physikalische Signale nat"urlich nur reellwertig sind, doch manchmal ist die Darstellung als komplexwertige Funktion besser handhabbar, siehe \eqref{complex_baseband}.
Das hei"st, dass die Abtastung im Zeitbereich vonstatten geht, was sofort den Begriff der \emph{Frequenz} auf den Plan ruft, da Frequenz mit Einheit $1/[s]$ eng mit Zeit $[s]$ verkn"upft ist.

Betrachten wir also erst einmal, welchen Einfluss Abtastung von Signalen mit einzelnen Frequenzen hat, am Beispiel von
%
\begin{equation}\label{eq:analog_cosine}
    x_a(t) = A \cos(\Omega t + \theta),
\end{equation}
%
wobei wir hier $A \in \R$ als Amplitude, $\Omega \in \R^+_0$ als Kreisfrequenz $\SI{1}{\radian\per\second}$, $t \in \R$ als Zeit $\SI{1}{\second}$ und die Phase $\theta \in \R$ mit Einheit $\SI{1}{\radian}$ nutzen.
Alternativ k"onnen wir auch zur Frequenz $F \in \R$ $\SI{1}{\per\second} = \SI{1}{\hertz}$ "ubergehen.
Dann erhalten wir
\[
x_a(t) = A \cos(2 \pi F t + \theta).
\]
Diese Funktion ist in \cref{fig:analog_cosine} dargestellt.
Wir sehen, dass die Funktion periodisch ist mit Periode $T_p = 1/F$.

\myemph{Das hei"st, dass $x_a(t + k \cdot T_p)$ f"ur $k \in \Z$ nicht vom Signal $x_a(t)$ zu unterscheiden ist!}

\begin{figure}[t]
    \begin{center}
        \includegraphics[width=0.48\textwidth]{img/sampling/analog_cosine.png}
        \includegraphics[width=0.48\textwidth]{img/sampling/two_phasors.png}
    \end{center}
    \caption{$x_a(t) = A \cos(2 \pi F t + \theta)$, Quelle: \cite{proakis2013}}\label{fig:analog_cosine}
\end{figure}
%
Es gibt noch eine alternative Darstellung von der obigen Funktion durch die Addition von zwei \emph{Phasoren} als
\[
    x_a(t) 
        = A \cos(2 \pi F t + \theta) 
        = \frac A2 \exp(\jmath (\Omega t + \theta)) 
            + \frac A2 \exp(- \jmath (\Omega t + \theta)).
\]
Da die beiden "uberlagerten Phasoren so interpretiert werden k"onnen als rotierten diese in gegens"atzliche Richtungen, ist es gerechtfertigt der physikalischen Intuition entgegen auch von \q{negativen} Frequenzen zu sprechen.
Wir erlauben also $F \in \R$, womit auch der Spezialfall $T_p = \infty$, also $x_a(t) = A$ abgedeckt ist.
Ein kleines Beispiel findet man in \Cref{py:cont_harms}.
%
\begin{listing}
    \begin{minipage}{0.49\textwidth}
        \inputminted{python3}{code/cont_harms.py}
    \end{minipage}
    \includegraphics[width=0.49\textwidth]{code/cont_harms.png}
    \caption{Berechnung und Darstellung von \eqref{eq:analog_cosine}}\label{py:cont_harms}
\end{listing}
\FloatBarrier
%
\subsubsection{Zeit-Diskrete Harmonische}
%
Als n"achstes gehen wir zu dem eigentlich interessanten Fall "uber, bei welchem wir von zeitdiskreten harmonischen Signalen sprechen.
Dabei gehen wir vorerst \emph{nicht} davon aus, dass das Signal durch Abtastung eines analogen Signals entstanden ist, sondern betrachten es ganz losgel"ost f"ur sich.
In Analogie zu \eqref{eq:analog_cosine} definieren wir
%
\begin{equation}\label{eq:discrete_cosine}
    x[n] = A \cos(\omega n + \theta) = A \cos(2 \pi f n + \theta).
\end{equation}
%
Wichtig bei diskreten Signalen ist, dass ihre physikalische Interpretierbarkeit nicht direkt gegeben ist, da $n \in \Z$ nur die diskreten Werte \q{nummeriert}, also \emph{einheitenlos} ist.
Deshalb hat $f \in \R$ lediglich als Einheit \q{Zyklen pro Sample}, was man auch daran sieht, dass f"ur $f = 1$ gilt $x[n] = A \cos(2\pi n + \theta) = A \cos(\theta)$.
Es existiert nun ein wichtiger Unterschied zwischen $x[\cdot]$ von \eqref{eq:discrete_cosine} und $x_a(\cdot)$ von \eqref{eq:analog_cosine}.
Das Signal $x[\cdot]$ ist nur periodisch, falls $f$ eine rationale Zahl ist, also $f = p/q$ f"ur $p,q \in \Z$ und $q \neq 0$.

Wieso?

Ein zeitdiskrtes Signal ist periodisch, falls $x[n + N] = x[n]$ f"ur alle $n \in \Z$.
F"ur unser Signal in \eqref{eq:discrete_cosine} hei"st das also, dass
\[
    \cos(2 \pi f n + \theta) 
        = \cos(2 \pi f (n + N) + \theta)
        + \cos(2 \pi f n + 2 \pi f N + \theta)
\]
Da $\cos$ Periode $2\pi k$ f"ur $k \in \Z$ besitzt, muss $2 \pi f N = 2 \pi k$ gelten, also
\[
    f = \frac kN.
\]
Andersherum kann man die kleinste Periode $N$ ermitteln, indem man $f = k/N$ vollst"andig k"urzt, sodass also Z"ahler und Nenner keine gemeinsamen Teiler mehr haben, und man dann den Nenner des resultierenden Bruches als $N$ setzt. 
Ein Beispiel wird in \Cref{py:disc_harms} gezeigt. 
Man kann interessante Ergebnisse erzielen, wenn man diesen Plot f"ur $\omega=0,\pi/8,\pi/4,\pi/2$ und $\omega=\pi$ erzeugt ("Ubung).
%
\begin{listing}
    \begin{minipage}{0.49\textwidth}
        \strut\vspace*{-\baselineskip}\newline
        \inputminted{python3}{code/disc_harms.py}
    \end{minipage}
    \begin{minipage}{0.49\textwidth}
        \strut\vspace*{-\baselineskip}\newline
        \includegraphics[width=\textwidth]{code/disc_harms.png}
    \end{minipage}
    \caption{Berechnung und Darstellung von \eqref{eq:analog_cosine}}\label{py:disc_harms}
\end{listing}

Die Signale der Form \eqref{eq:discrete_cosine} haben noch eine andere interessante Eigenschaft, die sich wieder aus der $2\pi$-Periodizit"at von $\cos$ ergibt. 
Betrachten wir noch einmal \eqref{eq:discrete_cosine} und wir finden, dass
\[
\cos(\omega n + \theta) = \cos(\omega n + 2\pi n + \theta) = \cos((\omega + 2\pi) n + \theta).
\]
Das hei"st, dass
\[
    x[n] = \cos(\omega n + \theta) = \cos((\omega + 2\pi k) n + \theta) = x_k[n]
\]
f"ur \emph{alle} $k \in \Z$ gilt.
Das hei"st, dass sich $x_k[\cdot]$ nicht von $x[\cdot]$ unterscheiden l"asst.
Man nennt dann jedes der $x_k[\cdot]$ einen \emph{Alias} von $x[\cdot]$.
Man kann deshalb auch sagen, dass f"ur jedes $\omega$ mit $\Abs{\omega} > \pi$ ein zugeh"origes $\omega_a$ mit $\Abs{\omega_a} < \pi$ existiert, sodass
\[
    \cos(\omega n + \theta) = \cos(\omega_a n + \theta)
\]
gilt.
Vergewissern sie sich von dieser Tatsache, indem sie verschiedene Aliase basierend auf \Cref{py:disc_harms} visualisieren ("Ubung).

Stellen wir uns f"ur einen kurzen Moment vor, dass wir wissen, dass wir $x[n]$ durch Abtastung einer Funktion wie in \eqref{eq:analog_cosine} erhalten haben.
Selbst wenn wir wissen, dass nur \emph{eine} Frequenz in diesem Signal vor Abtastung vorhanden war, k"onnen wir \emph{nicht} entscheiden, welche das war.
%
\FloatBarrier
%
\subsection{Komplexe Harmonische}
%
Wir wollen eine bestimmte Menge an Funktionen betrachten. Wir wollen kontinuierliche komplexe Schwingungen betrachten, welche mit einer Frequenz $F_k$ schwingen, welche ein ganzzahliges Vielfaches einer Frequenz $F_0$ ist.
Das hei"st, wir betrachten dann
\[
F_k = k \cdot F_0 \Text{f"ur} k \in \Z \Text{, was}
x_k(t) = \exp(\jmath 2 \pi F_k t) = \exp(\jmath 2 \pi k F_0 t)
\]
ergibt. 
Jedes der $x_k$ hat Periode $1/F_k = T_k = T_0/k$. 
Das hei"st f"ur wachsendes $\Abs{k}$ werden die Perioden immer um ein Vielfaches k"urzer. 
Umgekehrt haben dann alle $x_k$ gemeinsame Periode, $T_0$, da f"ur jedes $k$ gilt, dass $T_k \cdot k = T_0$.
Wir haben auch kein Problem mit Aliasing zwischen den $x_k$, da bei kontinuierlichen Signalen gilt, dass $x_{k_1} \neq x_{k_2}$, falls $k_1 \neq k_2$.

Wie wir in \Cref{sec:signals_vec} gesehen haben, k"onnen wir beliebige Linearkombination aus Signalen bilden und erhalten wieder ein Signal.
Wir k"onnen also f"ur eine Folge von $c_k \in \C$ die Linearkombination der $x_k$ bilden und erhalten
%
\begin{equation}\label{eq:fourier_series}
    x_a = \Sum{k \in \Z}{}{c_k x_k} : \R \rightarrow \C
    \Text{mit} 
    t \mapsto x_a(t) = \Sum{k \in \Z}{}{c_k x_k(t)}
        = \Sum{k \in \Z}{}{c_k \exp(\jmath 2 \pi k F_0 t)}.
\end{equation}
%
Die erste Schreibweise ist absichtlich ohne das Argument $t$, um zu verdeutlichen, dass Signale \emph{wirklich} als eigenst"andige Signale behandelt werden k"onnen und dass $x_a(t) \in \C$ \q{nur} die Auswertung von $x_a$ an der Stelle $t$ ist, welche \emph{strikt} von dem \emph{Vektor} $x_a$ zu unterscheiden ist.

Nat"urlich ist \eqref{eq:fourier_series} als Fourierreihe von $x_a$ bekannt, und die $c_k$ sind die Fourierkoeffizienten von $x_a$.
Wie in \Cref{sec:signals_vec} k"onnen wir also die $c_k$ durch \eqref{eq:fourier_series} mit $x_a$ \emph{identifizieren}, da uns die $c_k$ eine alternative Darstellung von $x_a$ liefern.
%
%
%
\section{Diskrete Signale und Systeme}\label{disc_sys}
%
\subsection{Diskrete Signale}

Wir sind nun endlich im Digitalen angekommen. 
Wir wollen uns als erstes verschiedene M"oglichkeiten der Klassifikation von diskreten Signalen $x[\cdot] : \Z \rightarrow \C$ ansehen.
Dabei folgend wir weitestgehend~\cite[Kap.~2.1]{proakis2013}.
In \cref{sec:spec_sig} haben wir bereits den Einheitssto"s $\delta[\cdot]$ und die Heavy-Side-Funktion $u[\cdot]$ kennengelernt.

\begin{itemize}
    \item Eine linear ansteigende Version $u_r[\cdot]$ von $u[\cdot]$ ist gegeben durch
    \[
        u_r[n] = n \cdot u[n] = \begin{cases}
            n, \Text{f"ur} n \geqslant 0 \\
            0, \Text{sonst}.
        \end{cases}
    \]
    In Python ist die Funktion auch sehr einfach zu implementieren:
\begin{minted}{python3}
def u_r(n: int) -> int:
    return n if n>0 else 0
\end{minted}
Will man die Funktion effizienter mittels Numpy~\cite{numpy} implementieren, dann liest sie sich
\begin{minted}{python3}
import numpy as np
def u_r_np(n: np.ndarray[int]) -> np.ndarray[int]:
    u_r = n.copy()
    u_r[n <= 0] = 0
    return u_r
\end{minted}
\item F"ur eine komplexe Zahl $a = r \exp(\jmath \theta) \in \C$ erh"alt man das zugeh"orige exponentielle Signal als
\[
    x[n] 
        = a ^ n 
        = r^n \exp(\jmath \theta n) 
        = r^n \cos(\theta n) + \jmath r^n \sin(\theta n).
\]
Damit gilt $x[0] = 1$ unabh"angig von $a$.
Beispielsweise erhalten wir das Signal $x_k$ aus \eqref{eq:disc_harms_comp} indem wir $r=1$ und $\theta = 2 \pi k f_0$ setzen.
Eine Implementierung von $x[n]$ is in \Cref{py:complex_exp} gegeben.
Es ist sicher interessant f"ur verschiedene Werte von $a$ und $\bm n$ die Ausgabe zu betrachten.

Beispielsweise kann man sehen, dass f"ur $a \in \R$ gelten muss, dass $\lim_{n \rightarrow \infty} x[n] = 0$, falls $a < 0$, aber $\lim_{n \rightarrow -\infty} \Abs{x[n]} = \infty$ andernfalls.
Weiterhin gilt f"ur $a \in \C$ und $\Abs{a} = 1$, dass dann auch $\Abs{x[n]} = 1$ f"ur alle $n$.
\end{itemize}
%
\begin{listing}
    \noindent
    \begin{minipage}{0.49\textwidth}
        \strut\vspace*{-\baselineskip}\newline
        \inputminted[firstline=4]{python3}{code/complex_exp.py}
    \end{minipage}%
    \begin{minipage}{0.49\textwidth}
        \strut\vspace*{-\baselineskip}\newline
        \includegraphics[width=\textwidth]{code/complex_exp.png}
    \end{minipage}
    \codecaption{dsv/code/complex_exp.py}{Berechnung und Darstellung eines komplexen exponentiellen Signals}\label{py:complex_exp}
\end{listing}
%
\subsubsection{Energie Diskreter Signale}

Oft ist es interessant zu bemessen, wie viel Energie in einem Signal vorhanden ist.
F"ur eine physikalisch korrekte Bemessung dieser Energie, m"usste man das Signal zwar mit Einheiten versehen, aber diese ergeben nur einen entsprechenden Proportionalit"atsfaktor.
Hierzu betrachten wir
\begin{equation}\label{eq:disc_sig_energy}
    E(x[\cdot]) 
        = \Sum{n \in \Z}{}{\Abs{x[n]}^2} 
        = \Sum{n \in \Z}{}{x[n]^\ast \cdot x[n]}.
\end{equation}
Wenn gilt $E(x[\cdot]) < \infty$, dann sprechen wir erstaunlicherweise von einem Signal endlicher Energie.

Es ist nun interessant sich eine Menge $\mathcal{E}$ zu definieren, die alle Signale enth"alt, welche endliche Energie besitzen, also 
\[
    \mathcal{E} = \{x : \Z \rightarrow \C \Text{mit} E(x[\cdot]) < \infty\}.
\]
Man kann sich nun "uberlegen, dass
\[
E(\alpha x[\cdot] + \beta y[\cdot]) 
    \leqslant E(\alpha x[\cdot]) + E(\beta y[\cdot])
    = \alpha^2 E(x[\cdot]) + \beta^2 E(y[\cdot]) 
    < \infty 
\]
gelten muss, falls $E(x[\cdot]),E(y[\cdot]) < \infty$. 
Das hei"st, dass Linearkombinationen von Signalen mit endlicher Energie wieder ein Signal mit endlicher Energie ergeben.
Das hei"st, dass die Signale endlicher Energie bilden einen \emph{Unterraum}.
Wir k"onnen noch einen Schritt weiter gehen und wie in \eqref{eq:dtft_inner_prod} die Summe in \eqref{eq:disc_sig_energy} als Skalarprodukt auffassen.

Definieren wir f"ur zwei Signale endlicher Energie die Abbildung $\ScPr{\cdot}{\cdot} : \mathcal{E} \times \mathcal{E} \rightarrow \C$ als
\begin{equation}\label{eq:disc_inner_prod}
    (x[\cdot], y[\cdot]) 
        \mapsto \ScPr{x[\cdot]}{y[\cdot]}
        = \Sum{n \in \Z}{}{x[n]^\ast \cdot y[n]},
\end{equation}
dann kann man sich "uberlegen, dass dies die Bedingungen an ein \emph{Skalarprodukt} erf"ullt.
Beispielsweise kann man nachrechnen, dass die unendliche Summe in \eqref{eq:disc_inner_prod} immer endlich ist, falls $x[\cdot], y[\cdot] \in \mathcal{E}$, da
\[
\Abs{\ScPr{x[\cdot]}{y[\cdot]}} \leqslant E(x[\cdot]) \cdot E(y[\cdot]) < \infty
\]
Nun kann man aber auch $E$ durch
\[
E(x[\cdot]) = \ScPr{x[\cdot]}{x[\cdot]}
\] 
ausdr"ucken.

\begin{Bsp}
Betrachten wir $x[n] = a^n \cdot u[n]$ f"ur $a = r \exp{\jmath \theta} \in \C$.
Dann berechnet sich $E(x[\cdot])$ durch
\[
E(x[\cdot]) 
    = \Sum{n \geqslant 0}{}{\Abs{a^n}^2} 
    = \Sum{n \geqslant 0}{}{\left(r^2\right)^n}.
\]
Ist nun $r \geqslant 1$, dann $E(x[\cdot]) = \infty$, falls aber $r < 1$, dann ergibt sich aus der geometrischen Reihe, dass
\[
    E(x[\cdot]) = \frac{1}{1 - r^2}
\]
gilt.
Das hei"st auch, dass das Heavy-Side-Signal $u[\cdot]$ keine endliche Energie besitzt.
\end{Bsp}
%
\subsubsection{Periodische Signale}
%
Gilt f\"ur ein Signal $x[\cdot]$, dass $x[n + N] = x[n]$ f"ur ein $N \in \N$ und \emph{alle} $n \in \Z$, so nennt man $x[\cdot]$ periodisch mit Periodenl"ange/Periode $N$, oder kurz $N$-periodisch, siehe beispielsweise \Cref{eq:disc_harms_comp}.
Falls $x[\cdot]$ nun $N$-periodisch ist, dann ist $x[\cdot]$ auch $kN$-periodisch, falls $k \in \N$.
Das hei"st, dass es sinnvoller ist, das \emph{kleinste} $N \in \N$ zu betrachten, sodass $x[\cdot]$ dann $N$-periodisch ist. 
Man nennt $N$ dann Fundamentalperiode.
Falls solch ein $N$ nicht existiert, dann nennt man $x[\cdot]$ aperiodisch, oder nicht-periodisch.
Falls $x[\cdot] \neq 0$, dann gilt f"ur periodische Signale, dass $E(x[\cdot]) = \infty$.
Beispielsweise haben wir bereits in \Cref{sec:sample_harm} gesehen, dass
\[
x[n] = \exp(\jmath 2 \pi f)
\]
periodisch mit Periode $N$ ist, falls $f = k/N$, also eine rationale Zahl ist.
%
\subsubsection{Symmetrie von Signalen}
%
Gilt f"ur ein Signal $x[n] = x[-n]$, dann nennt man es \emph{symmetrisch} bzw.~\emph{gerade}.
Gilt andererseits $x[n] = -x[-n]$, so nennt man es \emph{anti-symmetrisch} bzw.~\emph{ungerade}.

Ist ein beliebiges Signal $x[\cdot]$ gegeben, so kann man
\[
    x_g[n] = \frac 12 \left(x[n] + x[-n]\right)
    \Text{und}
    x_u[n] = \frac 12 \left(x[n] - x[-n]\right)
\]
definieren.
Dann ist $x_g[\cdot]$ gerade und falls $x[\cdot]$ bereits gerade ist, so gilt $x[\cdot] = x_e[\cdot]$.
Genauso ist $x_u[\cdot]$ ungerade und falls $x[\cdot]$ bereits ungerade ist, so gilt $x[\cdot] = x_u[\cdot]$.
Au"serdem gilt
\[
x[n] = x_g[n] + x_u[n].
\]
Wir haben das Signal $x[\cdot]$ also in einen geraden und einen ungeraden Teil zerlegt.
Dies ist manchmal sinnvoll, wenn man solch ein Signal linear transformiert und wei"s, dass die lineare Transformation f"ur gerade oder ungerade Signale gewisse Eigenschaften hat.
Wie man an \Cref{py:even_odd} gut sehen kann, muss gelten $x_u[0] =0$, da $x_u[0] = x[0] - x[0] = 0$ und $x_e[0] = x[0]$, da $2 x_e[0] = x[0] + x[0]$.
%
\begin{listing}
    \noindent
    \begin{minipage}{0.49\textwidth}
        \strut\vspace*{-\baselineskip}\newline
        \inputminted[firstline=4]{python3}{code/even_odd.py}
    \end{minipage}%
    \begin{minipage}{0.49\textwidth}
        \strut\vspace*{-\baselineskip}\newline
        \includegraphics[width=\textwidth]{code/even_odd.png}
    \end{minipage}
    \codecaption{dsv/code/even_odd.py}{Zerlegung eines Signals in seinen geraden und ungeraden Anteil.}\label{py:even_odd}
\end{listing}
%
\subsection{Diskrete Systeme}
%
%
Nachdem wir uns nun ein wenig mit diskreten Signalen vertraut gemacht haben, sind wir in der Lage und mit diskreten Systemen zu befassen.
Ganz allgemein kann man fast jeden Prozess, an dessen Anfang ein diskretes Signal steht und dessen Ergebnis wiederum ein diskretes Signal ist, als ein diskretes System auffassen.
Sobald man dieses System nun einmal vorliegen hat, will man Techniken und Werkzeugen entwickeln, wie man dieses System systematisch untersuchen kann -- eine Systematik der Systeme.

Ein System $\mathcal{T}$ wird mathematisch als Abbildung eines (Eingabe-)Signals $x[\cdot]$ auf ein anderes (Ausgabesignal) $y[\cdot]$ aufgefasst. Wir schreiben daf"ur dann
%
\begin{equation}\label{eq:gen_disc_sys}
    x[\cdot] \mapsto \mathcal{T}(x[\cdot])[\cdot] = y[\cdot].
\end{equation}
%
Das System $\mathcal{T}$ bildet also die Paare $(x[\cdot], \mathcal{T}(x[\cdot])) = (x[\cdot], y[\cdot])$.

Betrachten wir folgendes
\begin{Bsp}\label{ex:simple_sys}
    Gegeben sei das Eingabesignal
    \[
    x[n] = \begin{cases}
        \Abs{n}, \Text{falls} -3 \leqslant n \leqslant +3, \\
        0 \Text{sonst.}
    \end{cases}
    \]
    Wir sind nun an den Werten der Ausgabesignals $y[\cdot]$ interessiert, f"ur
    \begin{enumerate}[a)]
        \item das Einheitssystem $y[n] = x[n]$,
        \item das Einheitsdelay-System $y[n] = x[n-1]$,
        \item das Einheitsadvance-System $y[n] = x[n+1]$
    \end{enumerate}
    interessiert.
\end{Bsp}
Diese Systeme waren noch ein wenig simpel und man ist vielleicht noch nicht "uberzeugt, dass eine genauere Analyse von diskreten Systemen notwendig sein sollte.
Dies liegt vor allem daran, dass die Systeme in \Cref{ex:simple_sys} nur \q{lokal} gearbeitet haben, da Werte $y[n]$ nur von Werten $x[n-1]$, $x[n]$ und $x[n+1]$ abh"angen.

Betrachten wir wiederum die Mandelbrot-Iteration\footnote{siehe \href{%
https://github.com/SebastianSemper/lecturenotes/blob/main/dsv/code/mandelbrot.py
}{\path{dsv/code/mandelbrot.py}}}, aber diesmal als System
\[
    y[n+1] = y[n]^2 + x[n]
\]
f"ur verschiedene Eingangssignale $x[n] = c \in \C$ mit $y[0] = 0$.
Wir sind nun an solchen $c$ interessiert f"ur welche das System divergiert, also $\Abs{y[n]} > 2$ ab einem gewissen $n$.
Wir wollen aber solche $c$ finden, f"ur welche wir in einem gewissen Bereich $n \in [n_{\rm low}, n_{\rm hgh}]$ divergieren.
Das in \Cref{py:buddhabrot} gezeigte Beispiel ist hierbei am anderen Ende des Komplexit"atsspektrums, da man bei diesem System eher von einem \q{chaotischen} System sprechen sollte. 
Kleine Ver"anderungen an $x[n] = c$ haben gro"sen Einfluss auf das Divergenzverhalten der Folge $y[n]$\footnote{\url{https://erleuchtet.org/2010/07/ridiculously-large-buddhabrot.html}}.
\begin{listing}
    \noindent
    \begin{minipage}{0.49\textwidth}
        \strut\vspace*{-\baselineskip}\newline
        \inputminted[firstline=5,lastline=26]{python3}{code/buddhabrot.py}
    \end{minipage}%
    \begin{minipage}{0.49\textwidth}
        \strut\vspace*{-\baselineskip}\newline
        \inputminted[firstline=29,lastline=53]{python3}{code/buddhabrot.py}
    \end{minipage}

    \begin{center}
        \includegraphics[width=0.55\textwidth]{code/buddhabrot.png}
    \end{center}
    \codecaption{dsv/code/buddhabrot.py}{Sp"at divergierende Orbits der Mandelbrot-Iteration}\label{py:buddhabrot}
\end{listing}

Um zu sehen, wie Systeme von ihrem Anfangszustand abh"angen k"onnen, wollen hierf"ur ein etwas einfacheres Beispiel betrachten. 
Gegeben ist das System
\[
y[n] = \Sum{k=-\infty}{n}{x[k]}.
\]
Wir sehen hier, dass es f"ur die Berechnung von $y[n]$ nicht ausreicht, den Zustand des Eingangs zum Zeitpunkt $n$, also $x[n]$ zu kennen.
Schlie"slich m"ussen wir die gesamte Vergangenheit von $x[\cdot]$ bis zum Zeitpunkt $n$ in die Berechnung einflie"sen lassen.
Wir k"onnen das System aber umschreiben in 
\[
y[n] = \Sum{k=-\infty}{n-1}{x[k]} + x[n] = y[n-1] + x[n],
\]
wobei wir nun auch sehen, warum dieses System \emph{Akkumulator} genannt wird, da $y[\cdot]$ im Prinzip die Werte von $x[\cdot]$ \q{aufsammelt}.

Stellen wir uns nun vor, dass wir dieses System modellieren/simulieren wollen f"ur $n \leqslant n_0$, so ben"otigen wir entweder die Werte $x[n]$ f"ur $n < n_0$, oder die sogenannte \emph{Anfangsbedingung} $y[n_0] = y_0$.
Dies erinnert an das L"osen einer Differentialgleichung
\[
\dot{y}(t) = x(t),
\]
wonach dann gilt, dass
\[
y(t) = \Int{-\infty}{t}{x(s)}{s} 
\Text{oder}
y(t) = y_0 + \Int{t_0}{t}{x(s)}{s},
\]
damit $y(t_0) = y_0$.
Ein Beispiel f"ur die Wirkung von einem Akkumulator ist in \Cref{py:accumulator} gegeben. 
Man sieht sehr gut, welchen Einfluss die Anfangsbedingung auf $x_2[\cdot]$ hat, da dies den Grenzwert $\lim_{n \rightarrow \infty} y[n]$ ma"sgeblich beeinflusst.
Falls gilt, dass $y_0 = 0$, so spricht man vom Ruhezustand, beziehungsweise dem Nullzustand in dem sich das System zum Zeitpunkt $n = n_0$ befindet.
%
\begin{listing}
    \noindent
    \begin{minipage}{0.49\textwidth}
        \strut\vspace*{-\baselineskip}\newline
        \inputminted[firstline=4,lastline=18]{python3}{code/accumulator.py}
    \end{minipage}%
    \begin{minipage}{0.49\textwidth}
        \strut\vspace*{-\baselineskip}\newline
        \includegraphics[width=\textwidth]{code/accumulator.png}
    \end{minipage}%
    \codecaption{dsv/code/accumulator.py}{Akkumulator f"ur zwei verschiedene Eingangsignale.}\label{py:accumulator}
\end{listing}

\paragraph{Statisch vs. Dynamisch}
Man kann Systeme nun auf verschiedene Arten klassifizieren.
Beispielsweise nennen wir Systeme \emph{statisch}, wenn der Wert $y[n]$ nur von $x[n]$ abh"angt, aber nicht von vergangenen oder gar zuk"unftigen Werten (entweder von $y[\cdot]$ oder $x[\cdot]$).
Die Systeme
\[
y[n] = a x[n], \Text{oder} y[n] = \sqrt{x[n]} + x^4[n]
\]
sind statisch.
H"angt nun $y[n]$ von seiner eigenen Vergangenheit oder der von $x[n]$ ab, so nennen wir diese Systeme dynamisch, oder Systeme mit Ged"achtnis.
Die Systeme
\[
y[n] = x[n] + a x[n-1] \Text{,} y[n] = \Sum{k=0}{N}{x[n-k]}
\]
sind dynamisch und haben jeweils Ged"achtnis der L"ange $g = 1$, beziehungsweise $g = N-1$.
Man sieht bereits an \Cref{py:accumulator}, dass dynamische System im Allgemeinen interessanter sein werden.

\paragraph{Kausal vs. Akausal} 
H"angen die Werte $y[n]$ nur von $x[n], x[n-1], \dots$ ab, also nicht auch von $x[n+1], x[n+1], \dots$, so nennen wir das System kausal.
Intuitiv bedeutet dies die intuitiv bekannte Kausalit"at in dem Sinne, dass nur die Zeitliche Vergangenheit notwendig ist, um den aktuellen Zustand des Systems zu bestimmen.
Ist die nicht gegeben, so nennt man das System \emph{akausal} oder \emph{nicht-kausal}.

Was die Realisierung von akausalen Systemen angeht, wird es nicht m"oglich sein, diese in Echtzeit umzusetzen, da man diese nur mit einer Verz"ogerung, oder gar nicht f"ur sequentiell verf"ugbares $x[\cdot]$ implementieren kann.
Sind die Werte $x[n]$ jedoch beispielsweise durch Messung oder Simulation entstanden und \q{offline} verf"ugbar, so k"onnen solche Systeme durchaus angewandt werden und n"utzlich sein.

\paragraph{Stabil vs. Instabil} Eine der zentralen Eigenschaften, die auch bei analogen Systemen eine Rolle spielt ist Stabilit"at.
Zwar hat man normalerweise bereits einen intuitiven Begriff f"ur Stabilit"t im Sinn, doch formal gibt es hiervon verschiedene Auspr"agungen.
Wir beschr"anken uns auf die Version der \gls{bibo}-Stabilit"at, welche fordert, dass f"ur beschr"anktes Eingangssignal $x[\cdot]$ der Ausgang $y[\cdot]$ ebenfalls beschr"ankt bleibt.
Formal fordern wir also, dass falls
\[
\Abs{x[n]} \leqslant M_x \Text{f"ur alle} n \in \Z
\]
f"ur eine Konstante $M_x \in \R$ gilt, dann auch 
\[
\Abs{y[n]} \leqslant M_y \Text{f"ur alle} n \in \Z
\]
gelten muss. 
Man sieht, dass $M_x$ und $M_y$ an $x[\cdot]$ und $y[\cdot]$ gebunden sind, also keine \q{universellen} Konstanten sind.
Andersherum reicht es also f"ur Instabilt"at nur \emph{ein} beschr"anktes Eingangssignal $x[\cdot]$ konstruiert werden muss, f"ur welches $y[\cdot]$ nicht beschr"ankt bleibt.
Betrachten wir folgendes
\begin{Bsp}
Gegeben sei
\[
y[n] = C \cdot y[n-1]^2 + x[n]
\]
und wir nutzen als Eingang den Einheitssto"s $x[\cdot] = C \delta[\cdot]$, welcher beschr"ankt ist, mit $M_x = \Abs{C}$ f"ur $C \in \R$.
Doch dieser produziert f"ur $y[n] = 0$ f"ur $n \leqslant -1$ die Folge
\[
y[n] = \{0, \Start{C}, C^2, C^4, \dots, C^{2n}\},
\]
welche f"ur $\Abs{C} > 1$ gegen $\infty$ divergiert.
\end{Bsp}

\paragraph{Zeitvariant vs. Zeitinvariant} Systeme deren Eingabe-Ausgabe-Verhalten nicht zeitlich konstant ist, nennt man \emph{zeitvariant}.
Systeme, die f"ur zeit verz"ogerte Eingaben, die um den gleichen Zeitraum verz"ogerte Ausgaben produzieren, nennt man \emph{zeitinvariant}, siehe \Cref{exm:cont_lit}.
Formal fordern wir f"ur Zeitinvarianz, dass wenn f"ur beliebige Eingabe $x[\cdot]$
\[
x[\cdot] \overset{\mathcal{T}}{\rightarrow} y[\cdot]
\]
gilt, dass dann f"ur jedes $k \in Z$ auch gilt, dass
\[
x[\cdot -k] \overset{\mathcal{T}}{\rightarrow} y[\cdot - k]
\]
erf"ullt ist.
In der Schreibweise von \eqref{eq:gen_disc_sys} hei"st dies, dass f"ur 
\[
    y[n] = \mathcal{T}(x[\cdot])[n]
\]
auch gelten muss, dass
\[
    y[n - k] = \mathcal{T}(x[\cdot - k])[n]
\]
und zwar f"ur alle $x[\cdot]$ und $k$.

Was erst einmal relativ abstrakt daherkommt, ist eigentlich eine sehr intuitive Sache. 
Wenn wir beispielsweise an ein Audiointerface denken, so h"atten wir schon gerne, dass es egal ist, zu welchem Zeitpunkt jemand ins Mikrophon singt -- unabh"angig vom Zeitpunkt sollte die Aufnahme \q{gleich klingen}.
Das System, welches die Aufnahme und eventuelle Audioverarbeitung realisiert, sollte keine zeitlichen Ver"anderungen zeigen.
Auch in der umgekehrten Richtung, beim Abspielen von Ton, sollte es irrelevant sein, zu welchem Zeitpunkt man ein gewisses St"uck h"oren m"ochte -- das H"orerlebnis sollte davon nicht beeinflusst sein.

Im Grunde ist Zeitinvarianz also etwas, das wir normalerweise von einem System \q{erwarten} und nicht untersuchen wollen.

\paragraph{Linear vs. Nicht-Linear} Kommen wir zum Schluss dieser Klassifikationen zu einer der wichtigsten Unterscheidungen.
Ein System, das f"ur Eing"ange $x_1[\cdot]$ und $x_2[\cdot]$ die Antworten
\[
    y_1[n] = \mathcal{T}(x_1[\cdot])[n]
    \Text{und}
    y_2[n] = \mathcal{T}(x_2[\cdot])[n]
\]
produziert, nennen wir \emph{linear}, falls die Antwort des Systems auf den Eingang
\[
    x[\cdot] = a_1 x_1[\cdot] + a_2 x_2[\cdot]
\]
sich durch
\[
    y[n]
        = \mathcal{T}(x[\cdot])[n] 
        = \mathcal{T}(a_1 x_1[\cdot] + a_2 x_2[\cdot])[n] 
        = a_1 \mathcal{T}(x_1[\cdot])[n] 
            + a_2 \mathcal{T}(x_2[\cdot])[n] 
\]
ausdr"ucken l"asst.
%
\begin{figure}
    \centering\includegraphics[width=0.6\textwidth]{img/disc_sys/linear_sys.png}
    \caption{zeigt zwei verschiedene systemtheoretische Interpretation von linearen Systemen. Quelle: \cite{proakis2013}}\label{img:disc_sys:linear_sys}
\end{figure}
%

\Cref{img:disc_sys:linear_sys} zeigt die beiden m"oglichen Interpretationen dieser Eigenschaft.
Man kann sich also vorstellen, dass die Eing"ange \emph{erst} skaliert und addiert werden und dann das System durchlaufen, oder man prozessiert beide Eing"ange durch $\mathcal{T}$ und skaliert und addiert die \emph{Ausg"ange nachdem} das System im Grund \q{zweifach} angewandt wurde.

Anders gesagt \q{passen} lineare Systeme genau zu der linearen Struktur von Signalen, wenn wir sie als Vektoren in einem Vektorraum auffassen.
Als Konsequenz ergibt sich, dass sich lineare Systeme deutlich einfacher analysieren lassen, weil wir Werkzeuge der linearen Algebra benutzen k"onnen.
Diese Eigenschaft ist so attraktiv, dass man oft versucht nichtlineare Systeme durch geeignete lineare System zu approximieren (Bspw.: Pendel $\sin(x) \approx x$).
Man nimmt also Fehler in der Analyse in Kauf\footnote{\url{https://en.wikipedia.org/wiki/Hartman-Grobman_theorem}}, ist damit aber immerhin in der Lage "uberhaupt Aussagen treffen zu k"onnen.
%
%
\subsection{Diskrete \texorpdfstring{\acrshort{lti}}{LTI}-Systeme}\label{sec:disc_lti}
%

Wir schr"anken nun die Menge der Systeme ein, die wir betrachten wollen, indem wir fordern, dass das System $\mathcal{T}$ gleichzeitig linear und zeitinvariant ist.
Das hei"st, es gilt einerseits, dass ein Eingang $x[\cdot]$ zum System $\mathcal{T}$ mit Ausgang $y[\cdot]$ bei Verz"ogerung zu $x[\cdot - k]$ den entsprechend verz"ogerten Ausgang $y[\cdot - k]$ zur Folge hat.
Gleichzeitig kann der Ausgang des Systems $\mathcal{T}$ f"ur Eingange, die lineare Superpositionen sind als lineare Superposition von den entsprechenden Ausg"angen ausgedr"uckt werden, siehe \Cref{img:disc_sys:linear_sys}.

\subsubsection{Faltungsformel}

Wir wollen die Struktur von \gls{lti}-Systemen ausnutzen, um eine allgemeine und einfache Formel f"ur das Eingangs-Ausgangsverhalten von jenen angeben zu k"onnen.
Als Erstes verallgemeinern wir hierzu \Cref{img:disc_sys:linear_sys} zu beliebigen, aber endlichen Summen, also gegeben ist ein Eingang $x[\cdot]$ der Form
\begin{equation}\label{eq:lti_sys:input}
    x[n] = \Sum{k=1}{K}{a_k \cdot x_k[n]},
\end{equation}

wobei wir wissen, wie die Ausg"ange von jedem $x_k[\cdot]$ zu berechnen sind, also
\[
    y_k[n] = (\mathcal{T}x_k[\cdot])[n].
\]
Dann wissen wir wegen der Linearit"at und \eqref{eq:lti_sys:input}, dass
\begin{equation}\label{eq:lti_sys:superpos}
    y[n] 
        = \left[\mathcal{T}\left(
            \Sum{k=1}{K}{a_k \cdot x_k[\cdot]}
        \right)\right][n] 
        = \Sum{k=1}{K}{
            a_k (\mathcal{T}x_k[\cdot])[n]
        }
        = \Sum{k=1}{K}{
            a_k y_k[n]
        }
\end{equation}
gelten muss.
Es lohnt sich einige Zeit "uber diese Sache zu meditieren und es gibt verschiedene Interpretationen.
\begin{itemize}
    \item Wie bereits erw"ahnt passt dies zur linearen Struktur des Systems $\mathcal{T}$.
    \item Ist ein Eingangssignal aus anderen Signalen zusammengesetzt, dann setzt sich die Reaktion des Systems auf dieses zusammengesetzte Signal aus den Reaktionen auf die Signalbausteine zusammen. 
    Wichtig ist hierbei, dass die Art der Zusammensetzung sich nicht "andert. Die $a_k$ in \eqref{eq:lti_sys:superpos} sind die gleichen, wie in \eqref{eq:lti_sys:input}.
\end{itemize}

Wir wollen nun einen Schritt weiter gehen und eine Menge von $x_k[\cdot]$ angeben, die es erlauben \emph{alle} m"oglichen Signale darzustellen.
Dazu betrachten wir, was geschieht, wenn wir den Einheitssto"s $\delta[\cdot]$ mit einem beliebigen Signal multiplizieren.
Wir rechnen demzufolge f"ur ein beliebiges Signal
\[
x[n] \cdot \delta[n] = \begin{cases}
    x[0] \Text{f"ur} n = 0 \\
    0 \Text{sonst.}
\end{cases}
\]
Wenn wir nun die Einheitsst"o"se verschieben um $k \in \Z$ erhalten wir
\[
    x[n] \cdot \delta[n-k] = \begin{cases}
        x[k] \Text{f"ur} n = k \\
        0 \Text{sonst.}
    \end{cases}
\]
Im Grunde \q{pickt} $\delta[\cdot-k]$ bei Multiplikation den Wert von $x[\cdot]$ an der Stelle $k$ heraus.
Deshalb k"onnen wir nun schreiben
\begin{equation}
    x[n] = \Sum{k \in \Z}{}{
        x[k] \cdot \delta[n-k],
    }
\end{equation}
was nach Definition von $x_k[\cdot] = \delta[\cdot - k]$ genau die Form von \eqref{eq:lti_sys:input} mit $a_k = x[k]$ annimmt.
Die Kernbeobachtung ist nun, dass jedes $x_k[\cdot]$ eine verschobene Kopie von $\delta[\cdot]$ ist. 
Das hei"st, dass wir nun die \gls{lti}-Eigenschaft ausnutzen k"onnen, weil \eqref{eq:lti_sys:superpos} impliziert, dass wir nur $(\mathcal{T}\delta[\cdot])[\cdot]$ berechnen m"ussen und $(\mathcal{T}\delta[\cdot - k])[\cdot]$ sich als $(\mathcal{T}\delta[\cdot])[\cdot - k]$ ergibt.

Wir geben dem Kind nun einen Namen, also definieren wir $h: \Z \rightarrow \C$ als die Antwort des Systems $\mathcal{T}$ auf den Eingang $\delta[\cdot]$, also
\begin{equation}\label{eq:lti_sys:ir}
    h[n] = (\mathcal{T} \delta[\cdot])[n].
\end{equation}
Man nennt $h[\cdot]$ die \emph{Impulsantwort} des Systems.
Dann k"onnen wir also mit \eqref{eq:lti_sys:superpos} folgern, dass
\begin{equation}\label{eq:lti_sys:conv}
    y[n] 
        = (\mathcal{T} x[\cdot])[n] 
        = \Sum{k \in \Z}{}{
            x[k] h[n-k]
        }
\end{equation}
gelten muss.

Das hei"st, dass sich die Antwort $y[\cdot]$ eines diskreten \gls{lti}-Systems aus der \emph{Faltung} des Einganges $x[\cdot]$ mit der Impulsantwort $h[\cdot]$ ergibt.
Wir schreiben in Kurzform
\[
y[n] = (x \ast h)[n] = (h \ast x)[n].
\]
Es wird sich zeigen, dass sich viele Eigenschaften des Systems $\mathcal{T}$ an oft einfacher zu pr"ufenden Eigenschaften der Impulsantwort $h[\cdot]$ ergeben.
Das hei"st, dass $h[\cdot]$ in gewisser Weise das System $\mathcal{T}$ repr"asentiert.
%
\begin{listing}
    \noindent
    \begin{minipage}{0.49\textwidth}
        \strut\vspace*{-\baselineskip}\newline
        \inputminted[firstline=10,lastline=22]{python3}{code/moving_average.py}
    \end{minipage}%
    \begin{minipage}{0.49\textwidth}
        \strut\vspace*{-\baselineskip}\newline
        \includegraphics[width=\textwidth]{code/moving_average.png}
    \end{minipage}
    \codecaption{code/moving_average.py}{Gleitendes Mittel mit L"ange $\ell=4$. Wir vergleichen die direkte Berechnung mit der Berechnung "uber die Faltung}\label{py:moving_average}
\end{listing}

In \Cref{py:moving_average} zeigen wir das Verhalten eines gleitenden Mittelwertes (\emph{moving average}), welches sich durch
\[
y[n] = \frac{1}{\ell} \Sum{k=0}{k=\ell}{x[n-k]}
\]
ergibt.
Man sieht sch"on, dass durch das Mitteln die (in diesem Fall) zuf"allige Eingabe-Sequenz am Ausgang gegl"attet erscheint.
Wichtig bei diesem Beispiel ist die Tatsache, dass wir direkt einsehen, dass Anwenung der direkten Formel f"ur das gleitende Mittel aus eine Eingabe $x[\cdot]$ denselben Effekt hat, wie die Faltung mit $h[\cdot]$, das sich aus der Anwendung der Mittelung auf $\delta[\cdot]$ ergibt.

Es lohnt sich, sich einige Eigenschaften der Faltung zu merken. 
Diese sind:
\begin{itemize}
    \item Bi-Linearit"at: Es gilt $(a_1 x_1[\cdot] + a_2 x_2[\cdot]) \ast h[\cdot] = a_1 (x_1 \ast h)[\cdot] + a_2 (x_2 \ast h)[\cdot]$. 
    Dies ist die \q{normale} Linearit"at in den Eing"angen, die sich aus der Linearit"at des Systems $\mathcal{T}$ ergibt. 
    Es gilt aber auch $(x \ast (a_1 h_1 + a_2 h_2))[\cdot] = a_1 (x \ast h_1)[\cdot] + a_2 (x \ast h_2)[\cdot]$.
    Das hei"st, wenn wir ein System $\mathcal{T} = a_1 \mathcal{T}_1 + a_2 \mathcal{T}_2$ gegeben haben, dann ist die Impulsantwort des Systems $\mathcal{T}$ die gleiche Linearkombination der Impulsantworten $h_1[\cdot]$ und $h_2[\cdot]$ der beiden Systeme $\mathcal{T}_{1,2}$.
    Das hei"st wiederum, dass \gls{lti}-Systeme selbst ein linearer Raum sind!
    Wir sprechen hier von \emph{Bi}-Linearit"at, weil die Faltung eben linear in zwei Argumenten ist.
    \item Die Faltung ist assoziativ: Es gilt also, dass $(x \ast h_1) \ast h_2 = x \ast (h_1 \ast h_2)$. 
    Dies impliziert, dass die Verkettung von zwei \gls{lti}-Systemen wieder ein \gls{lti}-system ergibt, wobei sich die Impulsantwort der Verkettung durch Faltung der beiden Impulsantworten der verketteten Systeme ergibt.
    \item Kommmutativit"at: Es gilt $h_1 \ast h_2 = h_2 \ast h_1$, demnach auch, dass $x \ast (h_1 \ast h_2) = x \ast (h_2 \ast h_1)$.
    Das hei"st erstaunlicherweise, dass man verkettete \gls{lti}-Systeme in ihrer Reihenfolge vertauschen kann, ohne das Eingangs-Ausgangsverhalten des Gesamtsystems zu beeinflussen.
\end{itemize}

\begin{listing}
    \noindent
    \begin{minipage}{0.40\textwidth}
        \strut\vspace*{-\baselineskip}\newline
        \inputminted[firstline=10,lastline=33]{python3}{code/ramp_ma.py}
    \end{minipage}%
    \begin{minipage}{0.59\textwidth}
        \strut\vspace*{-\baselineskip}\newline
        \includegraphics[width=\textwidth]{code/ramp_ma.png}
    \end{minipage}
    \codecaption{code/ramp_ma.py}{Verkettung von mehreren Moving Averages der L"ange $\ell = 3$.}\label{py:ramp_ma}
\end{listing}
%
In \Cref{py:ramp_ma} untersuchen wir einige der oben genannten Eigenschaften.
Einerseits sehen wir, dass die Impulsantwort von $\texttt{MA}(\texttt{MA2}(\cdot))$ mit der von $\texttt{MA2}(\texttt{MA}(\cdot))$ identisch ist. Wir haben also Kommutativit"at nachgepr"uft.
Wir sehen au"serdem, dass sich die Impulsantwort der Verkettung aus Faltung der einzelnen Impulsantworten ergibt.
Aus dem abgetasteten \q{Rechteck} wird nach nochmaliger Anwendung ein abgetastetes, aber breiteres, \q{Dreieck}, was schlussendlich zu einer abgetasteten, st"uckweise quadratischen, Impulsantwort wird.
Generell kann man das komplette System als eine dreifache Verkettung gleitender Mittel der L"ange $\ell=3$ verstehen.
Au"serdem best"atigt \Cref{py:ramp_ma} bei Vergleich der verschiedenen Ausg"ange, dass wiederholtes Mitteln am Ausgang mit Anzahl der Mittelungen zunehmend \q{glattere} Signale erzeugt.
%
\subsubsection{Eigenschaften von \texorpdfstring{\acrshort{lti}}{LTI}-Systemen}
%
\begin{itemize}
    \item stabilitaet: $h[n]$ muss absolut summierbar sein, gegen $0$ gehen
    \item beispiel 2.3.6. (fuer uebung irgendwas ausdenken)
    \item \gls{fir} vs. \gls{iir}
\end{itemize}

\subsection{Kreuz- und Autokorrelation}\label{corr}

\begin{itemize}
    \item Definition
    \item eigenschaften
    \item synthetisches beispiel schwingung + noise, akf zeigt periodizitaet
    \item beispiel mit woelfer sunspot numbers
    \item beispiel mit m-sequenzen, niklas einladen, schaltung zeigen
    \item monster uebung: 2.65
\end{itemize}
%
%
\subsection{\texorpdfstring{$z$}{z}-Transformation}\label{ztrafo}
%
\begin{itemize}
    \item Wir wollen diskrete \gls{lti} systeme analysieren
    \item Wir wollen gewisse signale vlt kompakter aufschreiben koennen
    \item Definition, ROC, man kann an koeffizienten von $z^n$ die werte an zeitpunkt $n$ ablesen
    \item Beispiele mit endlichen signalen, tabelle \cite[p155, top]{proakis2013}
    \item beispiel mit unendlichem signal
    \item analyse von ROC durch betrachtung von $\Abs{X(z)}$, tabelle \cite[p155, top]{proakis2013}
    \item eigenschaften: linear, timeshift (intuition mit koeffizienten), zeitumkehrung(intuition mit spiegelung am einheitskreis), ableitung in $z$-Bereich (analogie zu fourier)
    \item faltungseigenschaft
    \item algo: beide sequenzen $z$-trafo, multiplikation, inverse $z$-trafo
    \item inverse: tabelle \cite[tabelle 3.3]{proakis2013}, allgemeine invertierung: schwierig
    \item alle eigenschaften: \cite[tabelle 3.2]{proakis2013}
    \item anwendung: korrelation von signalen
    \item uebung: problem 3.6 in \cite{proakis2013}, initial value theorem, some plots aus \cite{proakis2013}
    \item rationale z-trafos: definition, pole-nullstellen, umkehrung: von pole-nullstellen zu $X(z)$
    \item ausfuehrliche diskussion von \cite[fig 3.3.5, 3.3.6]{proakis2013} 
    
\end{itemize}

\subsection{Kreuz- und Autokorrelation}\label{corr}

\begin{itemize}
    \item Anwendung $z$-Trafo: M-Sequenzen? polynome analysieren, wann maximale laenge?
    \item Definition
    \item eigenschaften
    \item synthetisches beispiel schwingung + noise, akf zeigt periodizitaet
    \item beispiel mit woelfer sunspot numbers
    \item beispiel mit m-sequenzen, niklas einladen, schaltung zeigen
    \item monster uebung: 2.65
\end{itemize}

%
\clearpage
\microtypesetup{protrusion=false}
\addcontentsline{toc}{section}{Abk"urzungsverzeichnis}
\printglossary[type=\acronymtype]
\microtypesetup{protrusion=true}
%
%
%
\clearpage
\microtypesetup{protrusion=false}
\addcontentsline{toc}{section}{Literaturverzeichnis}
\printbibliography
\microtypesetup{protrusion=true}
\end{document}
%
%
\section{Fourier-Transformation}\label{fourier}
%
Wir wollen unsere Werkzeuge zur Analyse von Signalen und Systemen nun um das wahrscheinlich wichtigste erweitern.
Hierbei zerlegen wir Signale, beziehungsweise Systemantworten/Impulsantworten, in ihre \q{harmonischen} Anteile -- wir transformieren in den \emph{Frequenzbereich}.
Man diese Art der \emph{Analyse} auch oft \emph{Fourier-Analyse}.
Da wir verschiedene Arten von Signalen bereits kennengelernt haben, muss diese Zerlegung auch auf verschiedene Weisen durchgef"uhrt werden.
F"ur diskrete Signale, ergibt es beispielsweise wenig Sinn, eine Integraltransformation zu definieren, Aperiodische Signale wiederum k"onnen nicht in einer Fourier-Reihe entwickelt werden, da die inh"arent der Periodizit"at widerspricht.
Das hei"st, dass f"ur jede \q{Art} von Signal und dessen Eigenschaften, die passende Transformation existiert.
Weiterhin "andert sich auch immer die \emph{Interpretation} dieser Zerlegung in harmonische Komponenten.

Dar"uber hinaus werden wir auch den umgekehrten Weg gehen.
Es ist auch m"oglich, Signale aus dem Frequenzbereich in den jeweils richtigen Definitionsbereicht zu transformieren. 
In diesem Fall spricht von von \emph{Fouriersynthese}, da wir ein Signal aus dessen Information "uber harmonische Anteile erzeugen/synthetisieren.
Wir liefern nun also die Definitionen und Zusammenh"ange der Fourier-Transformation, die wir in \Cref{sec:sampling} ohne Erl"auterungen ausgenutzt haben.

\subsection{Fourier-Transformation kontinuierlicher Signale}\label{sec:fourier:cont}

\subsubsection{Fourier-Transformation kontinuierlicher periodischer Signale}\label{sec:fourier:cont:period}

Wir haben bereits in \Cref{sec:cont_complex_harm} mit \eqref{eq:cont_fourier_series} gesehen, dass man durch Linearkombination der komplexen Sinus-Funktionen
\[
\{\exp(\jmath 2 \pi k F_0 t), \fuer k \in \Z\}
\]
eine $1/F_0=T_0$-periodische Funktion $x: \R \rightarrow \C$ durch
\[
x(t) = \Sum{k \in \Z}{}{c_k \exp(\jmath 2 \pi k F_0 t)}
\]
erh"alt.
Dies ist demnach ein Fall von \emph{Fourier-Synthese}, da wir aus den Gewichten $c_k$ in der Linearkombination eine Funktion $x$ erhalten.
Wir wollen nun den umgekehrten Weg gehen, auf welchem wir f"ur eine gegebene Funktion $x$ die Koeffizienten $c_k$ bestimmen k"onnen.
Wir starten dazu mit 
\begin{equation}\label{eq:fourier:fourier_series}
    x(t) = \Sum{k \in \Z}{}{
        c_k \exp(\jmath 2 \pi k F_0 t)
    }
\end{equation}
und multiplizieren beide Seiten mit $\exp(-\jmath 2 \pi \ell F_0 t)$ und integrieren "uber eine Periode $[0,T_0]$.
Dann erhalten wir
\[
\Int{0}{T_0}{
    x(t) \exp(-\jmath 2 \pi \ell F_0 t)
}{t} 
= \Int{0}{T_0}{
    \left(
        \Sum{k \in \Z}{}{
            c_k \exp(\jmath 2 \pi k F_0 t)
        }
    \right)
    \exp(-\jmath 2 \pi \ell F_0 t)
}{t} 
\]
und nach Vertauschung von Summation und Integration, dass
\[
\Sum{k \in \Z}{}{
    c_k 
    \Int{0}{T_0}{\exp(\jmath 2 \pi (k - \ell) F_0 t)}{t}
}
= \Sum{k \in \Z}{}{
    c_k \left[
        \frac{
            \exp(-\jmath 2 \pi (k - \ell) F_0 t)
        }{
            \jmath 2 \pi F_0(k - \ell)
        }
    \right]_{0}^{T_0}
}.
\]
Da die Funktion $\exp(-\jmath 2 \pi (k - \ell) F_0 t)$ im Fall $k \neq  \ell$ Periode $T_0$ besitzt, sind alle Summanden in der rechten Summation identisch $0$, au"ser wenn $k = \ell$.
Dann ergibt sich f"ur $\exp(\jmath 2 \pi (k - \ell) F_0 t) = 1$, also
\[
\Int{0}{T_0}{\exp(\jmath 2 \pi (k - \ell) F_0 t)}{t} 
    = \Int{0}{T_0}{1}{t} 
    = T_0.
\]
Final erhalten wir f"ur die Berechnung der Fourier-Koeffizienten $c : \N \rightarrow \C$ als Berechnungsvorschrift
\[
    c[\ell] = \frac{1}{T_0}\Int{0}{T_0}{
        x(t) \exp(-\jmath 2 \pi \ell F_0 t)
    }{t}.
\]
Wir haben in diesem Fall also \emph{Fourier-Analyse} betrieben.
Au"serdem haben wir absichtlich die Schreibweise von $c_\ell$ auf $c[\cdot]$ angepasst, um zu verdeutlichen, dass man nun die $c[\cdot]$ als \emph{komplexes diskretes Signal} auffassen k"onnen.
Dieses diskrete Signal k"onnen wir nun durch die Fourier-Reihe mit dem Signal $x : \R \rightarrow \C$ \emph{identifizieren}.
Sowohl $x$ als auch $c[\cdot]$ sind also Darstellungen desselben Sachverhaltes -- im \q{Zeitbereich} und im zugeh"origen \q{Frequenzbereich}.
Wir sehen auch, dass sich f"ur kontinuierliche und periodische Signale ein \emph{diskreter} Frequenzbereich ergibt.

Obige Herleitung verschleiert aber einen viel tiefer liegenden Zusammenhang.
Definieren wir wie in \Cref{sec:cont_complex_harm} die Funktionen $x_k : \R \rightarrow \C$ als
\[
x_k(t) = \exp(\jmath 2 \pi k F_0 t),
\]
dann k"onnen wir obige Rechnung auch so auffassen.
Wir betrachten Multiplikation von $x$ mit $x_\ell^\ast$ und anschlie"sende Integration als Skalarprodukt $\ScPr{x}{x_\ell}$. 
Au"serdem haben wir aus der Fourier-Reihe gegeben, dass
\[
x = \Sum{k \in \Z}{}{c_k x_k}
\]
Wir haben also in dieser Denkweise lediglich auf beiden Seiten der Fourierreihe in \eqref{eq:fourier:fourier_series} das Skalarprodukt mit $x_\ell$ gebildet.
Also
\[
\ScPr{x}{x_\ell} = \ScPr{\Sum{k \in \Z}{}{c_k x_k}}{x_\ell}.
\]
Das Skalarprodukt ist linear, also k"onnen wir auch
\[
\ScPr{x}{x_\ell} = \Sum{k \in \Z}{}{c_k \ScPr{x_k}{x_\ell}}
\]
schreiben.
Wir m"ussen also nur $\ScPr{x_k}{x_\ell}$ bestimmen. 
Dies haben wir aber oben schon berechnet, denn das waren die Ausdr"ucke
\[
\ScPr{x_k}{x_\ell}
    = \Int{0}{T_0}{x_k(t) x_\ell(t)^\ast}{t}
    = \Int{0}{T_0}{\exp(\jmath 2 \pi k F_0 t) \exp(-\jmath 2 \pi \ell F_0 t)}{t}
    = \Int{0}{T_0}{\exp(\jmath 2 \pi (k - \ell) F_0 t)}{t}
\]
Von oben wissen wir, dass
\[
\ScPr{x_k}{x_\ell} 
    = \left[
        \frac{
            \exp(-\jmath 2 \pi (k - \ell) F_0 t)
        }{
            \jmath 2 \pi F_0(k - \ell)
        }
    \right]_{0}^{T_0}
    = \begin{cases}
        T_0 \fuer k = \ell \\
        0, \Text{sonst.}
    \end{cases}
\]
Die $x_k$ stehen also paarweise \emph{senkrecht/orthogonal} zu einander und es gilt $\ScPr{x}{x_\ell} = T_0$.
Sie bilden ein sogenannten \emph{Orthogonalsystem}.
H"atten wir $\hat{x}_k = x_k / \sqrt{T_0}$ definiert, w"urde sogar gelten $\ScPr{\hat{x}_k}{\hat{x}_k} = 1$ und die $\hat{x}_k$ bilden eine \emph{Orthonormalsystem}.

Wenn wir nun noch einmal obige Gleichung f"ur $\ScPr{x}{x_\ell}$ betrachten, finden wir
\[
\ScPr{x}{x_\ell} 
    = \Sum{k \in \Z}{}{c_k \ScPr{x_k}{x_\ell}}
    = T_0 c_\ell,
\]
weil alle Summanden durch die Orthogonalit"at verschwinden und nur im Falle von $k = \ell$ eben $T_0$ "ubrig bleibt.
Der Fourier-Koeffzient $c_\ell$ ergibt sich also als Skalarprodukt von $x$ mit dem zugeh"origen $x_\ell$.
Es ist wichtig zu sehen, dass wir nur f"ur die Berechnung von $\ScPr{x_k}{x_\ell}$ die spezielle Form der $x_k$ eingesetzt haben und sonst nur allgemein mit den Eigenschaften von Skalarprodukten gearbeitet haben.
Das hei"st, dass man ganz allgemein Signale \q{transformieren} beziehungsweise \emph{analysieren} kann, indem man sie als Summe von paarweise orthogonalen Signalen ausdr"uckt.
Die Fourier-Reihe ist nur ein Spezialfall von diesem allgemeinen Konzept!

\begin{listing}[h]
    \noindent
    \begin{minipage}{0.51\textwidth}
        \strut\vspace*{-\baselineskip}\newline
        \inputminted[firstline=10, lastline=44]{python3}{code/fourier_series.py}
    \end{minipage}%
    \begin{minipage}{0.48\textwidth}
        \strut\vspace*{-\baselineskip}\newline
        \includegraphics[width=\textwidth]{code/fourier_series.png}
    \end{minipage}
    \codecaption{dsv/code/fourier_series.py}{Berechnung und Darstellung von \eqref{eq:fourier:fourier_series}}\label{py:fourier_series}
\end{listing}

In \Cref{py:fourier_series} wird eine Rechteckfunktion $\Rect_{[0,T]}$ in ihre Fourier-Reihe entwickelt.
In diesem Fall m"ussen wir nat"urlich die Reihenentwicklung abbrechen, da kein $K_{\rm max}$ existiert, sodass $c[k] = 0$ f"ur $k > K_{\rm max}$.
Wir k"onnten die Reihe zwar analytisch ausrechnen, da wir jedes $x_k$ nur auf $[0,T]$ integrieren, also dem Bereich, auf dem die von uns definierte $\Rect_{[0,T]}$-Funktion Werte ungleich $0$ annimmt.
Der Einfachheit halber nutzen wir \texttt{scipy.integrate.quad}, was in der Lage ist numerische Integration relativ pr"azise durchzuf"uhren.

Es lohnt sich mit dem Wert von $K_{\rm max}$ zu experimentieren. 
Man sieht hierbei, dass gr"o"sere Werte von $K_{\rm max}$ an den Unstetigkeitsstellen $0$ und $T$ und in deren N"ahe nicht zu einer besseren "Ubereinstimmung der Fourierreiehe mit $x$ f"uhren.
Die Fourier-Reihe muss also nicht immer gegen $x$ konvergieren.
Im Falle von $\Rect_{[0,T]}$ ergibt sich das Problem genau aus dem Verhalten an $0$ und $T$ -- also Unstetigkeit, was ein generelles Problem bei der Entwicklung von Signalen in Fourier-Reihen darstellt.

Im oberen Plot von \Cref{py:fourier_series} sieht man auch, dass der Realteil der Koeffizienten $\Re{c[\cdot]}$ ein gerades diskretes Signal ist, also $c[k] = c[-k]$.
Der Imagin"arteil $\Im{c[\cdot]}$ hingegen ist ein ungerade Signal, also $c[k] = -c[-k]$\footnote{siehe \Cref{py:even_odd}}. 
Dies liegt daran, dass das Signal lediglich reelle Werte annimmt, weshalb die Symmetrie-Eigenschaften der $c[\cdot]$ allgemein f"ur reelle Signale gelten.

\begin{figure}
    \begin{center}
        \includegraphics[width=0.8\textwidth]{code/fourier_series_1.png}

        \includegraphics[width=0.8\textwidth]{code/fourier_series_2.png}
    \end{center}
    \caption{Mehr Versionen von \Cref{py:fourier_series}; Oben: $x(t) = \exp(-25(t-0.5)^2) \cos(16 \pi t)$; Unten: $x(t) = \sin(20 \pi t)$;}\label{fig:fourier:fourier_series}
\end{figure}

In \Cref{fig:fourier:fourier_series} sind noch mehr Eigenschaften der Fourier-Reihe deutlich gemacht.
Generell stellen wir fest, dass die beiden Signale jeweils gut durch eine endliche Fourier-Reihe approximiert werden k"onnen, da sie keine Unsteigkeiten aufweisen.
Im oberen Plot sieht man au"serdem, dass Achsen-Symmetrie des Signals $x$ dazu f"uhrt, dass die Imagin"arteile von $c[\cdot]$ verschwinden.
Wie man in \Cref{fig:fourier:fourier_series} unten sieht, ist es bei Anti-Symmetrie des Signals $x$ der Realteil von $c[\cdot]$, der verschwindet.
Dies liegt daran, dass der Realteil der $x_k$ eine gerade Funktion ist und der Imagin"arteil respektive eine ungerade Funktion.
Da wir in \Cref{py:even_odd} schon gesehen haben, dass ungerade und gerade Anteile eines Signals im Sinne von Skalarprodukten orthogonal sind, gilt dies auch f"ur die resultierenden Fourier-Koeffizienten.
In \Cref{fig:fourier:fourier_series} unten sieht man auch, dass man f"ur gewisse Signale die Fourier-Reihe direkt angeben kann.
Im Falle des Beispiels gilt n"amlich
\[
x(t) 
    = \sin(10 \cdot 2 \pi t) 
    = \frac{1}{2 \jmath} \left(
        \exp(10 \cdot 2 \pi t) + \exp(-10 \cdot 2 \pi t)
    \right) 
    = \frac{1}{2 \jmath} x_{10} + \frac{1}{2 \jmath} x_{-10}.
\]
Das hei"st, dass wir $x$ \emph{direkt} in seine Fourier-Reihe entwickelt haben, da wir es als Linearkombination der $x_k$ dargestellt haben.
Das hei"st es sind nur $x_{10}$ und $x_{-10}$ notwendig, um $x$ darzustellen und beide Koeffizienten haben ausschlie"slich imagin"are Anteile.

"Ahnlich wie bei der $z$-Transformation kann man also an der Fourier-Reihe Eigenschaften des Signals $x$ direkt ablesen, oder umgekehrt von Eigenschaften des Signals $x$ auf Eigenschaften der Fourier-Koeffizienten $c[\cdot]$ schlie"sen.
Au"serdem werden wir f"ur die noch folgenden Versionen der \gls{ft} sehr analoge Zusammenh"ange finden.

\subsubsection{Leichtungsdichte-Spektrum periodischer Signale}

Das Leichtungsdichtespektrum eines $T_0$-periodischen Signals $x: \R \rightarrow \C$ is gegeben durch
\begin{equation}\label{eq:fourier:period_psd}
P(x) = \frac{1}{T_0}\Int{0}{T_0}{\Abs{x(t)}^2}{t}
     = \frac{1}{T_0}\Int{0}{T_0}{x(t) x(t)^\ast}{t}
     = \frac{1}{T_0} \ScPr{x}{x}.
\end{equation}
Wir wollen nun $P(x)$ in Abh"angigkeit der Fourier-Koeffizienten $c[\cdot]$ berechnen.
Wir entwickeln also
\[
x(t) = \Sum{k \in \Z}{}{
    c_k \exp(\jmath 2 \pi k F_0 t)
}
\]
und setzen dies in $P$ ein, um
%
\begin{equation}\label{eq:fourier:series_parseval}
    \frac{1}{T_0}\Int{0}{T_0}{x(t) x(t)^\ast}{t}
        = \frac{1}{T_0}\Int{0}{T_0}{
            x(t) 
            \Sum{k \in \Z}{}{
                c_k^\ast \exp(-\jmath 2 \pi k F_0 t)
            }
        }{t}
        = \Sum{k \in \Z}{}{
            c_k^\ast 
            \frac{1}{T_0}\Int{0}{T_0}{
                x(t)
                \exp(-\jmath 2 \pi k F_0 t)
            }{t}
        }
        = \Sum{k \in \Z}{}{
            c_k^\ast c_k
        }
\end{equation}
%
als den \emph{Satz von Parseval}\footnote{\url{https://de.wikipedia.org/wiki/Satz\_von\_Parseval}} zu erhalten.
Die physikalische Interpretation ist, dass $\Abs{c[k]}$ der Leistung des Signals bei der Frequenz $k F_0$ entspricht.
Jeder Index $k$ hat also eine physikalische Gr"o"se, die mit ihm assoziiert ist.

Da nur die Frequenzen $k F_0$ f"ur $k \in \Z$ auftreten, also Frequenzen wie $0.1 F_0$ nicht vorhanden sind, sprechen wir von einem \emph{diskreten Spektrum}.
Es ergibt sich also direkt folgender wichtiger Zusammenhang: Periodische Signale im Zeitbereich besitzen ein \emph{diskretes} Spektrum.
Andersherum ergibt sich auch: Signale mit diskretem Spektrum sind periodisch.
Beide Argument ergeben sich aus der Fourier-Reihe.
Damit haben wir das duale Ergebnis zu \eqref{eq:spectrum_sampled} gefunden.
Dort f"uhrte Diskretisierung im Zeitbereich, also Sampling/Abtastung, zu einer Periodifizierung im Frequenzbereich.

\begin{listing}[h]
    \noindent
    \begin{minipage}{0.51\textwidth}
        \strut\vspace*{-\baselineskip}\newline
        \inputminted[firstline=5, lastline=14]{python3}{code/period_psd.py}
    \end{minipage}%
    \begin{minipage}{0.48\textwidth}
        \strut\vspace*{-\baselineskip}\newline
        \includegraphics[width=\textwidth]{code/period_psd.png}
    \end{minipage}
    \codecaption{dsv/code/period_psd.py}{Modizifiertes \Cref{py:fourier_series} und Plot von $P$ im Frequenzbereich.}\label{py:period_psd}
\end{listing}

In \Cref{py:period_psd} zeigen wir eine Modifikation von \Cref{py:fourier_series}, in welcher wir $T_0 = 2$ setzen, also $F_0=1/2$ erhalten. 
Auf der Frequenzachse sehen wir, dass demnach diese f"ur $K_{\rm max} = 30$ also von $-15$ bis $+15$ reicht.
%
%
\FloatBarrier
\subsubsection{Fourier-Transformation von kontinuierlichen aperiodischen Signalen}
%
Um die Einschr"ankung auf periodische Signale zu vermeiden, nutzen wir die \acrlong{ft}, wie wir sie bereits in \eqref{eq:sampling:fourier_trafo} f"ur ein Signal $x : \R \rightarrow \C$ durch
\begin{equation}\label{eq:fourier:fourier_trafo}
    X(F) = \Int{-\infty}{+\infty}{x(t) \exp(-\jmath 2 \pi F t)}{t}
\end{equation}
definiert haben.
Im Unterschied zu \eqref{eq:fourier:fourier_series} integrieren wir nun "uber ganz $\R$, da sich das Signal nicht mehr periodisch wiederholt.
Aus diesem Grund ergibt sich auch ein \emph{kontinuierliches} Spektrum, da wir uns nicht mehr auf eine abz"ahlbare Menge von diskreten Frequenzen zur"uckziehen k"onnen.
Deshalb ergibt sich auch f"ur die Synthese des Signals, dass wir auch in diesem Fall integrieren anstatt summieren m"ussen, es gilt also
\begin{equation}\label{eq:fourier:inv_fourier_trafo}
    x(t) = \Int{-\infty}{+\infty}{X(F) \exp(\jmath 2 \pi F t)}{F}.
\end{equation}
Da sich f"ur die meisten Signale, die wir betrachten werden, Integration und Summation \q{"ahnlich} verhalten, finden wir auch die obigen Eigenschaften bez"uglich Symmetrien, etc., von \eqref{eq:fourier:fourier_series} wieder.

Es gibt aber eine Verbindung zur Fourier-Reihe, die wir im Folgenden kurz erl"autern wollen.
Nehmen wir an, es existiert ein $T > 0$, sodass $\Abs{x(t)} = 0$ f"ur alle $t$ mit $\Abs{t} > T$.
Dann k"onnen wir das Signal $x$ periodifizieren mit Periode $T$, indem wir
\[
x_p(t) = \Sum{k \in Z}{}{x(t - kT)}
\]
setzen.
Dies haben wir bereits in "ahnlicher Form in \eqref{eq:spectrum_sampled} gesehen. 
Dort hat es sich aber aus Berechnungen ergeben und hier \emph{setzen} wir diesen Zusammenhang explizit.
Dann k"onnen wir $x_p$ in seine Fourier-Reihe 
\[
x_p(t) = \Sum{k \in \Z}{}{c_k \exp(\jmath 2 \pi k t/T)}
\Text{mit}
T\,c_k = \Int{-T/2}{+T/2}{x_p(t) \exp(- \jmath 2 \pi k t/T)}{t}
    = \Int{-\infty}{+\infty}{x(t) \exp(- \jmath 2 \pi k t/T)}{t}
\]
entwickeln.
Aus der Definition in \eqref{eq:fourier:fourier_trafo} und der letzten Gleichung sehen wir, dass sich die Koeffizienten der Fourier-Reihe finden lassen durch
\begin{equation}\label{eq:fourier:c_k_fourier}
    c_k = \frac 1T X\left(\frac kT\right),
\end{equation}
diese sich also auch aus der Fourier-Transformation ablesen lassen.
Das hei"st, dass wir auch
\[
x_p(t) = \Sum{k \in \Z}{}{\frac 1T X\left(\frac kT\right) \exp(\jmath 2 \pi k t/T)}
       = \Sum{k \in \Z}{}{X\left(k \Delta F\right) \exp(\jmath 2 \pi k t \Delta F) \Delta F}
\]
schreiben k"onnen.
Dabei haben wir im letzten Schritt $1/T = \Delta F$ gesetzt.
Dies k"onnen wir intuitiv (aber nicht rigoros!) so interpretieren, dass im Falle von $T \rightarrow \infty$ gilt, dass $\Delta F \rightarrow 0$.
Je l"anger der Bereich der Funktion $x$, auf welchem gilt $x \neq 0$, desto kleiner $\Delta F$.
Der nicht-periodische Fall $T = \infty$ ergibt sich also als Grenzfall, bei welchem obige Summation zu einer Integration wird und $\Delta F$ zu einem $\mathrm{d}F$.
Man kann die \acrlong{ft} also als Grenzfall der Fourier-Reihe f"ur $T \rightarrow \infty$ betrachten.
%
%
\subsubsection{Leistungsdichte-Spektrum aperiodischer Signale}
%
Analog zum Satz von Parseval f"ur periodische Signale in \eqref{eq:fourier:series_parseval} k"onnen wir auch hier wieder definieren und folgern, dass
\begin{equation}
E(x) = \Int{-\infty}{+\infty}{\Abs{x(t)}^2}{t}
     = \Int{-\infty}{+\infty}{\Abs{X(F)}^2}{F}
\end{equation}
auch wieder ein Parseval Theorem f"ur die \acrlong{ft} ergibt.

Andererseits kann man $X$ auch in Betrag und Phase zerlegen, da es im Allgemeinen eine komplexe Gr"o"se ist, also
\[
X(F) = \Abs{X(F)} \exp(\jmath \angle(X(F))).
\]
Man nennt dann $\Abs{X(F)}^2$ das \emph{Leistungsdichte-Spektrum} von $x$.

\begin{listing}[h]
    \noindent
    \begin{minipage}{0.51\textwidth}
        \strut\vspace*{-\baselineskip}\newline
        \inputminted[firstline=6, lastline=45]{python3}{code/fourier_trafo.py}
    \end{minipage}%
    \begin{minipage}{0.48\textwidth}
        \strut\vspace*{-\baselineskip}\newline
        \includegraphics[width=\textwidth]{code/fourier_trafo.png}
    \end{minipage}
    \codecaption{dsv/code/fourier_trafo.py}{Berechnung und Darstellung von \eqref{eq:fourier:fourier_trafo}}\label{py:fourier_trafo}
\end{listing}

In \Cref{py:fourier_trafo} zeigen wir das Vorgehen zur Fourier-Analyse mittels numerischer Integration.
Es ist hier anzumerken, dass wir \q{nur} die Funktion \texttt{rect} definieren m"ussen und der Rest, also die Funktionen \texttt{kernel}, \texttt{analyse} und \texttt{synthese} unabh"angig hiervon sind.
Wir haben hier ein \namecref{py:fourier_trafo}, welches sich Methoden der Funktionalen Programmierung bedient.
Die Funktionen texttt{analyse} und \texttt{synthese} haben als Eingabewert die entweder die Funktion, oder die \acrlong{ft} einer Funktion.
Au"serdem liefern sie als Ausgabewert wieder \emph{Funktionen}, die wir einfach \q{aufrufen} k"onnen.
%
\subsection{Fourier-Transformation diskreter Signale}\label{sec:fourier:disc}
%
Wir n"ahern uns langsam der Fourier-Analyse von diskreten Signalen.
Schlie"slich wollen wir etwaige spektrale Analysen und dergleichen im Digitalen durchf"uhren, um die Vorz"uge von digitalen Rechenwerken dabei nutzen zu k"onnen.
Die vorher eingef"uhrten Transformationen sind zwar hilfreich f"ur theoretische Argumentation, wie beispielsweise beim Sampling-\Cref{stm:sampling_theorem}. 
Deshalb wenden wir uns nun der spektralen Analyse von diskreten Signalen zu.
Wir werden aber im Verlauf auch wieder "ahnliche Zusammenh"ange wie in \eqref{eq:fourier:c_k_fourier} finden.
%
\subsubsection{Fourier-Transformation diskreter periodischer Signale}\label{sec:fourier:disc_period}
%
Wir beginnen mit diskreten Signalen, die gleichzeitig periodisch sind, also ein $N$ existiert, sodass
\[
x[n] = x[n+N] \Text{f"ur alle} n \in \Z 
\]
gilt.
Aus vorherigen Diskussionen in \Cref{sec:sampling} wissen wir einerseits, dass diskrete Signale ein periodisches Spektrum auf $(0,1)$ besitzen.
Andererseits wissen wir aus \Cref{sec:fourier:cont:period}, dass periodische Signale ein \emph{diskretes} Spektrum besitzen.
Wir finden also nun intuitiv, dass das Spektrum von diskreten periodischen Signalen \emph{ebenfalls} diskret und periodisch ist.
Denken wir zur"uck an \Cref{sec:sampling:disc_sin} so haben wir bereits alles Notwendige betrachtet.
Ein $N$-periodisches diskretes Signal ergibt sich aus der Linearkombination der diskreten Signale $x_k[\cdot] : \Z \rightarrow \C$ definiert durch
\[
x_k[n] = \exp\left(\jmath 2 \pi \frac k N n \right) \Text{mit} k = 0, \ldots, N-1.
\]
Das hei"st, f"ur $x[\cdot]$ setzen wir mit
\[
x[\cdot] = \Sum{k = 0}{N-1}{c[k] x_k[\cdot]}
\]
an.
Damit sind bereits beide Eigenschaften des Spektrums \q{eingepreist}.
Denn man erkennt in den $x_{k}[\cdot]$ die vorher erw"ahnte \emph{zweifache} Periodizit"at wieder, weil sowohl $x_{k+N}[\cdot] = x_{k}[\cdot]$ f"ur alle $k$ als auch $x_k[n+N] = x_k[n]$ f"ur alle $n$ gilt.

Es ist nun unser Ziel f"ur gegebene Werte $x[n]$ von $x[\cdot]$ die Werte des \emph{ebenfalls periodischen und diskreten} Signals $c[\cdot]$ zu bestimmen.
Dies verl"auft ganz analog zu \Cref{sec:fourier:cont:period}.
Wir definieren das Skalarprodukt $\ScPr{\cdot}{\cdot}$ f"ur $N$-periodische und diskrete Signale via
\[
\ScPr{x_1[\cdot]}{x_2[\cdot]} 
    = \Sum{n = 0}{N-1}{x_1[n] x_2[n]^\ast}.
\]
Das hei"st, dass wir periodisches Signal $x[\cdot]$ mit dem \emph{endlich-dimensionalen} Vektor $\bm x \in \C^{N}$ identifizieren, wir setzen also die Eintr"age des Vektors als $\bm x_i = x[i-1]$.
Dann k"onnen wir auch f"ur das entsprechende Skalarprodukt der Vektoren $\bm x_{1,2} \in \C^N$
\[
\ScPr{\bm x_1}{\bm x_2} 
    = \left(\bm x_2^\ast\right)^\trans \cdot \bm x_1 
    = \bm x_2^\herm \cdot \bm x_1
\]
schreiben.
Analog identifizieren wir die periodische und diskrete Sequenz $c[\cdot]$ mit dem Vektor $\bm c \in \C^N$.

Wie in \Cref{sec:fourier:cont:period} m"ussen wir nur 
\[
\ScPr{\bm x_k}{\bm x_\ell} 
    = \Sum{i=0}{N-1}{
        x_k[i] x_\ell[i]^\ast
    }
    = \Sum{i=0}{N-1}{
        \exp\left(\jmath 2 \pi \frac{i(k-l)}{N} \right)
    }
    = \begin{cases}
        N \Text{falls} k = \ell \\
        \frac{
            1 - \exp\left(\jmath 2 \pi \frac{(k-l)}{N} \right)^N   
        }{
            1 - \exp\left(\jmath 2 \pi \frac{(k-l)}{N} \right)
        } \Text{sonst}
    \end{cases}
    = \begin{cases}
        N \Text{falls} k = \ell \\
        0 \Text{sonst}
    \end{cases}
\]
berechnen.
Hierbei nutzen wir, dass $\exp(\jmath 2 \pi k) = 1$ f"ur alle $k \in \Z$.
Wie in \Cref{sec:fourier:cont:period} finden wir, dass also gilt $\ScPr{\bm x_k}{\bm x_\ell} = 0$, falls $k \neq \ell$ und $\ScPr{\bm x_k}{\bm x_k} = N$.
Damit ergibt sich bei Anwendung auf das eigentliche Signal $\bm x$ f"ur den Vektor $\bm c$, dass
\[
\ScPr{\bm x}{\bm x_\ell}
    = \ScPr{\Sum{k=1}{N}{\bm c_k \bm x_k}}{\bm x_\ell}
    = \Sum{k=1}{N}{\bm c_k \ScPr{\bm x_k}{\bm x_\ell}}
    = N \bm c_\ell
\Rightarrow
\bm c_\ell = \frac{1}{N}\ScPr{\bm x}{\bm x_\ell}.
\]
Zusammenfassend, kann man also sagen, dass sich folgende Analyse- und Synthesegleichungen ergeben:
\begin{equation}\label{eq:fourier:disc_analys_synth}
    x[n] = \Sum{k = 0}{N-1}{c[k]\exp(\jmath 2 \pi k n/N)}, \Text{und}
    c[k] = \frac{1}{N}\Sum{n = 0}{N-1}{x[n]\exp(-\jmath 2 \pi k n/N)}.
\end{equation}
In Vektorschreibweise l"asst sich der erste Teil durch
\begin{equation}\label{eq:fourier:disc_analys_synth_vec}
    \bm x = \Sum{k = 0}{N-1}{\ScPr{\bm x}{\bm x_k} \bm x_k}
\end{equation}
ausdr"ucken.

Weiterhin, k"onnen wir eine Matrix $\bm F_N \in \C^{N \times N}$ definieren, deren $k$-te Spalte den Vektor $\bm x_k$ beinhaltet.
Wir definieren also 
\[
\bm F_N = \left[
    \bm x_1, \ldots, \bm x_k, \ldots, \bm x_N 
\right]
\]
Dann k"onnen wir noch einen Schritt weitergehen und sehen, dass
\[
\bm c = \frac{1}{N} \bm F_N^\herm \bm x, \Text{und} \bm x = \bm F_N \bm c
\]
gilt.
Das hei"st, dass sich Fourier-Analyse und Fourier-Synthere von diskreten und periodischen Signalen durch eine Matrix-Vektor-Multiplikation durchf"uhren l"asst!
Das sind erst einmal gute Nachrichten, denn damit wissen wir, dass digitale Rechenwerke sehr gut darin sind, diese Transformation durchzuf"uhren.
Die hier vorgestellte Transformation nennen wir \gls{dft}.

In \Cref{py:dft_1} zeigen wir ein einfaches Beispiel f"ur $x[\cdot] = u[\cdot] - u[\cdot-k]$.
Au"serdem zeigen wir auch die beiden Berechungsmethoden der $c[\cdot]$ -- einerseits mit der Definition und andererseits "uber ein Matrix-Vektor-Produkt.
%
\begin{listing}[h]
    \noindent
    \begin{minipage}{0.51\textwidth}
        \strut\vspace*{-\baselineskip}\newline
        \inputminted[firstline=5, lastline=22]{python3}{code/dft_1.py}
    \end{minipage}%
    \begin{minipage}{0.48\textwidth}
        \strut\vspace*{-\baselineskip}\newline
        \includegraphics[width=\textwidth]{code/dft_1.png}
    \end{minipage}
    \codecaption{dsv/code/dft_1.py}{Berechnung und Darstellung von \eqref{eq:fourier:disc_analys_synth}}\label{py:dft_1}
\end{listing}
%
\FloatBarrier
%
\subsubsection{Fourier-Transformation diskreter aperiodischer Signale}\label{sec:fourier:disc_aperiod}
%
Wiederum analog zu \Cref{sec:fourier:cont} ben"otigen wir noch eine Transformation f"ur diskrete Signale, die keine Periodizit"at aufweisen.
Hierzu definieren wir die zugeh"orige Fourier-Transformation, die wir \gls{dtft} nennen, durch
\begin{equation}\label{eq:fourier:dtft}
    X(f) = \Sum{n \in \Z}{}{x[n] \exp(-\jmath 2 \pi f n)},
\end{equation}
genau wie in \Cref{sec:sampling} bei der Herleitung von \eqref{stm:sampling_theorem}.
Im Unterschied zur \q{analogen} Fourier-Transformation \eqref{eq:fourier:fourier_trafo} sehen wir, dass $X$ nur f"ur Frequenzen $f \in (0,1]$ definiert ist, da das diskrete Signal $x_f[n] = \exp(\jmath 2 \pi f n)$ periodisch in $f$ ist.
Es gilt also $x_{f + k}[\cdot] = x_{f}[\cdot]$ f"ur alle $k \in \Z$.
Dies \q{passt} auch zur Natur von diskreten Signalen, da deren Frequenzbereich \emph{immer} periodisch sein muss.

Wir k"onnen nun f"ur die inverse Transformation einmal \eqref{eq:fourier:dtft} mit \eqref{eq:fourier:fourier_series} vergleichen. 
Bis auf das Vorzeichen in der Funktion $\exp()$ gleicht \eqref{eq:fourier:dtft} einer Fourier-Reihe der periodischen Funktion $X$.
In der Tat, k"onnen wir die Folge $x[\cdot]$ als Fourier-Koeffizienten der Funktion $X$ durch
\begin{equation}\label{eq:fourier:idtft}
    x[n] = \Int{-1/2}{+1/2}{X(f)\exp(\jmath 2 \pi f n)}{f}
\end{equation}
wiederfinden.
Diese Synthese-Operation nennen wir dann \gls{idtft}.
Es ist wiederum anzumerken, dass die \gls{dtft} \q{nur} ein theoretisches Werkzeug ist, da sich im Allgemeinen die unendliche Summe in \eqref{eq:fourier:dtft} praktisch nicht realisieren l"asst, genauso wenig wie deren Resultat, eine kontinuierliche Funktion.

In \Cref{py:dtft} zeigen wir dennoch, wie man die \gls{dtft} einer aperiodischen Folge approximieren kann.
Wir studieren hierzu das Signal
\[
x[n] = \begin{cases}
    \frac{\omega}{\pi} \Text{f"ur} n = 0,\\
    \frac{\omega}{\pi} \frac{\sin(\omega n)}{\omega n} \Text{sonst}.
\end{cases}
\]
Dessen analytisch bestimmte \gls{dtft} $X$ ist
\[
X(f) = \Rect(f/(2 \pi \omega)),
\]
welche wir durch eine endliche Summe "uber die von uns verf"ugbaren Werte von $x[\cdot]$ bestimmen.

Nun treten hierbei mehrere Effekte zutage.
\begin{itemize}
    \item Die Approximation der \gls{dtft} mittels des \q{abgeschnittenen} Signals $x$ stimmt noch nicht sehr gut mit der analytischen L"osung "uberein. 
    Ver"andert man den Wert der Variable $N$ im Skript, tritt dieser Effekt st"arker oder schw"acher zutage.
    \item Auch bei Erh"ohung von $N$ stellt sich immernoch keine Konvergenz von $X_{\rm approx}$ gegen $X_{\rm true}$ ein.
    Dies liegt daran, dass die \gls{dtftf} von $x[\cdot]$ nicht konvergiert, wie wir in \Cref{py:fourier_series} bereits gesehen hatten.
    Diesen Effekt bezeichnet man allgemein als Gibbssches Phänomen\footnote{\url{https://de.wikipedia.org/wiki/Gibbssches_Phänomen}}.
    \item Augenscheinlich ist es besser die \gls{idtft} aus $X_{\rm approx}$ zu berechnen, als aus $X_{\rm true}$. Doch dies liegt legidlich daran, dass die numerische Integration der $\Rect$-Funktion nicht genau genug ist.
    Erh"oht man die Anzahl der Punkte im Array \texttt{F}, dann stimmen im dritten Plot die drei Graphen besser "uberein.
\end{itemize}
\begin{listing}[h]
    \noindent
    \begin{minipage}{0.51\textwidth}
        \strut\vspace*{-\baselineskip}\newline
        \inputminted[firstline=5, lastline=48]{python3}{code/dtft.py}
    \end{minipage}%
    \begin{minipage}{0.48\textwidth}
        \strut\vspace*{-\baselineskip}\newline
        \includegraphics[width=\textwidth]{code/dtft.png}
    \end{minipage}
    \codecaption{dsv/code/dtft.py}{Berechnung und Darstellung von \eqref{eq:fourier:dtft}}\label{py:dtft}
\end{listing}
\FloatBarrier
\begin{itemize}
    \item \begin{itemize}
        \item eigenschaften der kernel-funktion: perioden, wann reel, ..., was passiert bei abtastung?
        \item zusammenhang zur physik
        \item dc, nyquist-frequenzen, wann vorhanden?
        \item reelles signal: dc und nyquist reell
    \end{itemize}
    \item dft als approximation der dtft
    \begin{itemize}
        \item $s(t) = \exp(-t/\tau) \cdot \sin(\omega_0 \cdot t)$, 
        \item $s(t) = \exp(-(t - \mu)^2/\sigma^2)$ (Uebung)
    \end{itemize}
    \item spectral leakage;
\end{itemize}
\subsection{Anwendung der Transformationen}
\begin{itemize}
    \item fft (Uebung 1 jup, uebung 2 8.3.2)
    \item bild-kompression (DCT, JPEG)
    \item fensterung \footnote{\url{https://docs.scipy.org/doc/scipy/reference/signal.windows.html}}
    \item schnelle faltung (cyclic/non-cyclic)
    \item beliebig lange signale overlap add, overlap save (\"Ubung: irgendwas mit audio prosessing)
    \item Unschaerfe Relation (Fensterbreite vs. Frequenzaufloesung)\cite[chpt. 4.2]{mallat2008wavelets}
\end{itemize}

%
%
\section{Multiraten Systeme}\label{multirate}
%
\begin{itemize}
    \item system das signale in mehreren raten gleichzeitig benutzt
    \item erste option: digital zu analog, filtenr, analog zu digital
    \item vorteil: sampling rate kann beliebig (je nach signal bandbreite) veraendert werden
    \item nachteil: distortion von da, ad conversion introduce noise, quantization errors
    \item also: conversion direkt in digitaler domaene
    \item problem: how? wir haben das signal nur auf diskreten werten gegeben
    \item wir koennen aber mal die formel fuer perfekte rekonstruktion aufschreiben, die das richtige tut unter der bedingung, dass sampling rate nyquist einhaelt
    \item dann koennen wir diese interpolation einfach auf den neuen samplingpunkten auswerten
    \item resampling ist immerhin schonmal ein lineares system
    \item falls neue samplingrate kleiner, muessen wir auch hier noch nyquist einhalten, oder eben die formel selbst lowpass-filtern
    \item falls alte samplingrate = neue samplingrate haben wir ein LTI-system (uebung)
    \item wenn die samplingrate beliebgi sind, muss man genau hinschauen.
    \item erstmal bild malen
    \item dann gleichungen rumwursten
    \item lineares zeitvariantes system, also impulsantwort ist zeitabhaengig
    \item dann nochmal bild anschauen
    \item fuer rationales verhaeltnis: intuitiv kommt man je nach evrhaltnis wieder auf die gleichen fractional parts, also ist das resampling ein periodisch zeitvariantes lineares system 
    \item downsampling um faktor $D$ (in uebung programmieren, einmal mit weglassen und einmal als faltung)
    \item upsampling um faktor $I$ liefert $I$ verschiedene Impulsantworten des systems
    \item beispiel mit $I = 2$ in der vorlesung (Figure 11.1.4.!), beispiel mit $I = 3$ in uebung. einmal ohne interleaving als periodisch zeitvariantes systen, einmal mit interleaving
\end{itemize}
%
%
\section{STFT}\label{stft}
%
Als erstes wollen wir uns mit der Fourier-Analyse von Signalen besch"aftigen.
Hierbei ist das Ziel das Verhalten eines Signals im Frequenzbereich zu charakterisieren.
Wir wollen jedoch in unserer Analyse davon ausgehen, dass wir ein Signal $x[\cdot]$ vorliegen haben, dessen spektrale Eigenschaften \emph{zeitvariant} sind.
Hiermit ist \emph{nicht} gemeint, dass wir davon ausgehen, dass das Signal zeitvariant ist, denn das ist es nat"urlich.
Es geht darum, dass der Anteil der verschiedenen Frequenzen in einem Signal sich mit der Zeit "andert.
Nat"urlich ist hier eine der am einfachsten zug"anglichen Anwendungen die Analyse von Audiosignalen.

\begin{listing}[ht]
    \noindent
    \begin{minipage}{0.51\textwidth}
        \strut\vspace*{-\baselineskip}\newline
        \inputminted[firstline=10, lastline=44]{python3}{code/stft_1.py}
    \end{minipage}%
    \begin{minipage}{0.48\textwidth}
        \strut\vspace*{-\baselineskip}\newline
        \includegraphics[width=\textwidth]{code/stft_1.png}

\begin{minted}{python}
[ 48.2811  96.5622 149.6715 197.9527]
[ 53.109  111.0466 164.1559]
[  4.8281  62.7654 125.5309 183.4683]
[ 62.7654 130.3590 197.9527]
[ 72.4217 149.6715]
[ 53.109   82.0779 164.1559]
[ 53.109   91.7341 183.4683]
[ 53.109   96.5622 197.95270418]
\end{minted}
    \end{minipage}
    \codecaption{dsv/code/stft_1.py}{Analyse einer \q{Bassline}, die eine G-Dur-Tonleiter spielt.}\label{py:stft_1}
\end{listing}


%
%
\section{Wavelets}\label{wavelets}
%
\begin{itemize}
    \item motivation: analyse in zeit und frequenz gleichzeitig
    \item definition \cite[chpt 4.3, (4.30)]{mallat2008wavelets}
    \item convolution, i.e. linear filter
    \item real heisenberg uncertainty: heisenbergboxes
    \item mexican hat wavelet: second derivative of gaussian
    \item gabor wavelets (uebung)
    \item discrete wavelet transform \cite[ch 4.3.3]{mallat2008wavelets}
\end{itemize}
%
%
\section{B-Splines}\label{bsplines}
%
% !TEX root = ../dsv_script.tex
%
Eine Grundvoraussetzung f\"ur eine praktisch n\"utzliche digitale Signalverarbeitung ist die M\"oglichkeit zwischen dem analogen und digitalen Bereich wechseln zu k\"onnen. Hierbei sollte man auch genau quantifizieren k\"onnen, ob bei diesem Prozess Informationen verloren gehen, oder wie man garantieren kann, dass diese Umwandlung verlustfrei vonstatten geht. Meist nutzt man hierf\"ur das Nyquist-Shannon Sampling Theorem~\cite[Kapitel~1.4.2]{proakis2013}. Die hieraus resultierende sogenannte Nyquist-Sampling-Theorie fu{\ss}t bekannterma{\ss}en auf der Repr\"asentation von bandbegrenzten Signalen durch hinreichend dichte \"aquidistante Abtastwerte. Diese Theorie ist gut studiert, in Textb\"uchern aufbereitet und bildet die Grundlage f\"ur viele Messsysteme und Algorithmen im Digitalen, siehe beispielsweise~\cref{eadf-interpol}.

Es gibt jedoch auch einige Nachteile von Nyquist-Sampling, die aus dessen Annahmen und der daraus folgenden Verarbeitung entstehen. Einerseits kann ein endliches Signal im Allgemeinen \emph{nicht} bandbegrenzt sein. Weiterhin entstehen durch die Bandlimitierung von Signalen Gibbs-Artefakte, die besonders in Bildern nicht erw\"unscht sind. Geht es um die Auswertung $x(t)$ eines Signals $x$ zwischen den aufgenommenen Samples $x[n]$, also Interpolation, hat man das Problem, dass die $\Sinc$-Funktion nur sehr langsam mit Rate $1/t$ abf\"allt. Diese Eigenschaft f\"uhrt dazu, dass man f\"ur die Bestimmung eines Wertes $x(t)$ mit einer Genauigkeit von \SI{1}{\percent} etwa $100$ um $t$ benachbarte Samples betrachten muss. Das hei{\ss}t, vor allem bei 2D-Interpolation skaliert der resultierende Rechenaufwand nicht sehr g\"unstig, falls hohe Genauigkeit ben\"otigt wird.

Aus diesem Grund m\"ochten wir uns eine alternative Sampling-Theorie genauer ansehen -- die B-Splines~\cite{unser1999splines_mag}. Wir f\"uhren zun\"achst die auf Polynomen basierende Signalverarbeitung ein und vergleichen sie anschlie{\ss}end zur bereits bekannten Nyquist-Theorie.

\subsection{B-Splines als Polynome}

Allgemein bezeichnet man st\"uckweise definierte und stetig differenzierbare Polynome als Splines. Man bezeichnet die Stellen an denen zwei unterschiedliche Polynome zusammensto{\ss}en als Knoten. Ein Spline der Ordnung $\ell \in \N$ ist ein Polynom vom Grad $\ell$, ist also von der Form
\begin{equation}
    p(x) = 
        a_{\ell} t^{\ell} 
        + a_{\ell - 1} t^{\ell-1} 
        + \dots
        + a_1 t 
        + a_{0}.
\end{equation}
%
Ein Spline ist nun eine Funktion $s(t)$, welche f\"ur Knoten $n = 1, 2, \dots$ definiert ist durch
\begin{equation}
    s(t) = \begin{cases}
        p_1(t) \fuer x \in [1,2], \\
        p_2(t) \fuer x \in [2,3], \\
        \vdots
    \end{cases}
\end{equation}
wobei sich die Glattheit durch die Forderung ergibt, dass die Funktion und ihre Ableitungen an den Knoten stetig sei, also
\begin{equation}
    \lim\limits_{t \rightarrow n^-} s^{(m)}(t) =
    \lim\limits_{t \rightarrow n^+} s^{(m)}(t)
\end{equation}
erf\"ullt ist, wobei $s^{(m)}$ f\"ur $m \geqslant 0$ die $m$-te Ableitung des Splines $s$ repr\"asentiert. In einer Arbeit~\cite{schoenberg1988bsplines}, die sogar dem ber\"uhmten Paper von Shannon vorausgeht, beschreibt Schoenberg, dass sich diese Splines der Ordnung $\ell$ via
\begin{equation}\label{bsplines_summation}
    s(t) = \Sum{k \in \Z}{}{
        c[k] \beta^{\ell}(t - k)
    }
\end{equation}
darstellen lassen. Hierbei ist die Funktion $\beta^\ell: \R \rightarrow \R$ definiert als eine iterierte Faltung einer Rechteck-Funktion via
\begin{equation}
    \beta^\ell = \underbrace{
        \beta^0 \ast \dots \ast \beta^0
    }_{(\ell+1) \Text{mal}}, 
    \Text{wobei}
    \beta^0(t) = \begin{cases}
        1,\quad \Abs{t} < \frac{1}{2} \\
        \frac{1}{2}, \quad \Abs{t} = \frac{1}{2} \\
        0, \Text{sonst.}
    \end{cases}
\end{equation}
%
\begin{figure}
    \caption{}\label{bsplines_order3}
\end{figure}
%
In \cref{bsplines_order3} sind die Funktionen $\beta^\ell$ f\"ur $\ell = 0, \dots, 3$ dargestellt. Man erkennt sehr gut, dass der Grad der Glattheit von der Ordnung des Splines abh\"angt und dass die Funktionswerte $\beta^\ell(t)$ f\"ur $\Abs{t}>\ell+1/2$ verschwinden. Man spricht von Funktionen mit kompaktem Tr\"ager.

Wir wollen nun eine explizite Formel f\"ur $\beta^\ell$ entwickeln. Hierzu betrachten wir die Fourier-Transformation 
\begin{equation}
    B^\ell(\omega) 
        = \left(\frac{sin(\omega / 2)}{(\omega / 2)}\right)^{\ell + 1}
        = \frac{(\exp(\jmath \omega/2) - \exp(-\jmath \omega/2))^{\ell+1}}{
            (\jmath \omega)^{\ell + 1}
        }
\end{equation}
mit einigen Rechentricks (siehe \cite[Box 1.]{unser1999splines_mag}) kann man dies so lange umformen, bis man
\begin{equation}
    \beta^\ell(t) = \frac{1}{\ell !}\Sum{p = 0}{\ell+1}{
            \binom{\ell+1}{p}(-1)^p
            \left(t - p + \frac{\ell+1}{2}\right)_+^\ell
        }
    \Text{mit}
    (x)_+ = \begin{cases}
      x, \fuer x \geqslant 0, \\
      0, \Text{sonst},  
    \end{cases}
\end{equation}
erh\"alt. Damit ist $\beta^\ell$ wirklich ein Polynom $\ell$-ten Grades. Die Stetig- und Differenzierbarkeit muss man sich aber noch separat \"uberlegen. 

Weiterhin kann man zeigen, dass folgende Formeln f\"ur Differentiation und Integration von B-Splines gelten:
\begin{equation}\label{bsplines_deriv_int}
    \left(\beta^\ell\right)^\prime(t) =
        \beta^{\ell-1}(x + 1/2) - \beta^{\ell-1}(x - 1/2), 
    \Int{-\infty}{t}{\beta^\ell(s)}{s} = 
        \Sum{p = 0}{+\infty}{\beta^{\ell+1}(t - 1/2 - p)}
\end{equation}
Das he{\ss}t, dass man auch einen kompletten Spline $s$ differenzieren und integrieren kann, indem man nutzt, dass sowohl Differentiation, als auch Integration linear sind. Es gilt also mit \eqref{bsplines_summation} und \eqref{bsplines_deriv_int}:
\begin{equation}
    s^\prime(t) = \Sum{k \in \Z}{}{
        c[k] \left(\beta^{\ell}\right)^\prime(t - k)
    }
    = \Sum{k \in \Z}{}{
        c[k]\left(
            \beta^{\ell-1}(x + 1/2 - k) - \beta^{\ell-1}(x - 1/2 - k)
        \right)
    }.
\end{equation}
%
%
\subsection{Kubische B-Spline Interpolation}
%
%
Wir m\"ochten uns eine spezielle Version der B-Splines genauer Ansehen, da sie in der Anwendung den Spagat zwischen Komplexit\"at und Approximationsg\"ute sehr gut hinbekommen. Wir setzen hierzu $\ell=3$ und erhalten somit ein Polynom dritten Grades der Form
\begin{equation}
    \beta^3(t) = \begin{cases}
        \frac 23 - \Abs{x}^2 + \frac{\Abs{x}^3}{2} \fuer \Abs{x} < 1\\
        \frac{(2 - \Abs{x})^3}{6}, \fuer \Abs{x} \in [1, 2) \\
        0, \fuer \Abs{x} > 2.
    \end{cases}
\end{equation}
In Analogie zum bekannten Nyquist-Sampling wollen wir untersuchen, wie man nun aus gegebenen Abtastwerten $x[n]$ eine Darstellung wie in \eqref{bsplines_summation} herleiten k\"onnen, welche die abgetasteten Werte exakt interpoliert.
%
%
\section{Zuf\"allige Signale}\label{random}
%
\begin{itemize}
    \item zufallsgroessen: diskret, kontinuierlich (Uebung: laplace, binomial, normal)
    \item dichte von kontinuierlichen zufallsgroessen (Uebung: normalverteilung)
    \item fourier-trafo von dichten
    \item power estimation
    \item quantization
\end{itemize}
%
%
%
%
%
\clearpage
\microtypesetup{protrusion=false}
\addcontentsline{toc}{section}{Abk"urzungsverzeichnis}
\printglossary[type=\acronymtype]
\microtypesetup{protrusion=true}
%
%
%
\clearpage
\microtypesetup{protrusion=false}
\addcontentsline{toc}{section}{Literaturverzeichnis}
\printbibliography
\microtypesetup{protrusion=true}
\end{document}