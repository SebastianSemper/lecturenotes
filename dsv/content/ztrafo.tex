\begin{itemize}
    \item Wir wollen diskrete \gls{lti} systeme analysieren
    \item Wir wollen gewisse signale vlt kompakter aufschreiben koennen
    \item Definition, ROC, man kann an koeffizienten von $z^n$ die werte an zeitpunkt $n$ ablesen
    \item Beispiele mit endlichen signalen, tabelle \cite[p155, top]{proakis2013}
    \item beispiel mit unendlichem signal
    \item analyse von ROC durch betrachtung von $\Abs{X(z)}$, tabelle \cite[p155, top]{proakis2013}
    \item eigenschaften: linear, timeshift (intuition mit koeffizienten), zeitumkehrung(intuition mit spiegelung am einheitskreis), ableitung in $z$-Bereich (analogie zu fourier)
    \item faltungseigenschaft
    \item algo: beide sequenzen $z$-trafo, multiplikation, inverse $z$-trafo
    \item inverse: tabelle \cite[tabelle 3.3]{proakis2013}, allgemeine invertierung: schwierig
    \item alle eigenschaften: \cite[tabelle 3.2]{proakis2013}
    \item anwendung: korrelation von signalen
    \item uebung: problem 3.6 in \cite{proakis2013}, initial value theorem, some plots aus \cite{proakis2013}
    \item rationale z-trafos: definition, pole-nullstellen, umkehrung: von pole-nullstellen zu $X(z)$
    \item ausfuehrliche diskussion von \cite[fig 3.3.5, 3.3.6]{proakis2013} 
    \item Anwendung: M-Sequenzen?
\end{itemize}

