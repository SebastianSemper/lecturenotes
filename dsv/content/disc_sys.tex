
\cite[chp. 2.1]{proakis2013}


\subsection{Diskrete Signale}
\begin{itemize}
    \item darstellung: funktion, tabelle, folge, grafisch, $z$-trafo (\Cref{ztrafo})alle gleichwertig, aber verschieden nuetzlich, je nach situation
    \item beispiele: delta, heavyside, ramp, exponential
    \item energie in signal, durchschnittliche leistung, $E < \infty \Rightarrow P = 0$
    \item beispiel: heavyside, 
    \item \"Ubung: Problem 2.1, 2.2, 2.3
    \item periodic, aperiodic
    \item symmetrisch, anti-symmetrisch
    \item manipulation: verzoegerung ($+-$), skalierung, addition, multiplikation
\end{itemize}

\subsection{Diskrete Systeme}
\begin{itemize}
    \item JEDE manipulation eines signals kann als system aufgefasst werden
    \item system $\mathcal{T}$ ist als abbildung zu verstehen. signal rein, anderes signal raus: $y[n] = (\mathcal{T}x)[n]$ (auf bedeutung von schreibweise eingehen)
    \item beispiele: \cite[ex. 2.2.1]{proakis2013} (a-c vorlesung, rest uebung)
    \item vergangenheit der systeme: akkumulator
    \item blockschaltbilder: adder, multiplikator, scaler, delay, uebung \cite[ex 2.2.3]{proakis2013}(per hand, per modularem code)
    \item statisch vs. dynamisch (gedaechtnislos, mit gedaechtnis)
    \item causal, antikausal, relaxed state
    \item stabilitaet
    \item zeitvariant vs zeitinvariant: definition
\end{itemize}

\subsection{Diskrete \texorpdfstring{\acrshort{lti}}{LTI} Systeme}\label{disc_lti}

\begin{itemize}
    \item wiederholung der eigenschaften: shift-invarianz, linearitaet
    \item reaktion auf signal, das aus bausteinsignalen zusammengesetzt wurde
    \item darstellung als summe von skalierten, geshifteten einheitsstoessen
    \item faltungsformel
    \item algorithmus zur berechnung
    \item exerzieren von beispiel 2.3.2, uebung programmieren (naiv summieren, scipy convolve)
    \item exerzieren von beispiel 2.3.3, uebung programmieren der approximation
    \item assoziativitaet, kommmutativitaet
    \item cascadierung von mehreren systemen
    \item stabilitaet: $h[n]$ muss absolut summierbar sein, gegen $0$ gehen
    \item beispiel 2.3.6. (fuer uebung irgendwas ausdenken)
\end{itemize}

\subsection{Cross- und Autokorrelation}\label{corr}

\begin{itemize}
    \item Definition
    \item eigenschaften
    \item synthetisches beispiel schwingung + noise, akf zeigt periodizitaet
    \item beispiel mit woelfer sunspot numbers
    \item beispiel mit m-sequenzen, niklas einladen, schaltung zeigen
    \item monster uebung: 2.65
\end{itemize}