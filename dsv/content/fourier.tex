\begin{itemize}
    \item Fourier Transform (theoretisches tool, weil digital nicht umsetzbar): kernel, eigenschaften (linear, shift, skalierung), beispiele uebung 
    \item Discrete Time Fourier Transform (theoretisches tool, weil digital nicht umsetzbar)
    \item Discrete Fourier Transform
    \begin{itemize}
        \item eigenschaften der kernel-funktion: perioden, wann reel, ..., was passiert bei abtastung?
        \item implizite periodifizierung des signals?
        \item zusammenhang zur physik
        \item dc, nyquist-frequenzen, wann vorhanden?
        \item reelles signal: dc und nyquist reell
    \end{itemize}
    \item zusammenhang zwischen ft, dtft und dft (intuition mit periodifizierung vs. abtastung; non-intuition durch kernel angucken)
    \item dft als approximation der ft
    \begin{itemize}
        \item $s(t) = \exp(-t/\tau) \cdot \sin(\omega_0 \cdot t)$, 
        \item $s(t) = \exp(-(t - \mu)^2/\sigma^2)$ (Uebung)
    \end{itemize}
    \item spectral leakage;
\end{itemize}
\subsection{Anwendung der Transformationen}
\begin{itemize}
    \item fft (Uebung 1 jup, uebung 2 8.3.2)
    \item bild-kompression (DCT, JPEG)
    \item fensterung \footnote{\url{https://docs.scipy.org/doc/scipy/reference/signal.windows.html}}
    \item schnelle faltung (cyclic/non-cyclic)
    \item beliebig lange signale overlap add, overlap save (\"Ubung: irgendwas mit audio prosessing)
    \item Unschaerfe Relation (Fensterbreite vs. Frequenzaufloesung)\cite[chpt. 4.2]{mallat2008wavelets}
\end{itemize}
