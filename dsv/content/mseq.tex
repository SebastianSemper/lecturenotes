Wir wollen noch eine weitere wichtige Anwendung für die in dieser Vorlesung entwickelten Signalverarbeitungstechniken betrachten.
Für die Konzipierung eines RADAR-Systems macht man sich zu Nutze, dass sich ausbreitende elektromagnetische Wellen an Objekten, auf die sie währen der Ausbreitung treffen, reflektiert werden.
Mit einer Antenne kann man beispielsweise ein Signal $x[\cdot]$ mit einem \gls{tx} aussenden und mit einer zweiten Antenne am \gls{rx} auf das eventuelle Echo warten.
Wird ein Objekt mit Entfernung $d$ nun von dieser Welle getroffen und reflektiert einen Teil der Energie in Richtung \gls{rx}, so enthält das empfangene Signal $y[\cdot]$ eine zeitlich verzögerte Kopie von $x[\cdot]$ und eine gewisse Menge an Rauschen, das von den Schaltkreisen an \gls{tx} und \gls{rx} eingeprägt wird.
Es gilt also
\[
y[\cdot] = \gamma x[\cdot - \tau] + w[\cdot],
\]
wobei sich $\tau$ aus der Samplerate $F_s$, dem Abstand $d$ und der Lichtgeschwindigkeit $c$ via
\[
\tau = F_s \cdot \frac{2d}{c} 
\]
ergibt.
Die Zahl $\gamma \in \R$ repräsentiert einerseits Verluste, die sich durch die Ausbreitung im Raum ergeben und gleichzeitig aus der \q{Reflektivität} des Objektes.
Für gewöhnlich hat ein RADAR zwei Aufgaben. 
Einerseits soll detektiert werden, \emph{ob} ein Objekt von unserem RADAR angeleuchtet wurde und im Idealfall, wie groß $d$ ist.
Wir wollen uns also Signalverarbeitungstechniken ansehen, die die Bestimmung von $d$ erlauben.
%
\subsubsection{Zyklische Korrelationen}
%
Eine Möglichkeit, um $d$ zu erhalten Basiert auf Korrelation.
Hierzu schränken wir zunächst die Menge der Sendesequenzen ein, indem wir fordern, dass $x[\cdot]$ periodisch mit Periodendauer $N$ ist.
Die Korrelation $r_{x,y}$ von zwei periodischen Sequenzen $x[\cdot]$ und $y[\cdot]$ ist gegeben durch
\begin{equation}\label{eq:mseq:corr}
    r_{x,y}[\ell] = \frac{1}{N}\Sum{n = 0}{N}{
        x[n]y[n - \ell]
    }.
\end{equation}
Stellen wir uns vor, dass unser RADAR-System die Sendesequenz $x[\cdot]$ erzeugt, indem es eine endliche Sequenz $\bm x \R^{N}$ wiederholt aussendet, so ist $x[\cdot]$ periodisch mit Periodendauer $N$ und auch $y[\cdot]$ -- nach den ersten $N$ Samples.
Die Frage ist nun, was uns diese Korrelation nützt.
Man kann sich überlegen, dass
\[
\Abs{r_{x,y}[\ell]} 
    \leqslant \sqrt{
        r_{x,x}[0] \cdot r_{y,y}[0] 
    }
    = \sqrt{
        \mathcal{E}(x) \mathcal{E}(y)
    },
\]
siehe \Cref{eq:disc_sig_energy}.
Das heißt, dass die sog.\,\emph{Kreuzkorrelation} $r_{x,y}[\cdot]$ immer durch die Werte der sog.\,\emph{Autokorrelation} $r_{x,x}[\cdot]$ wie oben angegeben beschränkt ist.
Im Spezialfall $y=x$ ergibt sich
\[
\Abs{r_{x,x}[\ell]} \leqslant r_{x,x}[0],
\]
was heißt, dass die \gls{acf} $r_{x,x}[\cdot]$ ihr Maximum immer am Wert $\ell=0$ annimmt.

Betrachten wir nun wieder unser RADAR-System, so könnte es sich als sinnvoll erweisen, am Empfänger das Signal $y[\cdot]$ mit dem Sendesignal zu korrelieren, also
\[
r_{y,x}[\ell] 
    = \gamma r_{x,x[\cdot-\tau]}[\ell] + r_{w,x}[\ell]
    = \gamma r_{x,x}[\ell-\tau] + r_{w,x}[\ell]
\]
zu berechnen.
Da $r_{x,x}[\cdot]$ bei $0$ ein lokales Maximum hat, so hat $r_{x,x[\cdot-\tau]}[\cdot]$ ein lokales Maximum bei $\tau$!
Wir könnten also einfach den maximalen Wert in $r_{y,x}[\cdot]$ suchen und dessen Stelle in $d$ umrechnen.

Die Berechnung von $r_{y,x}[\cdot]$ für $N$ Werte erfordert in der Größenrdnung $N^2$ viele \glspl{flop}.
Man kann aber ziemlich einfach einsehen, dass man die Korrelation als Faltung via
\begin{equation}\label{eq:mseq:corr_conv}
    r_{x,y} = x[\cdot] \circledast y[-\cdot]
\end{equation}
ausdrücken kann.
Außerdem gilt für eine Sequenz $x[\cdot]$ und deren \gls{dft} $X[\cdot]$, dass
\begin{equation}\label{eq:mseq:dft_fold}
    \texttt{DFT}(x[-\cdot]) = X[\cdot]^\ast,
\end{equation}
also Umkehrung im Zeitbereich (Reflexion an $n=0$) entspricht Konjugation im Frequenzbereich.
Demzufolge können wir \Cref{par:fourier:cycl_conv} zusammen mit \eqref{eq:mseq:corr_conv} verwenden, um $r_{x,y}[\cdot]$ mit $N \log(N)$ vielen \glspl{flop} zu berechnen.
Das heißt, dass für die Signalverarbeitung effiziente Methoden bereitstehen, solange der Ansatz der Berechnung der Kreuzkorrelation des Empfangssignals mit der Sendesequenz als theoretisch fundiert herausstellt.
%
\begin{listing}[ht]
    \noindent
    \begin{minipage}{0.51\textwidth}
        \strut\vspace*{-\baselineskip}\newline
        \inputminted[firstline=6, lastline=29]{python3}{code/radar1.py}
    \end{minipage}%
    \begin{minipage}{0.48\textwidth}
        \strut\vspace*{-\baselineskip}\newline
        \includegraphics[width=\textwidth]{code/radar_1_1.png}

        \includegraphics[width=\textwidth]{code/radar_1_2.png}

        \includegraphics[width=\textwidth]{code/radar_1_3.png}
    \end{minipage}
    \codecaption{dsv/code/radar1.py}{Vergleich verschiedener Sendesignale $x[\cdot]$ bezüglich der resultierenden Korrelation $r_{x,y}[\cdot]$}\label{py:radar1}
\end{listing}

In \Cref{py:radar1} sind einige verschiedene Sendesignale miteinander verglichen, indem zwei Reflexionen mit verschiedenen Abständen simuliert werden und aus dem resultierenden Empfangssignal $y[\cdot]$ wiederum $r_{x,y}[\cdot]$ bestimmt wird.
Wie man sieht, hat das Sendesignal einen signifikanten Einfluss auf die \q{Form} der \gls{acf}, welche wiederum einen Einfluss auf die visuelle Qualität der Funktion $r_{x,y}[\cdot]$ hat, wenn man an die Anwendung im RADAR denkt.
Nutzt man nur einen einfachen Sinus als Sendesignal (\Cref{py:radar1}, oben), so kann man gar keine Information über den Abstand von eventuellen Reflexionen finden. 
Der Grund ist, dass sich Zeitversatz nur mit Signalen schätzen lässt, die eine gewisse Bandbreite abdecken, also mehrere Frequenzen gleichzeitig belegen.

Mit einer $\Sinc$-Funktion (\Cref{py:radar1} mitte) belegen wir bekanntermaßen im Frequenzbereich einen kontinuierlichen Bereich, was dazu führt, dass wir in der Tat sehr ausgeprägte lokale Maximum bei den wahren $\tau$ erhalten.
Je schmaler die Hauptkeule der $\Sinc$-Funktion im Zeitbereich, desto breiter ist der abgedeckte Frequenzbereich, was leicht durch Modifikation von \Cref{py:radar1} überprüft werden kann.
Dies ist von Vorteil, wenn man mehrere Reflexionen, die in etwa gleichen Abständen auftreten doch noch von einander unterscheiden möchte.
Es ist jedoch zu bemerken, dass ein im Zeitbereich \q{schmaler} $\Sinc$ an die Hardware, die ihn erzeugt einige unangenehme Herausforderungen stellt, weshalb solch ein Sendesignal in der Praxis einige Nachteile aufweist.

Schlussendlich sehen wir (\Cref{py:radar1}, unten), dass zufällige Sendesignale auch gute Eigenschaften bezüglich $r_{x,y}[\cdot]$ liefern.
Gleichzeitig haben diese den Vorteil, dass keine großen Sprünge in der Amplitude notwendig sind, da sich das Signal bei maximaler Amplitude nicht sehr weit von seinem Mittelwert entfernt\linkfootnote{https://en.wikipedia.org/wiki/Crest_factor}.
Deren Nachteil ist jedoch, dass man kein System auf gänzlich zufälligem Verhalten fußen lassen sollte.

Speziell für unsere Anwendung im RADAR streben wir also an, dass die \gls{acf} $r_{x,y}[\cdot]$ ein ausgepträgtes lokales Maximum hat, also einerseits sehr schnell abfällt, und andererseits auch wenig bis keine Nebenmaxima erzeugt, weil sich dies beispielsweise in \Cref{py:stft_harm} als Problem herausgestellt hatte.
Im Idealfall finden wir eine Sendesequenz, die sich einerseits gut für RADAR eignet und gleichzeitig sich beispielsweise effizient in Hardware erzeugen lässt.

\FloatBarrier
\subsubsection{Erzeugung der Sendesequenz}
%
\begin{figure}
    \begin{center}
        \includegraphics[width=0.6\textwidth]{img/lfsr_1.png}
    \end{center}
    \caption{Lineaeres Feedback Shift Register, Quelle \cite{proakis2013}}\label{fig:mseq:lfsr}
\end{figure}
%
Wir wollen im Folgenden eine mögliche Art der Sequenzerzeugung genauer untersuchen, da sie beide obigen Anforderungen erfüllt.
Hierzu betrachten wir sogenannte \glspl{lfsr}, wie in \Cref{fig:mseq:lfsr} dargestellt.
Diese bestehen aus $L$ Registern/Taps, welche jeweil ein Bit beinhalten.
Die Idee ist nun dieses \gls{lfsr} als diskretes System zu betrachten, dessen Werte in der Menge $\Z_2^L = \{0,1\}^L$ liegen können und diese Menge $\Z_2$ mit der binären Addition und Multiplikation ausgestattet wurde.
In jedem Takt werden die Werte von Links nach Rechts verschoben, wobei der Zustand des letzten Registers als Ausgabe fungiert.
Weiterhin werden einige der Zustände durch binäre Addition miteinander verknüpft und an den Eingang, also das erste Register, als Feedback zurückgeführt.
Es ist demnach auch leicht zu sehen, dass solch ein System sehr leicht und effizient in diskreter Hardware implementiert werden kann.

Zum Zeitpunkt $n$ ist damit die Ausgabe
\[
x[n] 
    = x[n-1] 
      + a_1 x[n-1] 
      + \ldots 
      + a_{L-1} x[n - (L-1)] 
      + x[n-L],
\]
wobei die Koeffizienten $a_1, \ldots, a_{L-1} \in \Z_2$, diese also definieren, welche Werte der Register für die Berechnung des Feedbacks genutzt werden.
Es ist anzumerken, dass man nun für einen Start der Simulation die Zustände $x[-L], \ldots, x[-1]$ kennen muss.
Man kann das System zum Zeitpunkt $n$ durch die Werte $x[n], \ldots, x[n-L]$ also eine Binärzahl mit $L$ Stellen/Bits beschreiben. 
Kennt man all diese Zustände, ist es auch möglich den Nächsten zu berechnen, indem auf die Zahl ein Bitshift und eine \q{maskierte} Addition über die Bits des aktuellen Zustands angewandt wird.

Man sieht nun, dass sich mit $L$ binären Registern $2^L$ Zustände abbilden lassen, und dass das System maximal $2^L-1$ von diesen durchlaufen kann, wenn man nicht im Zustand $0\ldots 0$ beginnt.
Die Werte $y[n]$ müssen demnach periodisch sein, denn sobald sich ein einzelner der $2^L-1$ möglichen Zustände ein zweites Mal einstellt, folgen ihm wieder diesselben wie beim ersten Auftreten.
Demnach ist auch die maximale Periode mit $2^L-1$ beschränkt.
Die Frage ist nun, ob es Koeffizienten-Konfigurationen gibt, für welche die maximale Periodenlänge erreicht wird.

Hierzu betrachten wir die $z$-Transformation des Systems, dann finden wir
\[
H(z) = 1 + a_1 z^{1} + a_2 z^2 + \ldots a_{L-1} z^{L-1} + z^L,
\]
woran wir wiederum sehen, dass das System nur durch die Wahl der Koeffizienten $a_1, \ldots, a_{L-1} \in \Z_2$ bestimmt wird.
Man kann zeigen, dass falls die Transferfunktion $H$ ein sogenanntes irreduzibles Polynom
\footnote{
    Siehe, \refurl{https://de.wikipedia.org/wiki/Irreduzibles_Polynom}. 
    Irreduzible Polynome treten im Zusammenhang mit endlichen Körpern auf, beispielsweise $\Z_2$, wo eine andere Art Polynomdivision angewandt werden kann/muss. 
    Irreduzible Polynome nehmen im Grunde hier die Rolle von Primzahlen ein, da sich jedes Polynom in ein Produkt aus irreduziblen Polynomen zerlegen lässt.
} ist, so hat die von diesem \gls{lfsr} erzeugte Sequenz maximale Länge, also $2^L-1$.
Das heißt, dass die algebraischen Eigenschaften der $z$-Transformierten der Transferfunktion $H$ wieder Aufschluss darüber geben, welche Eigenschaften das \gls{lfsr} als System betrachtet besitzt.
In \Cref{py:mlbs} wird ausgehend von einem Startwert des Registers (\texttt{0x1}) und einer sogenannten Tapkonfiguration (\texttt{0xD}) eine \gls{mlbs} durch einmaliges Simulieren des \gls{lfsr} erzeugt.
Die entstehende Folge hat ähnliche Eigenschaften, wie eine \q{wirklich} zufällige Zahlenfolge, doch entsteht aus einem vollständig deterministischem Prozess.
Man nennt solche Sequenzen \q{pseudo zufällig}.
%
\begin{listing}[ht]
    \noindent
    \begin{minipage}{0.51\textwidth}
        \strut\vspace*{-\baselineskip}\newline
        \inputminted[firstline=5, lastline=32]{python3}{code/mlbs.py}
    \end{minipage}%
    \begin{minipage}{0.48\textwidth}
        \strut\vspace*{-\baselineskip}\newline
        \includegraphics[width=\textwidth]{code/mlbs.png}
    \end{minipage}
    \codecaption{dsv/code/mlbs.py}{Simulation eines \acrshort*{lfsr} mittels binären Operationen}\label{py:mlbs}
\end{listing}
%
\FloatBarrier
%
\subsubsection{\texorpdfstring{\acrshort*{mlbs}}{MLBS} als Sendesequenz}
Zum Abschluss wollen wir nun noch untersuchen, wie gut sich die so gewonnenen Sequenzen für RADAR verwenden lassen.
Hierzu betrachten wir die Ausgabe von \Cref{py:radar2}, wo wir das Wissen aus \Cref{py:radar1} und \Cref{py:mlbs} kombiniert haben, um einerseits \q{effizient} eine
Sendesequenz zu erzeugen und andererseits, um die RADAR-Signalverarbeitung effizient zu gestalten.
%
\begin{listing}[ht]
    \noindent
    \begin{minipage}{0.51\textwidth}
        \strut\vspace*{-\baselineskip}\newline
        \inputminted[firstline=3, lastline=21]{python3}{code/radar2.py}
    \end{minipage}%
    \begin{minipage}{0.48\textwidth}
        \strut\vspace*{-\baselineskip}\newline
        \includegraphics[width=\textwidth]{code/radar2.png}
    \end{minipage}
    \codecaption{dsv/code/radar2.py}{Simulation eines RADAR-Systems basierend auf \acrshort*{mlbs}, die von \acrshort*{lfsr} erzeugt wurden. Oben verrauschtes Empfangssignal $y[\cdot]$, mitte \acrshort*{dft} $Y[\cdot]$, unten $r_{x,y}$}\label{py:radar2}
\end{listing}

Wir wir sehen, erzeugt das \gls{lfsr} eine Sequenz, welche sehr gut zu unseren Anforderungen passt. 
Einerseits ist die Aussteuerund des Signals sehr begrenzt und andererseits hat die \gls{acf} ideale Eigenschaften, um Abstände von Reflexionen zu bestimmen.
Man kann sogar beweisen, dass für eine Sequenz maximaler Länge $x[\cdot]$ gilt, dass
\begin{equation}
    r_{x,x}[\ell] = \begin{cases}
        1 \Text{für} \ell = 0, \\
        -\frac{1}{2^L -1} \Text{sonst.}
    \end{cases}
\end{equation}
Damit ist das lokale Maximum immer um den Faktor $N = 2^L-1$ höher als die Werte im Rest der Korrelationsfunktion $r_{x,x}$. 
Die bedeutet einerseits, dass dieses Verfahren auch mit eventuellem Messrauschen sehr gut umgehen kann.
Andererseits kann man auch durch den Einsatz von sehr langen Sequenzen, also mehr Registern im \gls{lfsr}, mit sehr wenig Sendeleistung durch die Transformation von $y[\cdot]$ nach $r_{x,y}[\cdot]$ immernoch Reflexionen detektieren.
