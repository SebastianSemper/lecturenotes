\subsection{Parameterschätzung}\label{sec:random:paramest}
%
%
Wir wollen nun unsere Sichtweise umdrehen.
Bisher sind wir implizit davon ausgegangen, dass die Parameter unserer Verteilungen, also beispielsweise $\mu$ und $\sigma^2$ bei $\mathcal{N}(\mu,\sigma^2)$, bekannt sind.
Wir also im Grunde während der Modellierung annehmen, dass $X \sim \mathcal{N}(\mu, \sigma^2)$, oder $X_n \sim \mathcal{N}(\mu_n, \sigma_n^2)$.
In der Praxis ist es jedoch oft so, dass man mit den Werten $X_n$ konfrontiert ist und sich dann im Nachgang