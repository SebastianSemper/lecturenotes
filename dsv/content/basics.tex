\begin{itemize}
    \item komplexe zahlen
        \begin{itemize}
            \item darstellungen
            \item multiplikation
            \item beispiel: mandelbrot-menge (\"Ubung)
        \end{itemize}
    \item signale
        \begin{itemize}
            \item kontinuierlich
            \item zeit-diskret, wert-diskret
            \item 1D (zeit), 2D/3D (raum)
            \item beispiele (\"Ubung): phasor als harmonische
        \end{itemize}
    \item signale als vektoren: 
        \begin{itemize}
            \item vektor-rechnung
            \item basen: bausteine von signalen
            \item lineare abbildungen: "passen zu vektoren" manipulation von signalen, \gls{ft}, \gls{dft}, differentiation, gegenbeispiele (exp, sin)
            \item basis-wechsel: verschiedene darstellung des gleichen signals (analyse, synthese)
            \item eigenwerte/vektoren: richtungen entlang derer die lineare abbildung "einfach" ist, i.e. skaliert
            \item geometrie von signalen: skalarprodukt, norm
            \item \"Uebung: dot, inner, norm, dft, eigs (random)
        \end{itemize}
    \item lineare systeme: 
        \begin{itemize}
            \item beispiel fuer lineare abbildungen
            \item zeitinvariant
            \item eigenfunktionen
            \item beispiele: verzoegerung, skalierung, kopierung (\"Ubung)
        \end{itemize}
\end{itemize}