\begin{itemize}
    \item system das signale in mehreren raten gleichzeitig benutzt
    \item erste option: digital zu analog, filtenr, analog zu digital
    \item vorteil: sampling rate kann beliebig (je nach signal bandbreite) veraendert werden
    \item nachteil: distortion von da, ad conversion introduce noise, quantization errors
    \item also: conversion direkt in digitaler domaene
    \item problem: how? wir haben das signal nur auf diskreten werten gegeben
    \item wir koennen aber mal die formel fuer perfekte rekonstruktion aufschreiben, die das richtige tut unter der bedingung, dass sampling rate nyquist einhaelt
    \item dann koennen wir diese interpolation einfach auf den neuen samplingpunkten auswerten
    \item resampling ist immerhin schonmal ein lineares system
    \item falls neue samplingrate kleiner, muessen wir auch hier noch nyquist einhalten, oder eben die formel selbst lowpass-filtern
    \item falls alte samplingrate = neue samplingrate haben wir ein LTI-system (uebung)
    \item wenn die samplingrate beliebgi sind, muss man genau hinschauen.
    \item erstmal bild malen
    \item dann gleichungen rumwursten
    \item lineares zeitvariantes system, also impulsantwort ist zeitabhaengig
    \item dann nochmal bild anschauen
    \item fuer rationales verhaeltnis: intuitiv kommt man je nach evrhaltnis wieder auf die gleichen fractional parts, also ist das resampling ein periodisch zeitvariantes lineares system 
    \item downsampling um faktor $D$ (in uebung programmieren, einmal mit weglassen und einmal als faltung)
    \item upsampling um faktor $I$ liefert $I$ verschiedene Impulsantworten des systems
    \item beispiel mit $I = 2$ in der vorlesung (Figure 11.1.4.!), beispiel mit $I = 3$ in uebung. einmal ohne interleaving als periodisch zeitvariantes systen, einmal mit interleaving
\end{itemize}