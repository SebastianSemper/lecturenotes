% !TEX root = ../dsv_script.tex

\subsubsection{Motivation}
\begin{itemize}
    \item Fuer Kommunikation ist das Ausbreitungsverhalten von EM zwischen den beteiligten geraeten wichtig
    \item abschaetzung von link budget, datenraten, simulation in software, verifizierung von geraeten
    \item channel sounding vermisst solche kanaele \cite{thomae2005multidim_hrpe}
    \item hohe bandbreiten, antennen arrays, hohe aufloesung
    \item extraktion der einzelnen ausbreitungswege zwischen sender und empfaenger
    \item da eine antenne bei uebertragung selbst ein \gls{lti} system ist, praegt sie dem signal ihre impulsantwort auf
    \item dieser einfluss soll aber aus den messungen entfernt werden
    \item deshalb muessen die benutzten antennen vorher korrekt charakterisiert werden
    \item andere anwendung ist die simulation, wo bereits informationen \"uber die ausbreitungsrichtungen vorliegen und jetzt realistische Messdaten simuliert werden sollen
    \item In any case: Wir muessen wissen wie das richtungs, frequenz und polarisations-abh. verhalten der antenne ist
    \item wir gehen in die antennenmesskammer und regen die antenne an (winkel, pol, freq)
    \item koennen nicht "alle" winkel messen
    \item antennen sind im winkel bandlimitiert \cite{delgaldo2007phd}
    \item koennen nyquist-sampling theorie anwenden
    \item muessen nicht alle winkel messen
\end{itemize}
\subsubsection{Messvorgang}
\begin{itemize}
    \item messung in echo-freier messkammer (wir nehmen das an, ist aber nicht so)
    \item anregung mit bekannter, direktiver und polarisierter referenz-antenne
    \item zu kalibrierende \gls{aut} wird an roboter geschnallt, der sie beliebig bezueglich referenz-antenne ausrichten kann
    \item messung des freqeunzganges der \gls{aut} fuer endliche menge an winkeln
    \item bereinigung der messdaten von einfluss des messsystems unter der annahme von \gls{lti} system
\end{itemize}
%
%
\subsubsection{Fourier-Interpolation}
%
%
Gegeben sei ein periodisches, analoges Signal $x: \R \rightarrow \C$, mit periode $T_p = 1/F_0$. Dieses Signal kann man in seine Fourier-Reihe via
\begin{equation}\label{fourier_series}
    x(t) = \Sum{k=-\infty}{+\infty}{
        c[k] \exp(\jmath 2 \pi k t F_0) 
    }
\end{equation}
entwickeln. Nun tasten wir dieses Signal uniform mit Samplerate $F_s = N/T_p = 1/T$ (also passend zur Periodendauer) ab und erhalten die Folge 
\begin{equation}
    x[n] = x(n T) = \Sum{k=-\infty}{+\infty}{
        c[k] \exp(\jmath 2 \pi k n T F_0) 
    } = \Sum{k=-\infty}{+\infty}{
        c[k] \exp\left(\jmath 2 \pi k \frac nN\right) 
    }
    \Text{f\"ur}
    n \in \N
\end{equation}
bestehend aus den Samples von $x$. Mit der Periodizit\"at von $\exp(\jmath 2 \pi t)$ und der Abtastung erhalten wir au{\ss}erdem noch
\begin{equation}
    x[n] = \Sum{k=0}{N-1}{
        \left[\Sum{\ell=-\infty}{+\infty}{
            c[k - \ell N]
        }\right] \exp\left(\jmath 2 \pi k \frac nN\right) 
    } = \Sum{k=0}{N-1}{
        \tilde{c}[k] \exp\left(\jmath 2 \pi k \frac nN\right) 
    },
\end{equation}
wobei wir 
\begin{equation}\label{aliased_ck}
    \tilde{c}[k] = \Sum{\ell=-\infty}{+\infty}{
        c[k - \ell N]
    }
\end{equation}
als Abk\"urzung benutzt haben. Ist nun die Funktion $x$ auch bandbegrenzt, d.h.~ihre Fourier-Transformierte $X : \R \rightarrow \C$ verschwindet au{\ss}erhalb eines gewissen Bandes, also
\begin{equation}
    X(F) = \Int{-\infty}{+\infty}{x(t) \exp(-\jmath 2 \pi F t )}{t} = 0 \Text{f\"ur} \Abs{F} > B,
\end{equation}
dann wissen wir, dass die Fourier-Transformation $X$ und die Folge $c[k]$ verkn\"upft sind via 
\begin{equation}
    c[k] = \frac{1}{T_p} X(k F_0),
\end{equation}
was impliziert, dass die Folge $c[k]$ verschwindet, also gilt
\begin{equation}
    c[k] = 0 \Text{f\"ur} \Abs{k} > \frac{B}{F_0}.
\end{equation}
Das hei{\ss}t, dass wir nun $F_s = N/T_p$ so gro{\ss} w\"ahlen m\"ussen, dass sich in \eqref{aliased_ck} kein Aliasing ergeben darf, also muss gelten
\begin{equation}
    N > \lceil B/F_0 \rceil \Text{bzw.} F_s > \lceil B/(F_0 T_p) \rceil.
\end{equation}
In diesem Falle gilt, dann dass $c[k] = X[k]$, wobei $X[k]$ die \gls{dft} der Folge $x[n]$ darstellt. Das hei{\ss}, dass wir die Fourier-Koeffizienten der kontinuierlichen Funktion $x$ durch die \gls{dft} der Abtastwerte $x[n]$ bestimmen k\"onnen. Mit \eqref{fourier_series} k\"onnen wir also die Folge $x[n]$ interpolieren, indem wir
\begin{equation}\label{dft_interpolation}
    x(t) = \Sum{k=-\frac{B}{F_0}}{+\frac{B}{F_0}}{
        X[k] \exp(\jmath 2 \pi k t F_0) 
    }
\end{equation}
schreiben. Man beachte hier, dass nun aus der Folge von diskreten Werten $x[n]$ eine analytische Formel in Form einer \emph{endlichen} Summation entstanden ist. Unter der Annahme der Bandlimitierung von $x$ ist diese Interpolation \emph{exakt} und kann effizient implementiert werden, durch die Vorberechnung der Folge $X[k]$ durch die \gls{fft}~\cite{FFTW05} der Folge $x[n]$. 
%
%
%
%
\subsubsection{Ableitung der EADF}
%
%
Wir wollen nun \eqref{dft_interpolation} aus zwei Dimensionen erweitern und folgen damit effektiv~\cite{landmann2004EADF}.
Aus{\ss}erdem \"andern wir das Argument der Funktion $x$ und deren Namen zu der \"ublicheren Schreibweise $a : [0, 2 \pi] \times [0, 2 \pi] \rightarrow \C$ mit Werten $a(\varphi, \vartheta)$. 
Wir nehmen nun hier an, dass $a$ in \emph{beiden} Argumenten $2\pi$-periodisch ist und schneiden somit die Funktion bereits auf unsere Anwendung zu.

Die Bandlimitierung von $a$ l\"asst sich nun so formulieren, dass die Bedingungen
\begin{align}
    A_{\varphi}(F_\varphi, \vartheta) 
    &= \Int{-\infty}{+\infty}{
        a(\varphi, \vartheta) \exp(\jmath 2 \pi F_\varphi \varphi)
    }{\varphi} = 0 
    \fuer F_\varphi > B_\varphi \Text{und alle} \vartheta \in [0, 2 \pi] \Text{und} \\
    A_{\vartheta}(\varphi, F_\vartheta) 
    &= \Int{-\infty}{+\infty}{
        a(\varphi, \vartheta) \exp(\jmath 2 \pi F_\vartheta \vartheta)
    }{\varphi} = 0 
    \fuer F_\vartheta > B_\vartheta \Text{und alle} \varphi \in [0, 2 \pi]
\end{align}
erf\"ullt sein m\"ussen. Nun lassen sich alle obigen Argumente ``schnittweise'' auf eine abgetastete Version von $a$ in der Form $a[n_\varphi, n_\vartheta]$ anwenden. Das hei{\ss}t, wir landen schlussendlich bei einer Interpolations-Formel
\begin{equation}\label{2d_dft_interpolation}
    a(\varphi, \vartheta) = \Sum{k_\varphi=-\frac{B_\varphi}{F_\varphi}}{+\frac{B_\varphi}{F_\varphi}}{
        \Sum{k_\vartheta=-\frac{B_\vartheta}{F_\vartheta}}{+\frac{B_\vartheta}{F_\vartheta}}{
            A[k_\varphi, k_\vartheta] 
            \cdot \exp(\jmath 2 \pi k_\varphi \varphi F_\varphi)
            \cdot \exp(\jmath 2 \pi k_\vartheta \vartheta F_\vartheta)
        }
    },
\end{equation}
welche absolut analoge $2D$-Version zu \eqref{dft_interpolation} darstellt. Auch in diesem Fall, k\"onnen wir das $2D$-Array $A[k_\varphi, k_\vartheta]$ durch eine $2D$-\gls{dft}, bzw. der \gls{fft}, von uniformen Samples der Funktion $a$ effizient vorberechnen.

Um einen m\"oglichst effizienten Algorithmus f\"ur die Auswertung der Interpolante zu erhalten, sollte man \eqref{2d_dft_interpolation} geeignet umschreiben. Moderne Rechenarchitekturen und Scientific-Computing-Libraries sind auf schnelle Matrix-Vektor-Produkte optimiert. Nehmen wir an, wir wollen \eqref{2d_dft_interpolation} f\"ur mehrere Winkelpaare $(\varphi_1, \vartheta_1), \dots, (\varphi_L, \vartheta_L)$ auswerten. Dann berechnen wir zun\"achst zwei $2D$ arrays
\begin{align}
    D_\varphi &= [\exp(\jmath 2 \pi k_\varphi \varphi F_\varphi)]_{\ell = 1, k_\varphi=-\frac{B_\varphi}{F_\varphi}} \in \C^{L \times 2 \frac{B_\varphi}{F_\varphi} + 1} \Text{und}\\
    D_\vartheta &= [\exp(\jmath 2 \pi k_\vartheta \vartheta F_\vartheta)]_{\ell = 1, k_\vartheta=-\frac{B_\vartheta}{F_\vartheta}} \in \C^{L \times 2 \frac{B_\vartheta}{F_\vartheta} + 1}, 
\end{align}
was uns erlaubt \eqref{2d_dft_interpolation} das folgende Vektor-Matrix-Vektor-Produkt
\begin{equation}\label{fast_2d_dft}
    a(\varphi_\ell, \vartheta_\ell) = 
        D_\varphi[\ell, :] \cdot A \cdot D_\vartheta[\ell, :]^\trans
\end{equation}
umzuschreiben. Damit besteht der Interpolations-Algorithmus zun\"achst aus der Vorberechnung des Arrays $A$, sowie bei Ausf\"uhrung dann aus der Berechnung von $D_\varphi$ und $D_\vartheta$, sowie der Auswertung von \eqref{fast_2d_dft}.

Man sieht hier der sch\"on, dass die Laufzeitkomplexit\"at von \eqref{fast_2d_dft} ma{\ss}geblich von der r\"aumlichen Bandbegrenzung der Antennen-Richtcharakteristik beeinflusst wird. Je h\"oher die Bandbreite, desto h\"oher ist nicht nur der Aufwand bei der Messung, sondern auch bei der Interpolation.