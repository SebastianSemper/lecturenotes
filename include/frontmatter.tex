\pagenumbering{roman}
\glsunsetall
%
\maketitle
\tableofcontents
\clearpage
%
\listoffigures
\addcontentsline{toc}{section}{Abbildungsverzeichnis}
\listoflistings
\addcontentsline{toc}{section}{Code-Schnipsel-Verzeichnis}
\clearpage
%
\pretocmd{\section}{\FloatBarrier\clearpage}{}{}
\pretocmd{\subsection}{\FloatBarrier}{}{}
%
\glsresetall
\setcounter{page}{1}
\pagenumbering{arabic}

\begin{center}
    \begin{minipage}{0.6\textwidth}
        Die Idee hinter dem Skript zur Vorlesung ist, dass es die Zuh\"orer der B\"urde des Mitschreibens entledigt und Zeit und Platz zum Folgen der Vorlesung frei macht.
        Das Skript sollte deshalb immer zur Vorlesung und \"Ubung mitgebracht und im Idealfall mit Notizen versehen werden, bzw. zum Nachschlagen verwendet werden.
    \end{minipage}
\end{center}