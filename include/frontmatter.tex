\pagenumbering{roman}
\glsunsetall
%
\maketitle
\tableofcontents
\clearpage
%
\listoffigures
\addcontentsline{toc}{section}{Abbildungsverzeichnis}
\listoflistings
\addcontentsline{toc}{section}{\code{verzeichnis}}
\clearpage
%
\pretocmd{\section}{\FloatBarrier\clearpage}{}{}
\pretocmd{\subsection}{\FloatBarrier}{}{}
%
%
\begin{center}
    \begin{minipage}{0.75\textwidth}
        \begin{center}
            \textbf{\large Allgemeine Hinweise}
        \end{center}
        Die Idee hinter dem Skript zur Vorlesung ist, dass es die Zuh"orer der B"urde des Mitschreibens entledigt und Zeit und Platz zum Folgen der Vorlesung frei macht.
        Das Skript sollte deshalb immer zur Vorlesung und "Ubung mitgebracht und im Idealfall mit Notizen versehen werden, bzw. zum Nachschlagen verwendet werden.

        Eine stetig aktualisierte Version des Skriptes findet man unter \url{https://github.com/SebastianSemper/lecturenotes}.

        Als Begleitmaterial ist auch eine Auswahl an Codeschnipseln bereitgestellt, die einerseits in Ausz"ugen im Skript direkt eingebunden sind, aber auch unter dem obigen Link aufzufinden sind.

        Zum Ausf"uhren der Codeschnipsel empfehlen wir ein Python-Environment\footnote{\url{https://github.com/conda-forge/miniforge}}, in welchem folgende Pakete installiert sein sollten:
        \begin{itemize}
            \item \texttt{numpy} -- \url{https://numpy.org}
            \item \texttt{scipy} -- \url{https://scipy.org}
            \item \texttt{matplotlib} -- \url{https://matplotlib.org}
            \item \texttt{cupy} -- \url{https://cupy.dev} (optional f"ur ZOOMG GPU speed)
            \item \texttt{sympy} -- \url{https://sympy.org} (optional f"ur symbolisches Rechnen in Python)
        \end{itemize}
        Diese k"onnen ganz einfach via
        \begin{minted}{bash}
        conda create -n dsv python=3.10
        conda activate dsv
        python -m pip install numpy scipy matplotlib
        \end{minted}
        installiert werden.

        Alternativ steht unter \url{https://jup.rz.tu-ilmenau.de/hub/login} ein Jupyter-Hub zur Verf"ugung, der eine Python IDE im Browser bereitstellt.
    \end{minipage}
\end{center}
%
\clearpage
%
\glsresetall
\setcounter{page}{1}
\pagenumbering{arabic}