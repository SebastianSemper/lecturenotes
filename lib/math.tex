\usepackage{amsmath}
\usepackage{amsfonts, amssymb, yhmath}
\usepackage{amsthm}
\usepackage{thmtools}
\usepackage{bm}

%% end packages
%% 

\declaretheorem[
	numberwithin=section,
  title=Lemma,
  refname={lemma,lemmas},
  Refname={Lemma,Lemmas}
]{Lem}

\declaretheorem[
	numberwithin=section,
  title=Theorem,
  refname={theorem,theorems},
  Refname={Theorem,Theorems}
]{Thm}

\declaretheorem[
	numberwithin=section,
  title=Proposition,
  refname={proposition,propositions},
  Refname={Proposition,Propositions}
]{Pro}

\declaretheorem[
	numberwithin=section,
  title=Corollary,
  refname={corollary,corollaries},
  Refname={Corollary,Corollaries}
]{Cor}

\declaretheorem[
	numberwithin=section,
  title=Example,
  refname={example,examples},
  Refname={Example,Examples}
]{Exm}

\declaretheorem[
	numberwithin=section,
  title=Remark,
  refname={remark,remarks},
  Refname={Remark,Remarks}
]{Rem}

\declaretheorem[
	numberwithin=section,
  title=Algorithm,
  refname={algorithm,algorithms},
  Refname={Algorithm,Algorithms}
]{Alg}

\declaretheorem[
	numberwithin=section,
  title=Definition,
  refname={definition,definitions},
  Refname={Definition,Definitions}
]{Def}

\declaretheorem[
	numberwithin=section,
  title=Discussion,
  refname={discussion,discussions},
  Refname={Discussion,Discussions}
]{Dis}

\newcommand{\Ind}[1]{{\boldsymbol{\mathrm{#1}}}}

\newcommand{\trans}{\mathrm{T}}
\newcommand{\herm}{\mathrm{H}}
\newcommand{\forward}[2]{\bm{\phi}_{#1}(#2)}
\newcommand{\backward}[2]{\bm{\beta}_{#1}(#2)}

\newcommand{\R}{\mathbb{R}}
\newcommand{\C}{\mathbb{C}}
\newcommand{\K}{\mathbb{K}}
\newcommand{\N}{\mathbb{N}}
\newcommand{\Z}{\mathbb{Z}}
\newcommand{\PP}{\mathbb{P}}
\newcommand{\Ps}[1]{{\mathfrak{P}\left( #1\right)}}
\newcommand{\As}{\mathcal{A}}
\newcommand{\Hs}{\mathcal{H}}
\newcommand{\Fs}{\mathcal{F}}
\newcommand{\Gs}{\mathcal{G}}
\newcommand{\Ts}{\mathcal{T}}
\newcommand{\Li}{\mathcal{L}}
\newcommand{\RT}{{\mathscr{R}}}
\newcommand{\FT}{{\mathscr{F}}}
\newcommand{\BT}{{\mathscr{B}}}
\newcommand{\HT}{{\mathscr{H}}}

\newcommand{\Real}[1]{{\rm Re}\left\{#1\right\}}
\newcommand{\Imag}[1]{{\rm Im}\left\{#1\right\}}
\newcommand{\Conj}[1]{\overline{#1}}

\newcommand{\E}{\bm{\mathrm{E}}}
\newcommand{\Ex}{{\mathbb{E}}}
\newcommand{\Prb}{\bm{\mathrm{P}}}

\newcommand{\ScPr}[2]{{\left\langle #1,#2 \right\rangle}}
\newcommand{\Norm}[1]{{\left\Vert #1\right\Vert}}
\newcommand{\Abs}[1]{{\left| #1 \right|}}
\newcommand{\Text}[1]{{\hspace{3mm} \text{#1} \hspace{3mm}}}
\newcommand{\brac}[2]{{\left(\frac{#1}{#2}\right)}}
\newcommand{\br}[1]{{\left(#1\right)}}

\newcommand{\Int}[4]{{\int\limits_{#1}^{#2}{#3}\,\mathrm{d}{#4}}}
\newcommand{\Sum}[3]{{\sum\limits_{#1}^{#2}{#3}}}
\newcommand{\Prod}[3]{{\prod\limits_{#1}^{#2}{#3}}}
\newcommand{\SumB}[3]{{\left(\sum\limits_{#1}^{#2}{#3}\right)}}
\newcommand{\ProdB}[3]{{\left(\prod\limits_{#1}^{#2}{#3}\right)}}

\newcommand{\ConvA}[2]{{\xrightarrow[]{#1 \rightarrow #2}}}
\newcommand{\RA}[1]{\overset{#1}{\Rightarrow}}
\newcommand{\LRA}[1]{\overset{#1}{\Leftrightarrow}}
\newcommand{\V}[0]{\hspace{1.2mm}\middle\vert\hspace{1.5mm}}
\newcommand{\D}[0]{\hspace{0.5mm}:\hspace{1.0mm}}

\DeclareMathOperator*{\Argmin}{argmin}
\DeclareMathOperator*{\Argmax}{argmax}
\DeclareMathOperator*{\Arg}{arg}
\DeclareMathOperator*{\BlkDiag}{blkdiag}
\DeclareMathOperator*{\Conv}{conv}
\DeclareMathOperator*{\Cov}{Cov}
\DeclareMathOperator*{\Cone}{cone}
\DeclareMathOperator*{\Diag}{diag}
\DeclareMathOperator*{\DFT}{DFT}
\DeclareMathOperator*{\id}{id}
\DeclareMathOperator*{\Ker}{ker}
\DeclareMathOperator*{\Median}{median}
\DeclareMathOperator*{\Max}{max}
\DeclareMathOperator*{\Mat}{mat}
\DeclareMathOperator*{\Min}{min}
\DeclareMathOperator*{\Mod}{Mod}
\DeclareMathOperator*{\Toep}{\bm{\mathrm{Toep}}}
\DeclareMathOperator*{\Tr}{tr}
\DeclareMathOperator*{\Rk}{rk}
\DeclareMathOperator*{\Rect}{rect}
\DeclareMathOperator*{\Ran}{ran}
\DeclareMathOperator*{\Sign}{sign}
\DeclareMathOperator*{\Sinc}{sinc}
\DeclareMathOperator*{\Supp}{supp}
\DeclareMathOperator*{\Sup}{sup}
\DeclareMathOperator*{\Span}{span}
\DeclareMathOperator*{\Spark}{spark}
\DeclareMathOperator*{\Surf}{surf}
\DeclareMathOperator*{\Vol}{vol}
\DeclareMathOperator*{\Pack}{pack}
\DeclareMathOperator*{\Pb}{\bm{\mathrm{P}}}
\DeclareMathOperator*{\Vectorize}{vec}
\DeclareMathOperator*{\Unif}{Unif}


\newcommand{\for}{\Text{for}}

\newcommand{\tx}{\rm tx}
\newcommand{\rx}{\rm rx}
\newcommand{\mycaption}[2]{\caption[#2]{\emph{#1} -- {#2}}}
\newcommand{\ilcode}[1]{\texttt{#1}}
\newcommand{\TODO}[1]{{\color{red} TODO:#1}}